\documentclass[titlepage=firstcover, captions=tableheading]{scrartcl}
\usepackage{microtype}
\usepackage{amsmath}
\usepackage{polyglossia}
\usepackage{graphicx}
\usepackage{booktabs}
\usepackage{siunitx}
\usepackage{hyperref}
\usepackage{caption}
\usepackage{float}

\begin{document}
\section{Ladungsträger pro Atom}

Die Ladungsträger pro Atom berechnen sich nach:

\begin{displaymath} 
    Z= \frac{n*m}{\rho * N_A}
\end{displaymath}
Dabei ist m die molare Masse von Kupfer, $\rho$ ist die Dichte von Kupfer und N\textsubscript{A} ist die Avogadro-Konstante.
$\rho = 8.96$
$m = 63.546e-3$

Die Rechnung ergibt:
\begin{center}
    \begin{tabular}{l @{${}\pm{}$}ll @{${}\pm{}$}l}\\
        \toprule
        Ladungsträger pro Volumen & Gauß-Fehler & Ladungsträger pro Atom & Gauß-Fehler\\
        \midrule
        $-2.6088847454095256*10^{30}$ &  $2.608884745409526 *10^{27} $  &   -30724.4558 &  30.7244\\
        $-9.375861670676898 *10^{30}$  & $9.375861670676899  *10^{27}$    &  -110418.1579 &  110.4181\\
        $-1.7340192518405735*10^{31}$ &  $1.7340192518405736*10^{28} $ &  -204212.9228 &   04.2129   \\   
        $-2.8998249303089926*10^{31}$ &  $2.899824930308992 *10^{28} $  &  -341508.1603 &  341.5081\\
        $-4.108537223356665 *10^{31}$  & $4.108537223356666  *10^{28}$    &  -483856.4474 &   483.8564\\
        $-5.506911710141514 *10^{31}$  & $5.506911710141516  *10^{28}$     &  -648540.9748 &   648.5409\\
        $-6.565587430250072 *10^{31}$  & $6.5655874302500725 *10^{28}$   &  -773219.6731 &   773.2196 \\
        $-7.809670538066778 *10^{31}$  & $7.8096705380667795 *10^{28}$   &  -919733.5295 &    919.7335 \\ 
        $-8.963167117987643 *10^{31}$  & $8.963167117987643  *10^{28}$    &  -1055579.1425 &  1055.5791 \\
        $-1.001706478691359 *10^{32}$  & $1.001706478691359  *10^{29}$   &  -1179695.1366 &  1179.6951 \\
        \bottomrule
    \end{tabular}
\end{center}

\section{mittlere Flugzeit}

Die mittlere Flugzeit berechnet sich nach:

\begin{displaymath}
    \tau = \frac{2m_0}{ne_0^2*\rho}
\end{displaymath}

Die Rechnung ergibt:
\begin{center}
    \begin{tabular}{l @{${}\pm{}$}lll}\\
    \toprule
    Mittlere Flugzeit & Gauß-Fehler\\
    \midrule
    $-3.036*10^{24}$ &  $3.0362*10^{27}$ \\
    $-8.448*10^{25}$ &  $8.4484*10^{28}$ \\
    $-4.568*10^{25}$ &  $4.5681*10^{28}$ \\
    $-2.731*10^{25}$ &  $2.7316*10^{28}$ \\
    $-1.927*10^{25}$ &  $1.9279*10^{28}$ \\
    $-1.438*10^{25}$ &  $1.4384*10^{28}$ \\
    $-1.206*10^{25}$ &  $1.2064*10^{28}$ \\
    $-1.014*10^{25}$ &  $1.0142*10^{28}$ \\
    $-8.837*10^{26}$ &  $8.8374*10^{29}$ \\
    $-7.907*10^{26}$ &  $7.9076*10^{29}$ \\
    \bottomrule
    \end{tabular}
\end{center}

\subsection{mittlere freie Weglänge}

Die mittlere freie Weglänge berechnet sich dur die Formel:

\begin{displaymath}
    l = -\tau * v
\end{displaymath}

Einsetzen der Werte aus den voherigen Rechnungen liefert:

\begin{center}
    \begin{tabular}{l @{${}\pm{}$}lll}\\
    \toprule
    Mittlere freie Weglänge & Gauß-Fehler\\
    \midrule
    2483.064  & 2.483 \\
     690.926  & 0.6909    \\
     373.584  & 0.3735     \\
     223.393  & 0.223 \\
     157.672  & 0.1576  \\
     117.634  & 0.1176 \\
     98.6663  & 0.0986 \\
     82.9488  & 0.0829 \\
     72.2738  & 0.0722  \\
     64.6699  & 0.0646 \\
    \bottomrule
    \end{tabular}
\end{center}
\end{document}