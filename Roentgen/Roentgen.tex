\section{Theoretische Grundlagen}
\subsection{Emission von Roentgenstrahlung}
Zur Erzeugung von Roentgenstrahlung wird in einer evakuierten Roehre per gluehelektrischem Effekt aus einem Gluehdraht Elektronen emittiert und durch ein E-Feld zur Anode hin beschleunigt. Beim Eintreten der Elektronen in das Anodenmaterial wird Roentgenstrahlung frei in Form des kontinuierlichen Bremsspektrums und der charakteristischen Strahlung des Anodenmaterials.
\subsubsection{Bremsspektrum}
Das Bremsspektrum resultiert aus der Abbremsung des Elektrons beim Eintritt in das Anodenmaterial. Die Energie der ausgesendeten Roentgenquants entspricht exakt dem Energieverlust des Elektrons durch die Abbremsung. Dabei entsteht ein kontinuierliches Spektrum, da das Elektron sowohl einen Teil seiner Energie, als auch die gesamte kinetische Energie abgeben kann. Natuerlicherweise ensteht der Roentgenquant mit der groessten Energie, also der geringsten Wellenlaenge $\lambda_{min}=\frac{hc}{e_0U}$, bei vollstaendiger Abbremsung des Elektrons. Dabei wird die gesamte kinetische Energie $E_{kin}=e_0U$ in Strahlungsenergie $E=h\nu$ umgewandelt.
\subsubsection{Charakteristische Strahlung}
Durch das Auftreffen der Elektronen kann das Anodenmaterial auch ionisiert werden. Dabei werden die Schalenelektronen angeregt und auf eine hoehere Schale gehoben, so dass in einer niedrigeren Schale eine Leerstelle entsteht. Beim Zurueckfallen auf eine niedrigere Schale wird ein Roentgenquant frei, welcher exakt der Energiedifferenz zwischen dem oberen und unteren Niveau entspricht, also $h\nu=E_m-E_n$. Dementsprechend besteht das charakteristische Spektrum aus scharfen Linien. Dieses Linienspektrum ist charakteristisch fuer das Anodenmaterial. Bezeichnet werden diese Linien als $K_\alpha, K_\beta, L_\alpha,...$, wobei der Buchstabe die Schale bezeichnet, auf der der Uebergang endet und der griechische Buchstabe anzeigt, woher das Elektron stammt. Ausserdem ist zu beachten, dass die Huellenelektronen in einem Mehrelektronenatom die Kernladung abschirmen. Dieser Umstand reduziert die Coulomb-Anziehung betraechtlich, was die Bindungsenergie $E_n$ auf der n-ten Schale zu $E_n=-R_{\infty}z^2_{eff}\cdot\frac{1}{n^2}$ ergibt. Dabei wird die effektive Kernladung $z_eff=z-\sigma$ beruecksichtigt mit der Abschirmkonstante $\sigma$ und der Rydbergenergie $R_{\infty}=13.6eV$. Die Abschirmkonstante ist fuer jedes Elektron verschieden.
\subsection{Absorption}
Die Absorption von Roentgenstrahlung unter 1 MeV wird dominiert vom Comptoneffekt und dem Photoeffekt. Die Absorption nimmt mit hoeherer Energie ab, steigt jedoch sprunghaft an, wenn die Energie der Roentgenquanten grade groesser als die Bindungsenergie einer naechsten Schale ist. Die Lage dieser Absorptionskanten $h\nu_{abs}=E_n-E_{\infty}$ entspricht nahezu genau der Bindungsenergien der Elektronen. Diese Kanten werden ebenso als K-, L-, M-,...Absorptionskante bezeichnet. Diese Kanten werden aufgrund der Feinstruktur noch in kleinere Teile unterteilt, etwa die L-Kanten ($L_I, L_{II}, L_{III}$). Dabei muss zur Berechnung der Bindungsenergie $E_{n,j}$ die Sommerfeldsche Feinstrukturformel benutzt werden. Sie ergibt sich zu 
\begin{equation*}
    E_{n,j}=-R_\infty(z^2_{eff,1}\cdot\frac{1}{n^2}+\alpha^2z^4_{eff,2}\cdot\frac{1}{n^3}(\frac{1}{j+\frac{1}{2}}-\frac{3}{4n})).
\end{equation*}
Dabei ist $R_\infty$ die Rydbergenergie, $\alpha$ die Sommerfeldsche Feinstrukturkonstante, $z_{eff}$ die effektive Kernladungszahl, j der Gesamtdrehimpuls und n die Hauptquantenzahl. Die Abschirmkonstante $\sigma_K$ ergibt sich fuer die K-Schale (n=1) zu $\sigma_K=Z-\sqrt{\frac{E_K}{R_\infty}-\frac{\alpha^2Z^4}{4}}$.
Durch die Unmoeglichkeit der Aufloesung der $L_I$- und $L_{II}$-Kante in diesem Versuch vereinfacht sich die Berechnung der Abschirmkonstante fuer $\sigma_L$ zu $\sigma_L=Z-\Bigg(\frac{4}{\alpha}\sqrt{\frac{\Delta E_L}{R_\infty}}-\frac{5\Delta E_L}{R_\infty}\Bigg)^{\frac{1}{2}}\Bigg(1+\frac{19\alpha^2\cdot\Delta E_L}{32R_\infty}\Bigg)^{\frac{1}{2}}$.
Die Energie E der resultierenden Roentgenquanten kann ueber die Braggsche Reflexion untersucht werden. Dazu faellt der Roentgenquant auf ein dreidimensionales Gitter (z.B. ein LiF-Kristall). An jeder Gitterebene wird der Quant gebeugt. Sie interferieren miteinander, wobei sie konstruktiv interferieren beim Glanzwinkel $\theta$. Die Braggsche Bedingung lautet $2dsin\theta=n\lambda$ mit der Gitterkonstante d, der Wellenlaenge $\lambda$ und der Beugungsordnung n.
\section{Aufbau}
\begin{figure}[H]
    \centering
    \captionsetup{justification=centering}
    \includegraphics[height=7cm]{"Aufbau_Roentgen.png"}
    \captionbelow{Der beschriftete Aufbau des Versuchs\\ Aus: \cite{Anl}}
    \label{Fig:Aufbau}
\end{figure}
Im Wesentlichen besteht der Versuch aus einer Kupferroentgenroehre, einem LiF-Kristall und einem Geiger-Mueller-Zaehlrohr. Bedient werden kann das Geraet mit einem Computer oder manuell. In diesem Fall ist die Auslesung der Daten in Gaenze mit dem Rechner zu empfehlen. Dazu wird im Programm "measure" das Roentgengeraet angewaehlt. Die Messart muss Auf Spektren gestellt werden. Am Geraet selber soll die Beschleunigungsspannung $U_B=35kV$ sein mit einem Emissionsstrom von $I=1mA$. Der Drehmodus, der abgetastete Kristallwinkel sowie die Integrationszeit koennen im Programm ausgewaehlt werden.
\section{Durchfuehrung}
\subsection{Ueberpruefung der Bragg-Bedingung}
Die Bragg-Bedingung wurde ueberprueft, indem der Kristall auf einen Winkel von $\theta=14^\circ$ eingestellt wurde und mit dem Geiger-Mueller-Zaehlrohr in Schritten von $\Delta\theta_{GM}=0.1^\circ$ in einem Winkelbereich von $26^\circ$ bis $30^\circ$ abgefahren wurde. Die Integrationszeit betraegt $t=5s$.  
\subsection{Analyse des Emissionsspektrums}
Die Messung des Emissionsspektrums erfolgt mit einer Integrationszeit von $\Delta t=10s$ und einer Schrittweite von $\Delta\theta=0.1^\circ$. Daher kann auch kein Detailspektrum der $K_\alpha$- bzw. $K_\beta$-Linie aufgezeichnet werden. Aufgezeichnet werden die Daten mit einer Computer zur spaeteren Benutzung. Daraus wird schliesslich auch die Halbwertsbreite, das Aufloesungsvermoegen $A=\frac{E_K}{\Delta E_{FWHM}}$ und die Abschirmkonstanten berechnet.
\subsection{Analyse der Absorptionsspektren}
Zur Bestimmung der Absorptionsspektren werden zwischen den LiF-Kristall und das Geiger-Mueller-Zaehlrohr verschiedene Absorber eingesetzt und mit einer Integrationszeit von $t=20s$ das Spektrum aufgenommen. Daraus wird die Absorptionsenergie $E_K$ anhand der Lage der Absorptionskante bestimmt. Ausserdem werden das Intensitaetsmaximum $I^{max}_K$ und das Intensitaetsminimum $I^{min}_K$ der K-Kante abgelesen. An der Mitte der Kante wird schliesslich der Winkel $\theta$ abgelesen und daraus mit Hilfe der Braggschen Bedingung die Absorptionsenergiem $E_{K,abs}$ berechnet. Aus diesen berechneten Absorptionsenergien lassen sich dann die Abschirmkonstanten bestimmen. Schliesslich soll mit Hilfe der Gleichung $E_K=Rh(z-\sigma)^2$ die Rydbergfrequenz R und die Rydbergenergie $R_\infty=h\cdot R$ berechnet werden. 