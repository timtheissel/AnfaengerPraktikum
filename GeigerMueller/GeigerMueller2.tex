\documentclass[titlepage=firstcover, captions=tableheading]{scrartcl}
\usepackage{microtype}
\usepackage{amsmath}
\usepackage{polyglossia}
\usepackage{graphicx}
\usepackage{booktabs}
\usepackage{siunitx}
\usepackage{hyperref}
\usepackage{caption}
\usepackage{float}
\setdefaultlanguage{german}
\title{V703 Das Geiger-Mueller-Zaehlrohr}
\author{
Connor Magnus Böckmann \\ email: \href{mailto:connormagnus.boeckmann@tu-dortmund.de}{connormagnus.boeckmann@tu-dortmund.de}
\and Tim Theissel \\ email: \href{mailto:tim.theissel@tu-dortmund.de}{tim.theissel@tu-dortmund.de}}
\begin{document}
\maketitle
\newpage
\tableofcontents
\newpage
\section{Zielsetzung}
In diesem Versuch wird das Geiger-Mueller-Zaehlrohr untersucht, welches ionisierende Strahlung detektieren und messen kann. Es ist in der Lage einen elektrischen Impuls auszugeben, sollte ein \alpha- oder \beta-Teilchen im Inneren detektiert werden. Dieser Impuls kann dann von einem Impulszaehler gezaehlt werden und die pro Zeit- und Flaecheneinheit einfallenden Teilchen bzw. Quanten messen und dadurch die Intensitaet bestimmen. 
\section{Theoretische Grundlagen}
Der prinzipielle Aufbau des Zaehlrohrs ist in ?? zu sehen. Das Zaehlrohr besteht aus einem Stahlzylinder mit dem Radius $r_k$, welcher die Kathode darstellt. In seinem Inneren befindet sich ein Draht, welcher die Anode (Radius $r_a$) darstellt. 
\end{document}
