\documentclass[titlepage=firstcover, captions=tableheading]{scrartcl}
\usepackage{microtype}
\usepackage{amsmath}
\usepackage{polyglossia}
\usepackage{graphicx}
\usepackage{booktabs}
\usepackage{siunitx}
\usepackage{hyperref}
\usepackage{caption}
\usepackage{float}
\setdefaultlanguage{german}
\title{V703 Das Geiger-Müller-Zählrohr}
\author{
Connor Magnus Böckmann \\ email: \href{mailto:connormagnus.boeckmann@tu-dortmund.de}{connormagnus.boeckmann@tu-dortmund.de}
\and Tim Theissel \\ email: \href{mailto:tim.theissel@tu-dortmund.de}{tim.theissel@tu-dortmund.de}}
\begin{document}
\maketitle
\newpage
\tableofcontents
\newpage
\section{Zielsetzung}
In diesem Versuch wird das Geiger-Mueller-Zaehlrohr untersucht, welches ionisierende Strahlung detektieren und messen kann. Es ist in der Lage einen elektrischen Impuls auszugeben, sollte ein \alpha- oder \beta-Teilchen im Inneren detektiert werden. Dieser Impuls kann dann von einem Impulszaehler gezaehlt werden und die pro Zeit- und Flaecheneinheit einfallenden Teilchen bzw. Quanten messen und dadurch die Intensitaet bestimmen. 
\section{Theoretische Grundlagen}
\subsection{Aufbau und prinzipielle Funktionsweise}
Der prinzipielle Aufbau des Zaehlrohrs ist in \ref{Fig:Aufbau} zu sehen.
\begin{figure}[H]
    \centering
    \includegraphics{"Aufbau_Geiger.png"}
    \caption{Aufbau eines Geiger-Mueller-Zaehlrohrs mit Endfenster\\ Aus: Anleitung V703 Seite 220}
    \label{Fig:Aufbau}
\end{figure}
 \noindent Das Zaehlrohr besteht aus einem Stahlzylinder mit dem Radius $r_k$, welcher die Kathode darstellt. In seinem Inneren befindet sich ein Draht, welcher die Anode (Radius $r_a$) darstellt. Der Zylinder ist versiegelt und mit einem Gasgemisch gefuellt. Durch das Anlegen einer ausseren Spannung U (ca. 300 bis 2000V) bildet sich ein elektrisches, radialsymmetrisches Feld aus. Die Feldstaerke betraegt im Abstand r von der Mittelachse $E(r)=\frac{U}{r\ln(\frac{r_k}{r_a})}$.
 Wenn ein geladenes Teilchen in das Zaehlrohr eintritt, wird es in dem E-Feld beschleunigt. Diese Beschleunigung steigt bei Annaeherung an den Draht mit $\frac{1}{r} \text{mit} (r_a < r < r_k)$. Theoretisch kann diese als beliebig gross werden, wenn nur der Drahtradius $r_a$ hinreichend klein gewaehlt wird.\\
 Sollte nun ein geladenes Teilchen ins Zaehlrohr gelangen, bewegt es sich so lange durch den Gasraum bis seine Energie durch Ionisation aufgebraucht ist. Da eine Ionenpaarbildung im Mittel nur etwa 26eV benoetigt, gegenueber einer Teilchenenergie von etwa 100keV, ist die Anzahl der positiven Ionen und Elektronen proportional zur Energie des ins Zaehlrohr eingefallenen Teilchens. Die angelegte Spannung hat nun einen grossen Einfluss auf die nach der Primaerionisation stattfindenden Prozesse. Eine Visualisierung findet sich in dem Diagramm in \ref{Fig:Spannung}.
 \begin{figure}[H]
    \centering
    \includegraphics[width=0.6\textwidth]{"Spannung_Geiger.png"}
    \caption{Anzahl der erzeugten Ionenpaare gegenueber der angelegten Spannung\\ Aus: Anleitung V703 Seite 221}
    \label{Fig:Spannung}
\end{figure}
\noindent Das Diagramm laesst sich dabei in verschiedene Bereiche einteilen. Zu Beobachten ist zum Beispiel der Bereich fuer sehr kleine Spannungen. Dabei werden die Ionen nicht stark genug beschleunigt, um den Draht zu erreichen. Viele rekombinieren sich vorher bereits (Bereich I). Je groesser die Spannung wird, desto groesser wird die Wahrscheinlichkeit, dass die Ionen sich nicht vor Erreichen des Drahtes rekombinieren. Es erreichen somit quasi alle Elektronen den Draht. In diesem Bereich ist der Ionisationsstrom zwischen Kathodenzylinder und Anodendraht proportional zur Energie und Intensitaet der einfallenden Teilchen. Dieser Zustand nennt sich Ionisationskammer und ist in Bereich II zu sehen. Zur tadellosen Funktion wird hierbei aber eine grosse Strahlungsintensitaet benoetigt. Bei noch hoerer Spannung wird die Feldstaerke in Drahtnaehe so gross, dass die entstehenden Elektronen zwischen zwei Zusammenstoessen genuegend Energie aufnehmen, um ihrerseits andere Gasteilchen ionisieren zu koennen. Dieser Vorgang nennt sich Stossionisation. Unter ausreichend grosser Spannung koennen die freigesetzten Elektronen selbst andere Teilchen ionisieren. Es bildet sich eine so genannte Townsend-Lawine. Hier ist nun auch die pro Teilchen einfallende Ladung gross genug, um sie als Ladungsimpuls zu messen. Durch die vorhandene Proportionalitaet zwischen der Ladung Q und der Teilchenenergie, laesst sich am Ladungsimpuls ein Mass fuer die Teilchenenergie festmachen. Eine Apparatur, die sich diese Proportionalitaet zu Nutze macht, nennt sich Proportionalitaetszaehlrohr (Bereich III).
Bei wiederum hoeheren Spannungen U ist die Ladung nicht mehr proportional zur Primaerionisation. Dieser Bereich nennt sich Ausloesebereich (Bereich IV). In diesem Bereich arbeitet ein Geiger-Mueller-Zaehlrohr unter normalen Bedingungen. Statt einer lokalen Elektronenlawine breitet sich die Lawine nicht nur in radialer Richtung, sondern auch entlang des Drahtes aus. Ausgeloest wird dieser Prozess durch die in der primaeren Lawine entstandenen UV-Photonen, welche durch die Anregung von Argon-Atomen im Fuellgas freisetzen. Diese Photonen koennen sich auf Grund ihrer Ladungsneutralitaet auch senkrecht zum E-Feld bewegen. Ihre Energie, welche sie durch Zusammenstoesse abgeben koennen, bildet den Grundstein fuer neue Elektronenlawinen im vollstaendigen Zaehlvolumen. Die am Draht gemessene Ladung haengt dann nicht mehr von der ersten Ionisation ab, sondern bloss vom Volumen des benutzten Zaehlrohrs und der angelegten Spannung. Bei dieser Spannung kann das Zaehlrohr nur noch als Intensitaetsmessgeraet benutzt werden. Eine Energiemessung ist nicht mehr moeglich. Dafuer kann die freigesetzte Ladung eines einfallenden Teilchens nun auf Grund ihrer nun relativ grossen Groesse mit geringem elektronischem Aufwand gut gemessen werden. Der so genannte Ausloesebereich beginnt in \ref{Fig:Spannung} dort, wo \alpha- und \beta-Kurve in einander uebergehen. Dort ist der Ladungsimpuls abgekoppelt vom Ionisationsvermoegen der einfallenden Strahlung.
\subsection{Totzeit und Nachentladungen}
Die entstehenden positiven Ionen haben eine bedeutend hoehere Masse als die Elektronen, weshalb sie deutlich langsamer abwandern. Sie halten sich also laenger im Raum zwischen Anode und Kathode. Aus diesem Grund bilden sie eine temporaere, radialsymmetrische, positive Raumladung aus. Diese wird auf Grund der zylindrischen Form des Rohrs auch Ionenschlauch genannt. Dadurch wird kurzzeitig fuer eine Zeit T die Feldstaerke in Drahtnaehe soweit herab, dass praktisch keine Stossionisation mehr moeglich ist. In dieser Zeit ist das Zaehlrohr nicht in der Lage eintreffende Teilchen zu detektieren, weshalb man diese Zeit auch Totzeit T nennt. Der Zustand loest sich durch Wandern der positiven Ladung in Richtung des Mantels auf. Die normale Feldstaerke stellt sich wieder her nachdem die Ionen vollstaendig neutralisiert wurden. Diese Zeit der Wiederherrstellung des Normalzustandes nennt sich treffenderweise Erholungszeit $T_E$, welche sich an die Totzeit anschliesst, bis das Zaehlrohr wieder unter Normalbedingungen detektieren kann.\\
Auf der Manteloberflaeche koennen die auftreffenden Ionen, durch ihre Neutralisationsenergie, Elektronen freisetzen. Diese Sekundaerelektronen werden zum Messdraht hin beschleunigt durch das E-Feld und sorgen auf diesem dann fuer ein erneutes Ausloesen des Messgeraets. Dadurch kann durch ein einzelnes eintreffendes Teilchen mehrere Entladungen am  Zaehlrohrdraht hervorgerufen werden. Diese zusaetzlichen Entladungen nennen sich Nachentladungen. Der zeitliche Abstand der Nachentladungen entspricht der Laufzeit $T_L$ der Elektronen von der Zylinderwand zum Draht. Diese Nachentladungen sind sehr unerwuenscht, da sie das Vorhandensein von ionisierender Strahlung vortaeuschen. Sie sollten also so weit wie moeglich unterbunden werden, was durch eine kleine Beimengung von Alkoholdaempfen zum Fuellgas gut gelingt, denn diese Alkoholdämpfe werden von den Edelgasionen ionisiert und wandern dann zur Kathode. Das Gute daran ist, dass die Alkoholionen an der Kathode keine Elektronen emittieren. Es gibt also keine weiteren Entladungen und die Messung wird nicht durch weitere Elektronen, die von der Kathode aus kommen beeinflusst.
\subsection{Charakteristik des Zaehlrohrs}
Die Charakteristik eines Geiger-Mueller-Zaehlrohr wird gegeben durch eine Auftragung der eintreffenden Teilchenzahl N gegen die Betriebsspannung U. In etwa sieht eine Charakteristik so aus:
\begin{figure}[H]
    \centering
    \includegraphics[width=0.6\textwidth]{"Charakteristik_Geiger.png"}
    \caption{Charakteristik eines Geiger-Mueller-Zaehlrohres bei konstanter Strahlungsintensitaet \\ Aus: Anleitung V703 Seite 224}
    \label{Fig:Charakteristik}
\end{figure}
\noindent Der Ausloesebereich beginnt hierbei etwa bei der Spannung $U_E$. Der linear steigende, abgeflachte Teil nennt sich Plateau. Die Steigung in diesem Bereich der Kurve ist bestenfalls moeglichst klein. Idealerweise waere diese also null, was aber nicht erreichbar ist. Diese Unmoeglichkeit ruegt von vereinzelten Nachentladungen her, welche trotz des Alkoholdampfes vorkommen. Am Ende des Plateaus nimmt die Zahl der Nachentladungen extrem zu, was den starken Anstieg erklaert. Je hoeher die Qualitaet des Zaehlrohres, desto laenger und flacher ist das Plateau.\\
Nach Ende des Plateaus geht das Zaehlrohr in den BEreich der selbststaendigen Gasentladung ueber. Dort zuendet ein einzelnes ionisierendes Teilchen eine Dauerentladung. Dies ist sehr schaedlich fuer das Zaehlrohr auf Grund der hohen Stromdichten und wird das Messgeraet schnell zerstoeren, sollte also auf jeden Fall vermieden werden.
\subsection{Ansprechvermoegen eines Zaehlrohres}
Das Ansprechvermoegen eines Zaehlrohres ist ein Mass dafuer wie wahrscheinlich es ist, dass ein einfallendes Teilchen auch tatsächlich detektiert wird. Fuer \alpha- und \beta-Teilchen ist das Ansprechvermoegen nahezu 100\%. Allerdings muss dazu natuerlich gewaehrleistet sein, dass die Teilchen auch tatsaechlich ins Innere des Zaehlrohres gelangen. Da diese Teilchen sehr wahrscheinlich mit Materie wechselwirken, werden sie im Metallmantel des Zaehlrohres absorbiert. Aus diesem Grund gibt es Endfensterzaehlrohre, welche am Ende des Volumens von einer duennen Mylar-Folie verschlossen werden. Diese kann selbst von \alpha-Teilchen durchdrungen werden. Ein solches Endfensterzaehlrohr ist in \ref{Fig:Aufbau} dargestellt.\\
Bei Photonen hingegen liegt das Ansprechvermoegen bei geringen Werten von etwa 1\%. Das folgt aus der geringen Wechselwirkungswahrscheinlichkeit von Photonen mit hoher Energie mit Materie. Die Nutzung eines Geiger-Mueller-Zaehlrohres zur Detektion von \gamma-Strahlung ist also nur bei hohen Intensitaeten dieser moeglich und sinnvoll. 
\section{Experimementelle Untersuchungen am Zaehlrohr}
Fuer die gemachten Messungen wurde die im folgende skizzierte Messapparatur verwendet.
\begin{figure}[H]
    \centering
    \includegraphics[width=0.6\textwidth]{"Apparatur_Geiger.png"}
    \caption{Aufbau der Messapparatur \\ Aus: Anleitung V703 Seite 226}
    \label{Fig:Apparatur}
\end{figure}
\noindent Ueber den Widerstand R fliesst die am Zaehldraht gesammelte Ladung Q ab. Ein Spannungsimpuls wird erzeugt. Nach Auskoppelung im Kondensator C wird dieser verstaerkt und im Zaehlgeraet gezaehlt. Auf dem Schirm eines Oszillographen kann dieser Impuls dann sichtbar gemacht werden.
\subsection{Aufnahme der Charakteristik}
Ein \beta-Strahler wird vor das Einlassfenster des Messgeraets gestellt. Dabei wird die Zaehlrate gemessen und in Abhaengigkeit von der Betriebsspannung gesetzt. Zu Beachten ist dabei, dass die Zaehlrate 100 pro Sekunde nicht ueberschreitet. Hierbei sollte den Moeglichkeiten entsprechend sehr genau gemessen werden, da die Plateausteigeung sowieso schon sehr gering ist. Desweiteren darf beim verwendeten Zaehlrohr die Betriebsspannung nicht ueber 700 Volt hinausgehen. Andernfalls geraet das Zaehlrohr in den Bereich der bereits genannten selbststaendigen Gasentladung. Bei Missachtung folgt schnell die Zerstoerung des Messgeraetes. 
\subsection{Sichtbarmachung von Nachentladungen}
Es sollen die Nachentladungen mit einem Oszilloskop qualitativ sichtbar gemacht werden. Dazu wird die Intensitaet der Strahlenquelle soweit gesenkt, dass auf dem Bildschirm des Oszilloskop waehrend der Laufzeit des Strahls von links nach recht kein weiterer Impuls zu sehen ist. Die Zaehlrohrspannung muss so gering sein, dass die Wahrscheinlichkeit fuer NAchentladungen sehr gering ist. Hier werden 350V verwendet. Nun wird die Zaehlrohrspannung schrittweise bis au 700V erhoeht. Der zeitliche Abstand zwischen der Primaer- und Nachentladung wird gemessen.
\subsection{Bestimmung der Totzeit mit der Zwei-Quellen-Methode}
Die Totzeit T sorgt fuer eine registrierte Impulsrate $N_r$, welche immer kleiner ist als die tatsaechliche Anzahl der eingedrungenen Teilchen $N_W$. Wenn $N_r$ Impulse pro Zeiteinheit registriert werden, so ist das Geraet fuer den Bruchteil $TN_r$ nicht in der Lage Teilchen zu regestrieren, weshalb es nur im Bereich $1-TN_r$ messbereit ist. Die wahre Impulsrate ist also:
\begin{equation}\label{2}
    N_W=\frac{Impulsrate}{Messzeit}=\frac{N_rt}{(1-TN_r)t}=\frac{N_r}{1-TN_r}
\end{equation}
Nun kann mit dieser Formel die Totzeit bestimmt werden. Dazu werden zwei Radioaktive Praeparate benoetigt. Es wird dazu zuerst die Zaehlrate $N_1$ des Messgeraets mit dem ersten Praeparat gemessen. Danach wird ein zweites Praeparat hinzugenommen ohne die Postion des ersten Praeparates zu veraendern. Die Zaehlrate $N_{1+2}$ wird gemessen. Daraufhin wird das erste Praeparat entfernt und die Messrate $N_2$ gemessen.\\
Ohne Totzeit waere 
\begin{equation}
    N_{1+2}=N_1+N_2\nonumber
\end{equation}
Stattdessen laesst sich aber beobachten, dass $N_{1+2}<N_1+N_2$ ist. Gemaess \ref{2} gilt fuer die von den Praeparaten emittierten und ins Zaehlrohr eingedrungenen Teilchen
\begin{align}
    N_{W_1}&=\frac{N_1}{1-TN_1}\nonumber\\
    N_{W_2}&=\frac{N_2}{1-TN_2}\nonumber\\
    N_{W_{1+2}}&=\frac{N_{1+2}}{1-TN_{1+2}}\nonumber
\end{align}
Ausserdem ist $N_{W_{1+2}}=N_{W_1}+N_{W_2}$. Daraus ergibt sich dann
\begin{equation}\label{Totzeit}
    \frac{N_{1+2}}{1-TN_{1+2}}=\frac{N_1}{1-TN_1}+\frac{N_{1+2}}{1-TN_{1+2}}
\end{equation}
Da die Groessen $N_1$, $N_2$ und $N_{1+2}$ bekannt sind kann aus \ref{Totzeit} die Totzeit T bestimmt werden. T laesst sich naeherungsweise schreiben als 
\begin{align}
    T\approx\frac{N_1+N_2-N_{1+2}}{2N_1N_2}
\end{align}
\subsection{Messung der freigesetzten Ladungsmenge pro Teilchen}
Wie in \ref{Fig:Apparatur} kann mit einem empfindlichen Strommessgeraet der mittlere Strom des Zaehlrohrs gemessen werden. 
\begin{align}
    \bar{I}:=\frac{1}{\tau}\int_0^\tau \frac{U(t)}{R} \text{dt}\nonumber\\
    \text{(\tau>>T)}\nonumber
\end{align}
Daraus laesst sich nun die pro eindringendes Teilchen freigesetzte Ladungsmenge bei bekannter Impulszahl pro Zeiteinheit berechnen. Es gilt der Zusammenhang 
\begin{align}
    \bar{I}=\frac{\Delta Q}{\Delta t}Z\nonumber
\end{align}
Dabei ist \Delta Q die pro Zeitintervall \Delta t transportierte Ladungsmenge, wenn Z die Anzahl der registrierten Teilchen ist.

\section{Auswertung}

\subsection{Anmerkung zur Fehlerrechnung}

Die Fehlerrechnungen in dieser Auswertung werden mit pythons 'ufloats' durchgeführt. Diese beziehen direkt den Fehler der eingesetzten Wert mit ein und berechnet den Gauß-Fehler. Die Gaußsche Fehlerfunktion sieht folgendermaßen aus:

\begin{displaymath}
    \Delta f = \sqrt{\left(\frac{\partial f}{\partial x}\right)^2 (\Delta x)² +
                     \left(\frac{\partial f}{\partial y}\right)^2 (\Delta y)² + ... +
                     \left(\frac{\partial f}{\partial z}\right)^2 (\Delta z)²
    }
\end{displaymath}


\subsection{Bestimmung des Untergrundes}

Zur Bestimmung des Untergrundes wurden 7 Messungen durchgeführt.
Für die Messungen wurde das Zählrohr eingeschaltet ohne eine Probe auf das Zählrohr zu stecken.
Die Messungen liefen jeweils 300s.

\noindent Die enstanden Werte dieser Messungen sind in Tabelle \ref{tab:1} zu finden.

\noindent Aus diesen Werten wird nun ein arithmetischer Mittelwert berechent.
Dieser wird dann noch auf ein Zeitintervall von 30s/15s skaliert, damit diese Mittelwerte in den Folgenden Aufgaben ohne weitere Anpassung verwendet werden können.

\noindent Es ergibt sich eine Untergrundrate von $13.9 \approx 14$ [Imp/30s] bzw. $6.95 \approx 7$ [Imp/15s]

\subsection{Bestimmung der Halbwertszeit von Vanadium}

In diesem Teil wurde  die aktivierte Vanadiumprobe umgehend auf das Geiger-Müller-Zählrohr gesteckt und es wurden die Impulse in einem Zeitintervall von 30s abgelesen.
Diese Messdaten sind in Tabelle \ref{tab:2} zu finden. 
Von diesen Werten wird der im letzten Teil berechnete Mittelwert abgezogen.
Anschließend werden die entstehenden Werte mit Fehler in eimen halblogarithmischen Diagramm dargestellt.

\begin{figure}[H]
    \centering
    \includegraphics{"22.png"}
    \caption{Bestimmung der Halbwertszeit von Vanadium}
    \label{Fig:Halbwertszeit}
\end{figure}

\noindent Als Ausgleichsfunktion wurde das Zerfallsgesetz verwendet.
Dieses wurde für die Bestimmung der Halbwertszeit dann noch umgestellt.
\begin{align}
    N(t) &= N_0 * e^{-\lambda t} \nonumber \\
    \frac{1}{2} N_0 &= N_0 * e^{-\lambda T} \nonumber \\
    T &= \frac{\ln(2)}{\lambda} \nonumber
\end{align}

\noindent Für die erste Ausgleichsrechnung wurden alle gemessenen Werte verwendet.
Mit der Python-Funktion "scipy.optimize curve fit" ergibt sich:
\begin{align}
    N_0 &=  208.761825 \pm 6.069003 \; [\text{Imp}] \nonumber \\
    \lambda &= 0.002657 \pm 0.000106 \; [1/s] \nonumber
\end{align}

\noindent Daraus lässt sich die Halbwertszeit berechnen zu:
\begin{displaymath}
    T_1 = (261 \pm 10) s 
\end{displaymath}

Diese Rechnung wurde, wie auch alle weiteren Rechnungen, in python mit 'ufloats' durchgeführt. Diese beziehen den Fehler der Werte, welche in die Formel eingesetzt werden, direkt mit ein und der dazugehörige Gauß-Fehler wird automatisch berechnet. 

\noindent Bei der zweiten Ausgleichrechnung werden nun nur die Werte bis zur doppelten Halbwertszeit für die Ausgleichsrechnung verwendet.
Also alle Messwerte bis zu einer Zeit von maximal 520s.
Für die Parameter ergeben sich diesmal die Werte:

\begin{align}
    N_0 &=  218.126931 \pm 7.601301 \; [\text{Imp}] \nonumber \\
    \lambda &= 0.002937 \pm 0.000170 \; [1/s] \nonumber 
\end{align}

\noindent Die etstehende Halbwertszeit hat nun den Wert:

\begin{displaymath}
    T_2 = (236 \pm 14) s 
\end{displaymath}

\subsection{Halbwertszeit von Rhodium}

Analog zu der Vorgehensweise bei Vanadium wird auch bei der Halbwertszeit zunächst ein halblogarithmisches Diagramm erstellt.
Diesmal wird die Ungergrundrate wieder angepasst, da das Zeitintervall nun nur 15s beträgt.
Die gemessenen Werte für Rhodium sind in Tabelle \ref{tab:3} zu sehen.

\noindent Das entstehende Diagramm mit dem angepassten Untergrund und dem Fehler $\sqrt{N}$ sieht folgendermaßen aus:

\begin{figure}[H]
    \centering
    \includegraphics{"3.png"}
    \caption{Bestimmung der Halbwertszeit von Rhodium}
    \label{Fig:HalbwertszeitRh}
\end{figure}

\noindent In diesem Diagramm sind bereits die benötigten Ausgleichsrechnunhen eingetragen.
Zu erkennen sind die verschiedenen Steigungen der beiden Zerfälle.
Für die Ausgleichsrechnung des langsamen Zerfalls muss der Zeitpunkt bestimmt werden, ab dem nur noch der langlebige Zerfall stattfindet.
Dieser Zeitpunkt wurde auf $t* \approx 250s$ geschätzt.
Um sicher zu sein, dass der exponentielle Zerfall möglichst keinen Einfluss mehr hat wurden bei der Ausgleichsrechnung alle Daten verwendet, die nach 300s oder später aufgenommen wurden.

\noindent Mit dem Zerfallsgesetz als Ausgangsfunktion ergeben sich folgende Parameter:

\begin{align}
    A_0 &= 72.317788 \pm 16.092044 \; [\text{Imp}] \nonumber  \\
    \lambda &= 0.002646 \pm 0.000501 \; [1/s] \nonumber
\end{align}
    


\noindent Mit diesen Parametern entsteht die langlebige Ausgleichsgerade. 
Dann wird das Intervall betrachtet, in dem der kurzlebige Zerfall stattfindet. 
Um sicher zu gehen, dass der Einfluss des kurzlebigen Zerfalls wird das Intervall [0s;165s] betrachtet.
Dann werden auf dem Intervall für alle Zeitpunkte des Intervall die zugehörigen Werte der Ausgleichsfunktion des langlebigen Zerfalls berechnet.
Anschließend werden von den Werten, die im Diagramm aufgetragen sind die entsprechenden werte der Ausgleichsfunktion abgezogen.
Mit den entstehenden Werten wird nun die Ausgleichsrechnung für den kurzlebigen Zerfall durchgeführt. 

\noindent Mit dem Zerfallsgesetz als Ausgangsfunktion ergeben sich folgende Parameter:

\begin{align}
    A_0 &= 772.987833 \pm 20.717019 \; [\text{Imp}] \nonumber  \\
    \lambda &= 0.015228 \pm 0.000551 \; [1/s] \nonumber
\end{align}

\noindent Der dritte Graph im Diagramm ist die Summme aus beiden Ausgleichsfunktionen.

\noindent Nun lassen sich die Verschiedenen halbwertszeiten berechnen

\begin{align}
    T &= \frac{\ln(2)}{\lambda} \nonumber
\end{align}

\noindent Mit den Parametern lassen sich folgende Halbwertszeiten berechnen:

\begin{tabular}{ll@{${}\pm{}$}l}
    \toprule
    Halbwertszeit & Wert [s] & Unsicherheit [s] \\
    \midrule
    $T_{lang}$ & 260 & 50 \\
    $T_{kurz}$ & 45.5 & 1.6 \\
    \bottomrule   
\end{tabular}

\section{Diskussion}

\subsection{Vanadium}

Die berechneten Halbwertszeiten für Vanaium betragen: 
\begin{align}
    T_1 &= 261 \pm 10 s \nonumber \\
    T_2 &= 236 \pm 14 s \nonumber
\end{align}

\noindent Dabei ist T\textsubscript{1} die aus allen Werten berechnete Halbwertszeit und bei T\textsubscript{2} wurden die letzten sehr kleinen Werte weggelassen, da sie so klein sind, dass sie in den Untergrund übergehen. 

\noindent Der Literaturwert beträgt $T_{lit}=224.4s$.
Die prozentualen Abweichungen der Messwerte zu dem Literaturwert betragen für T\textsubscript{1} zu T\textsubscript{lit} $\approx 16.31 \%$ und für T\textsubscript{2} zu T\textsubscript{lit} $\approx 5.17 \%$

\noindent Es fällt auf, dass T\textsubscript{2} deutlich näher an der tatsächlichen Halbwertszeit liegt als T\textsubscript{1}. 
Daran ist deutlich zu erkennen, dass die kleinen Werte zu späteren Zeitpunkten die Rechnung negativ beeinflussen. Das liegt vor allem daran, dass diese Werte viel schwieriger zu messen sind. Es wird kaum noch Strahlung von der Probe ausgesendet. Damit ist kaum noch zu unterscheiden ob der gemessene Wert nicht einfach daran liegt, dass die Untergrundrate den Mittelwert überschritten hat. Das Problem existiert für alle anderen Werte zwar auch, aber weil dort mehr Strahlung von der Probe kommt und selbst eine hohe Untergrundrate daher einen geringeren Einfluss hat. Dennoch liegt die restliche Abweichung vom Literaturwert vermutlich auch an diesem Problem, dass die Untergrundrate nicht immer gleich ist. 
Die geringere Abweichung vom Literaturwert der Methode zur Berechnung von T\textsubscript{2} lässt keinen Zweifel zu, dass diese Methode die bessere Wahl, da sie genauere Ergebnisse liefert. 

\noindent Es gibt auch noch einen kleineren Ablesefehler. Allerdings liegt die Abweichung vom Literaturwert im Rahmen gewöhnlicher Messunsicherheiten. 

\subsection{Rhodium}

\noindent Die berechnete Halbwertszeit für den langlebigen Zerfall von Rhodium beträgt: $260 \pm 5s$.
Der Literaturwert dieser Halbwertszeit liegt bei 260.4s.

\noindent Diese Abweichung von gerade einmal 0.4s oder $\approx 0.15\%$ ist sehr gering. Das spricht für eine sehr gute Durchführung des Versuchs. Durch gewöhnliche unvermeidbare Messfehler ergibt sich noch eine Unsicherheit der berechneten Zeit. Bei diesem Experiment wurden allerdings sämtliche Fehlerquellen minimiert. 

\noindent Für den kurzlebigen Zerfall gibt es folgende zu Vergleichende Werte:
\begin{align}
    T_{kurz} &= 45.5 \pm 1.6 \nonumber\\
    T_{lit} &= 42.3  \nonumber
\end{align}

\noindent Der Wert von T\textsubscript{kurz} weist zwar mit $\approx 7.56\%$ eine größere Abweichung als T\textsubscript{lang} auf, aber diese ist auch nicht unerklärbar groß. Im Gegenteil, die Abweichung ist sogar sehr gering. Die Abweichung liegt hier an Fehlern durch die Ausgleichrechnung und der Unsicherheit von T\textsubscript{lang}. Denn T\textsubscript{kurz} wird mit T\textsubscript{lang} bestimmt und setzt somit die Unsicherheit weiter fort.   

\noindent Zusammenfassend lässt sich über dieses Experiment sagen, dass es überaus erfolgreich war. Alle Abläufe die zu sehen sein sollten konnten dargestellt werden und die enstandenen Werte sind durch ihre geringe Abweichung auch von überaus hoher Qualität. Es gibt nur Abweichungen durch praktisch unvermeidbare Fehler. Diese fallen aber sehr klein aus und sind deshalb im Bereich der zu erwartenden Abweichung.

\section{Literatur}

Anleitung V702:\\
\url{https://moodle.tu-dortmund.de/pluginfile.php/1502371/mod_folder/content/0/V702.pdf?forcedownload=1}\\
DantenHinweiseGeigerMueller:\\
\url{https://moodle.tu-dortmund.de/pluginfile.php/1502371/mod_folder/content/0/DatenHinweiseAktivierung.pdf?forcedownload=1}\\
Literaturwert der Halbwertszeit von Vanadium:\\
\url{https://www.internetchemie.info/chemische-elemente/vanadium-isotope.php}\\
Literaturwerte der Halbwertszeiten von Rhodium:\\
\url{https://www.internetchemie.info/chemische-elemente/rhodium-isotope.php}\\


\section{Tabellen}
\begin{minipage}{\linewidth}
    \begin{table}[H]
        \centering
    
    \begin{tabular}{c}
        \toprule
        $N_U$ [Imp/300s]\\
        \midrule
        129 \\
        143 \\
        144 \\
        136 \\
        139 \\
        126 \\
        158 \\
        \bottomrule
        
    \end{tabular}
    \captionof{table}{Messung zur Bestimmung des Untergrundes}
    \label{tab:1}
\end{table}
\end{minipage}


\begin{minipage}{\linewidth}
    \begin{table}[H]
        \centering
    
    \begin{tabular}{ll}
        \toprule
        t [s] & N [Imp] \\
        \midrule
        30	& 189 \\
        60	& 197 \\
        90	& 150 \\
        120	& 159 \\
        150	& 155 \\
        180	& 132 \\
        210	& 117 \\
        240	& 107 \\
        270	& 94 \\
        300	& 100 \\
        330	& 79 \\
        360	& 69 \\
        390	& 81 \\
        420	& 46 \\
        450	& 49 \\
        480	& 61 \\
        510	& 56 \\
        540	& 40 \\
        570	& 45 \\
        600	& 32 \\
        630	& 27 \\
        660	& 43 \\
        690	& 35 \\
        720	& 19 \\
        750	& 28 \\
        780	& 27 \\
        810	& 36 \\
        840	& 25 \\
        870	& 29 \\
        900	& 18 \\
        930	& 17 \\
        960	& 24 \\
        990	& 21 \\
        1020 &	25 \\
        1050 &	21 \\
        1080 &	24 \\
        1110 &	25 \\
        1140 &	17 \\
        1170 &	20 \\
        1200 &	19 \\
        1230 &	20 \\
        1260 &	18 \\
        1290 &	16 \\
        1320 &	17 \\
        \bottomrule
        
    \end{tabular}
    \captionof{table}{Messwerte zur Bestimmung der Halbwertszeit von Vanadium}
    \label{tab:2}
\end{table}
\end{minipage}


\begin{minipage}{\linewidth}
    \begin{table}[H]
        \centering
    
    \begin{tabular}{ll}
        \toprule
        t [s] & N [Imp] \\
        \midrule
        15 &	 667 \\
        30 &	 585 \\ 
        45 &	 474 \\
        60 &	 399 \\
        75 &	 304 \\
        90 &	 253 \\
        105	&    213 \\
        120	&    173 \\
        135	&    152 \\
        150	&    126 \\
        165	&    111 \\
        180	&     92 \\
        195	&     79 \\
        210	&     74 \\
        225	&     60 \\
        240	&     52 \\
        255	&     56 \\
        270	&     53 \\
        285	&     41 \\
        300	&     36 \\
        315	&     37 \\
        330	&     32 \\
        345	&     36 \\
        360	&     38 \\
        375	&     34 \\
        390	&     40 \\
        405	&     21 \\
        420	&     35 \\
        435	&     33 \\
        450	&     36 \\
        465	&     20 \\
        480	&     24 \\
        495	&     30 \\
        510	&     30 \\
        525	&     26 \\
        540	&     28 \\
        555	&     23 \\
        570	&     20 \\
        585	&     28 \\
        600	&     17 \\
        615	&     26 \\
        630	&     19 \\
        645	&     13 \\
        660	&     17 \\
        \bottomrule
        
    \end{tabular}
    \captionof{table}{Messwerte zur Bestimmung der Halbwertszeit von Rhodium}
    \label{tab:3}
\end{table}
\end{minipage}
\end{document}
%\end{document}
