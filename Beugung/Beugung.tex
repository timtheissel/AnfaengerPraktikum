\section{Zielsetzung}
In diesem Versuch sollen grundsaetzliche Gesetze der Optik dargestellt und bearbeitet werden. Dazu wird die Reflexion, die Beugung und die Transmission untersucht.
\section{Theoretische Grundlagen}
Licht ist elektromagnetische Strahlung, wobei das optische Spektrum mit dem ultravioletten Spektrum mit einer Wellenlaenge von etwa 100nm bis 380nm beginnt und sich bis ins infrarote Spektrum erstreckt mit etwa 780nm bis 1mm Wellenlaenge. Vom Menschen wahrgenommen kann dabei jedoch nur das Spektrum zwischen 380nm und 780nm. Die Maxwellgleichungen beschreiben dabei die Ausbreitung der elektromagnetischen Welle, koennen jedoch durch die einfacheren Gesetze der Strahlenoptik ersetzt werden fuer Phaenomene wie Brechung an Grenzflaechen oder Reflexion.\\
Die Ausbreitungsrichtung der Welle wird in der Strahlenoptik durch die Wellennormale charakterisiert, welche senkrecht auf der Wellenfront steht. Diese wird vereinfachend Lichtstrahl genannt. Die Ausbreitung dieser Welle erfolgt in verschiedenen Materialien mit unterschiedlichen Geschwindigkeiten. Aus diesem Grund wird der Lichtstrahl beim Uebergang von einem Material in ein anderes mit anderem Brechungsindex gebrochen. Der Brechungsindex ist dabei ein Mass fuer die Aenderung der Lichtgeschwindigkeit im jeweiligen Medium verglichen mit der Lichtgeschwindigkeit im Vakuum. Wenn die Ausbreitungsgeschwindigkeit im anderen Medium hoeher ist, spricht man von einem optisch dichteren Material, wohingegen ein Material mit geringerer Ausbreitungsgeschwindigkeit optisch duenner genannt wird. Ausserdem ist wichtig, dass sich Lichtstrahlen in der Strahlenoptik nicht gegenseitig beeinflussen koennen, wenn sie sich kreuzen. Der Strahl muss ausserdem umkehrbar sein und sich geradlinig verbreiten, sollte das Medium homogen sein.
\newpage \subsection{Reflexion}
Ein Lichtstrahl wird an einer Grenzflaeche reflektiert. Das Reflexionsgesetz gibt dann, dass der Einflasswinkel $\alpha_1$ dem Reflexionswinkel/Ausfallswinkel $\alpha_2$ entspricht.
\begin{equation}
    \alpha_1=\alpha_2 \label{Reflexion}
\end{equation}
\noindent Dies ist nocheinmal graphisch in Abbildung 1 dargestellt. 
\begin{figure}[H]
    \centering
    \captionsetup{justification=centering}
    \includegraphics[height=5cm]{"Reflexion_Beugung.png"}
    \captionbelow{Reflexion an einer Grenzflaeche\\ Aus: Anleitung V400 Seite 2}
    \label{Fig:Reflexion}
\end{figure}
\subsection{Brechung}
Breitet sich ein Lichtstrahl in einem Medium mit anderem Brechungsindex aus, nachdem es durch eine Grenzflaeche eingetreten ist, ist wie bereits genannt die Ausbreitungsgeschwindigkeit verschieden. Nach dem Fermatschen Prinzip erfaehrt der Lichtstrahl dann eine Richtungsaenderung an der Grenzflaeche. Der Lichtstrahl wird gebrochen. Das Snellius'sche Brechungsgesetz liefert dann:
\begin{equation}
    n_1sin\alpha=n_2sin\beta \label{Brechung}
\end{equation}
\begin{figure}[H]
    \centering
    \captionsetup{justification=centering}
    \includegraphics[height=5cm]{"Brechung_Beugung.png"}
    \captionbelow{Beugung an einer Grenzflaeche\\ Aus: Anleitung V400 Seite 2}
    \label{Fig:Brechung}
\end{figure}
\subsection{Reflexion und Transmission}
Ein Lichtstrahl wird fuer gewoehnlich an einer Grenzflaeche nicht vollstaendig reflektiert oder gebrochen. Es wird nur ein Teil reflektiert und ein Teil gebrochen. Die Anteile an der Gesamtintensitaet sind also der Reflexionsanteil R und der Transmissionsanteil T. Es muss ausserdem immer $R+T=1$ gelten. 
\begin{figure}[H]
    \centering
    \captionsetup{justification=centering}
    \includegraphics[height=5cm]{"Transmission_Beugung.png"}
    \captionbelow{Reflexion und Transmission an einer Grenzflaeche\\ Aus: Anleitung V400 Seite 2}
    \label{Fig:Transmission}
\end{figure}
\subsection{Beugung}
Bei Licht kann ein weiteres interessantes Phaenomen beobachtet werden, die Ausbreitung im Schattenraum nach Kollision mit einem Hindernis. Hierzu benoetigt die Erklaerung den Bereich der Wellenoptik, da die Strahlenoptik nicht in der Lage ist dieses Verhalten zu erklaeren. Zur Beschreibung einer elektromagnetischen Welle sind vor allem die Frequenz $\mu$ beziehungsweise die Wellenlaenge $\lambda$ und die Ausbreitungsgeschwindigkeit $v$ interessant. Kommt zu einer Ueberlagerung von zwei oder mehr Wellen, addieren sich die Amplituden nach dem Superpositionsprinzip in jedem Punkt auf. Dabei kann es zu konstruktiver Interferenz, also einer Erhoehung der Amplitude, oder destruktiver Interferenz, einer Verringerung der Amplitude kommen. Diese destruktive Interferenz ist bei gleicher Frequenz, gleicher Intensitaet und einem Gangunterschied von einer halben Wellenlaenge, also $\lambda/2$ am deutlichsten. Die Welle wird komplett ausgeloescht.\\
Zum Verstaendnis des Verhaltens des Lichts im Beugungsversuch am Gitter ist das huygensche Prinzip zu verstehen. Dieses besagt, dass jeder Punkt einer Welle, Quelle einer neuen Elementarwelle der selben Frequenz ist. Die Einhuellende dieser Elementarwellen stellt dann die eigentliche Wellenfront dar.\\
Wird nun das Licht durch einen Spalt geschickt, erfaehrt jeder Punkt im Spalt Beugung. Alle diese neu gebeugten Wellen haben die selbe Frequenz und die gleiche Phasenbeziehung. Stellt man nun dahinter im Abstand L vom Spalt einen Schirm auf, laesst sich ein Muster aus Streifen mit grosser und geringer Helligkeit erkennen. Diese so genannten Interferenzstreifen entstehen- wie der Name schon vermuten laesst- durch konstruktive Interferenz in den hellen Bereichen und destruktive Interferenz in den dunklen Teilen. Die hellen Interferenzstreifen erscheinen dabei an Orten, welche diese Beziehung erfuellt:
\begin{equation}
    a \sin\alpha=k\lambda
\end{equation}
Dabei stellt a die Breite des Spalts dar, welcher mit Licht der Wellenlaenge $\lambda$ beleuchtet wird. Die k-ten Maxima an Intensitaet entstehen dann in einem Winkel $\alpha$ zur der Ausbreitungsrichtung.\\
Dieser Umstand laesst sich nun auf n-viele Spalte verallgemeinern. Das Mass fuer die Anzahl an Spalten nennt sich Gitterkonstante d und ist der Kehrwert der Anzahl an Spalten pro Millimeter des Gitters. Wie bereits beim einzelnen Spalt auch laesst sich das k-te Intensitaetsmaximum mit dieser Beziehung herleiten:
\begin{equation}
    d\sin\alpha=k\lambda
\end{equation}
\section{Versuchsaufbau}
Der Versuch besteht aus einer durchsichtigen Bodenplatte, auf welcher sich zwei Laserdioden im Halbkreis bewegen lassen. Eine der Dioden ist gruen mit einer Wellenlaenge von $\lambda_G=532$, wohingegen die andere Diode rot ist mit $\lambda_R=635nm$. Im Zentrum des Halbkreises lassen sich verschiedene Elemente wie Spiegel, Prismen oder planparallele Platten befestigen. Die umgebende Luft ist in diesem Experiment immer das optisch duennere Medium mit Lichtgeschwindigkeit c und einem Brechungsindex von $n\approx 1$.
Zum Schutz vor dem Laserlicht laesst sich in Reflexionsrichtung ein Schirm aufstecken, welcher bei Verwendung beider Dioden erhoeht werden soll. Zur Bestimmung der Winkel muss eine Vorlage unter den Versuchsaufbau gelegt werden. Die moeglichst genaue Ausrichtung ist hierbei sehr wichtig. Alle zu untersuchenden Elemente sind der Einfachheit halber mit Vertiefungen zum Aufstecken auf kleine Alustifte in der Platte ausgestattet, welche benutzt werden muessen. Ausserdem ist es wichtig nicht direkt auf die Einstrahlflaechen der Elemente zu fassen, da diese dadurch zerstoert werden koennten.
\begin{figure}[H]
    \centering
    \captionsetup{justification=centering}
    \includegraphics[height=5cm]{"Aufbau_Beugung.png"}
    \captionbelow{Die Versuchsapparatur\\ Aus: Anleitung V400 Seite 4}
    \label{Fig:Aufbau}
\end{figure}
\section{Durchfuehrung}
\subsection{Reflexion}
Zur Beobachtung des Reflexionsgesetzes ist die gruene Laserdiode der Wellenlaenge $\lambda_G=532$ zu verwenden. Ausserdem wird der Spiegel als reflektierendes Element benutzt, welcher in der Mitte des Halbkreises auf die Apparatur aufgesteckt wird. Als Vorlage wird Vorlage A benoetigt, welche, wie bereits genannt, unter die Grundplatte gelegt wird. Auf ihr ist eine Winkelskala abgebildet auf, welcher der Einfallswinkel sowie der Ausfallswinkel abgelesen werden kann. Es sollen sieben Einfalls- und Ausfallswinkel gemessen werden.
\subsection{Brechung}
Fuer diesen Teil des Versuchs wird ebenfalls der gruene Laser, sowie die Vorlage A verwendet. Als optisches Element dient hierbei jedoch ein planparalleler Quader. Dieser muss ebenfalls auf die dafuer vorgesehenen Stifte gesteckt werden, um Reproduzierbarkeit und Ablesbarkeit zu garantieren. Der Einfallswinkel kann erneut an der Vorlage abgelesen werden; der Brechungswinkel wird aber nun an einer Skala abgelesen, welche an der Rueckseite des Quaders aufgebracht ist, abgelesen. Erneut sollen sieben Einfallswinkel und ihr zugehoeriger Brechungswinkel gemessen werden. Ausserdem soll der Strahlversatz s mit
\begin{align}
    s&=d\frac{sin(\alpha-\beta)}{cos\beta}\\
    hier: d&=5.85cm \text{(Tiefe der Platte)}
\end{align}
bestimmt werden.
\subsection{Brechung am Prisma}
Das Prisma wird ausgezeichnet durch nicht parallele Grenzflaechen. Die Einfallsflaeche und die Ausfallsflaeche begrenzen den brechenden Winkel $\gamma$. Der Einfallswinkel $\alpha$ ist bei weissem Licht fuer alle Wellenlaengen gleich, jedoch ist der Brechungswinkel $\beta$ bei nicht-monochromatischem Licht je nach Wellenlaenge verschieden. Dies nennt sich Dispersion. Beim Durchgang durch das Prisma erfaehrt das Licht insgesamt eine Ablenkung von
\begin{equation}
    \delta=(\alpha_1+\alpha_2)-(\beta_1+\beta_2)
\end{equation}
Die Winkel $\beta_1$ und $\beta_2$ koennen mit Hilfe des Brechungsgesetzes und der Winkelbeziehung $\beta_1+\beta_2=\gamma$ bestimmt werden. Es soll sowohl der rote, als auch der gruene Laser benutzt werden. Der brechende Winkel des verwendeten Kronglasprismas betraegt $\gamma=60^{\circ}$. Es sollen fuenf Einfallswinkel $\alpha_1$ mit beiden Lasern gemessen werden. Als Vorlage dient dazu Vorlage C.
\begin{figure}[H]
    \centering
    \captionsetup{justification=centering}
    \includegraphics[height=5cm]{"Prisma_Beugung.png"}
    \captionbelow{Das Prisma\\ Aus: Anleitung V400 Seite 7}
    \label{Fig:Prisma}
\end{figure}
\subsection{Beugung am Gitter}
Zur Beugung am Gitter wird der Laserdiodentraeger auf $0^{\circ}$ eingestellt und der Schirmhalter neben der Bodenplatte auf die Vorlage in Strahlrichtung gestellt. Ausserdem muss ein Transmissionsschirm im auf der Vorlage markierten Abstand postioniert werden. Dieser ist ausserdem im Kreis zu positionieren, um die Winkel $\varphi$ direkt ablesen zu koennen. Es sollen nun die Maxima fuer beide Laser und alle Gitter gemessen werden. So kann mit der Gitterkonstante d und der Beugungsordnung k die Wellenlaenge $\lambda$ des Lasers berechnet werden:
\begin{equation}
    \lambda=d\frac{sin\varphi}{k}
\end{equation}