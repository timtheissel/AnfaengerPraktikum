\section{Auswertung}

\subsection{Reflexionsgesetz}

Zuerst soll der Lichtstrahl eines Lasers an einem Spiegel reflektiert werden. Damit wird das Reflexionsgesetz überprüft, welches besagt, dass Einfallswinkel \alpha gleich Ausfallswinkel \beta ist. Bei der Messung entstehen folgende Wertepaare:

\begin{minipage}{\linewidth}
    \begin{table}[H]
        \centering
    \captionof{table}{Messung der Einfalls- und Ausfallswinkel für das Reflexionsgesetz}
    \begin{tabular}{ll}
        \toprule
        Einfallswinkel [$^\circ$] & Ausfallswinkel [$^\circ$] \\
        \midrule
        20 & 20 \\
        30 & 30 \\
        40 & 40 \\
        45 & 45 \\
        50 & 50 \\
        60 & 60 \\
        70 & 70 \\
        \bottomrule   
    \end{tabular}
    
    \label{tab:1}
\end{table}
\end{minipage}

\subsection{Brechungsgesetz}

Bei der zweiten Messreihe wird ein Plexiglas Quader verwendet. Der durch ihn durch gerichtete Laser wird gebrochen. Nach der Messung des Brechungswinkels und dem Ablesen des Einfallswinkels kann mithilfe des Brechungsgesetzes der Brechungsindex von Plexiglas bestimmt werden.
Dafür wird lediglich die Formel:
\begin{displaymath}
    \frac{\sin\alpha}{\sin\beta} = \text{n}
\end{displaymath}

\noindent benötigt. Mit den gemessenen Werten ergeben sich folgende Werte für den Brechungsindex:

\begin{minipage}{\linewidth}
    \begin{table}[H]
        \centering
    \captionof{table}{Messwerte zur Berechnung des Brechungsindex von Plexisglas und dazu berechneter Brechungsindex}
    \begin{tabular}{lll}
        \toprule
        Einfallswinkel [$^\circ$] & Brechungswinkel [$^\circ$] & Brechungsindex \\
        \midrule
        10 &  6.5 & 1.53 \\
        20 & 13.5 & 1.47 \\
        30 & 19.5 & 1.50 \\
        40 & 25.5 & 1.49 \\ 
        50 & 31.0 & 1.49 \\
        60 & 35.5 & 1.49 \\
        70 & 39.0 & 1.49 \\
        \midrule
        Mittelwert & & (1.494$\pm$0.019)\\
        \bottomrule   
    \end{tabular}
    
    \label{tab:2}
\end{table}
\end{minipage}

\noindent Die Lichtgeschwindigkeit in Plexiglas ergibt sich indem, das folgende Verhältnis gebildet wird:

\begin{displaymath}
    n = \frac{c_0}{c_\text{M}}
\end{displaymath}

\noindent Dabei ist c$_0$ die Lichtgeschwindigkeit im Vakuum, c$_M$ die Lichtgeschwindigkeit im Medium, in diesem Fall Plexiglas und n ist der berechnete Brechungsindex von Plexiglas. Diese Formel lässt sich wie folgt nach der Lichtgeschwindigkeit von Plexiglas umstellen:

\begin{displaymath}
    c_\text{M} = \frac{c_0}{n} 
\end{displaymath}

\noindent Damit ergibt sich die Lichtgeschwindigkeit in Plexiglas zu (2.007$\pm$0.026)$*10^8$ $\frac{m}{s}$.

\subsection{Planparallele Platten}

Bei dieser Aufgabe wird der Strahlenversatz bestimmt. Dafür wurden erneut die Werte für Einfalls- und Brechungswinkel aus Aufgabe 2 verwendet. Mit der Formel:

\begin{displaymath}
    s = d\frac{\sin(\alpha-\beta)}{\cos(\beta)}
\end{displaymath}

\noindent lässt sich der Strahlenversatz berechnen. Dabei entstehen folgende Werte:

\begin{minipage}{\linewidth}
    \begin{table}[H]
        \centering
    \captionof{table}{Gemessene Einfalls- und Brechungswinkel und dazu berechneter Strahlenversatz}
    \begin{tabular}{lll}
        \toprule
        Einfallswinkel [$^\circ$] & Brechungswinkel [$^\circ$] & Strahlenversatz [cm]  \\
        \midrule
        10 &  6.5 & 0.36 \\
        20 & 13.5 & 0.68 \\
        30 & 19.5 & 1.13 \\
        40 & 25.5 & 1.62 \\ 
        50 & 31.0 & 2.22 \\
        60 & 35.5 & 2.98 \\
        70 & 39.0 & 3.88 \\
        \bottomrule   
    \end{tabular}
    
    \label{tab:3}
\end{table}
\end{minipage}

\noindent Mit dem Wissen aus der vorherigen Aufgabe ist es nun auch möglich den Brechungswinkel mithilfe des Brechungsgesetzes zu bestimmen und dann mit den berechneten Winkeln den Strahlenversatz zu bestimmen.

\begin{minipage}{\linewidth}
    \begin{table}[H]
        \centering
    \captionof{table}{Strahlenversatz mit berechneten Brechungswinkeln}
    \begin{tabular}{lll}
        \toprule
        Einfallswinkel [$^\circ$] & Brechungswinkel [$^\circ$] & Strahlenversatz [cm]  \\
        \midrule
        10 &  4.35 & 0.57 $\pm$0.006 \\
        20 & 8.99 & 1.13 $\pm$0.011 \\
        30 & 12.91 & 1.76 $\pm$0.016 \\
        40 & 16.75 & 2.41 $\pm$0.019 \\ 
        50 & 20.17 & 3.10 $\pm$0.020 \\
        60 & 22.87 & 3.83 $\pm$0.018 \\
        70 & 24.91 & 4.57 $\pm$0.014 \\
        \bottomrule   
    \end{tabular}
    
    \label{tab:4}
\end{table}
\end{minipage}

\subsection{Prisma}

Ziel dieser Aufgabe ist es die Ablenkung zu bestimmen. Dafür wurden die Winkel $\alpha_1$ und $\alpha_2$ gemessen. Daraus lassen sich nun mit dem Brechungsgesetz $\beta_1$ und $\beta_2$ bestimmen.
Bei dieser Rechnung ergeben sich folgende Winkel:

\begin{minipage}{\linewidth}
    \begin{table}[H]
        \centering
    \captionof{table}{Brechungswinkel Prisma}
    \begin{tabular}{lll}
        \toprule
        Einfallswinkel [$^\circ$] & Brechungswinkel 1 [$^\circ$] & Brechungswinkel 2 [$^\circ$]  \\
        \midrule
        35 & 22.58 & 37.42 \\
        40 & 25.48 & 34.52 \\
        45 & 28.25 & 31.75 \\
        50 & 30.85 & 29.15 \\
        60 & 35.43 & 24.57 \\
        \bottomrule   
    \end{tabular}
    
    \label{tab:5}
\end{table}
\end{minipage}

\noindent Mit diesen Winkeln lässt sich nun die Dispersion bestimmen. Dafür wird die angegebene Formel verwendet. Für die Dispersion ergeben sich folgende Werte:

\begin{minipage}{\linewidth}
    \begin{table}[H]
        \centering
    \captionof{table}{Dispersion}
    \begin{tabular}{lll}
        \toprule
        Einfallswinkel [$^\circ$] & roter Laser [$^\circ$] & grüner Laser [$^\circ$]  \\
        \midrule
        35 & 42 & 43.0 \\
        40 & 39 & 40.0 \\
        45 & 38 & 38.5 \\
        50 & 37 & 38.0 \\
        60 & 38 & 39.0 \\
        \bottomrule   
    \end{tabular}
    
    \label{tab:6}
\end{table}
\end{minipage}

\subsection{Beugung am Gitter}

Ziel dieser Aufgabe ist es die Wellenlänge des verwendeten Lasers zu bestimmen. Dafür kann die Beugung am Gitter verwendet werden. Durch diese entstehen Interferenzmuster auf dem Schirm. Für die auftretenden Beugungsmaxima gilt dann folgende Beziehung:
\begin{displaymath}
    \lambda = d\frac{\sin\phi}{k}
\end{displaymath}

\noindent Nach der Messung der Beugungswinkel ergeben sich für die Wellenlänge folgende Werte:

\begin{minipage}{\linewidth}
    \begin{table}[H]
        \centering
    \captionof{table}{Wellenlänge d = 600Lines/mm }
    \begin{tabular}{llll}
        \toprule
        Beugunswinkel [$^\circ$] & roter Laser [nm] & Beugunswinkel [$^\circ$] & grüner Laser [nm]  \\
        \midrule
        -23 & 651.21 & -19.5 & 556.34 \\
        0   & 0    & 0     & 0    \\
        23  & 651.21  & 19.5  & 556.34  \\
        \bottomrule   
    \end{tabular}
    
    \label{tab:7}
\end{table}
\end{minipage}

\begin{minipage}{\linewidth}
    \begin{table}[H]
        \centering
    \captionof{table}{Wellenlänge d = 300Lines/mm }
    \begin{tabular}{llll}
        \toprule
        Beugunswinkel [$^\circ$] & roter Laser [nm] & Beugunswinkel [$^\circ$] & grüner Laser [nm]  \\
        \midrule
        -22.5 & 638.00 & -19 & 542.61 \\
        -11 & 636.03 & -9 & 521.44 \\
        0   & 0    & 0     & 0    \\
        11  & 636.03 & -9 & 521.44 \\
        22.5  & 638.00  & 19  & 542.61  \\
        \bottomrule   
    \end{tabular}
    
    \label{tab:8}
\end{table}
\end{minipage}

\begin{minipage}{\linewidth}
    \begin{table}[H]
        \centering
    \captionof{table}{Wellenlänge d = 100Lines/mm }
    \begin{tabular}{llll}
        \toprule
        Beugunswinkel [$^\circ$] & roter Laser [nm] & Beugunswinkel [$^\circ$] & grüner Laser [nm]  \\
        \midrule
        0   & 0    & 0     & 0    \\
        4 & 697.56 & 3 & 523.36 \\
        7.5 & 652.63 & 6 & 523.40\\
        11  & 639.52 & 9.5 & 552.40 \\
        15 & 654.03 & 12.5 & 545.15 \\
        19 & 651.14 & 16 & 551.27\\
        23  & 651.21  & 19.5  & 556.34  \\
        27 & 648.58 & 23 & 558.19 \\
        32 & 662.40 & 26 & 548.00 \\
           &        & 29 & 538.70\\
           &        & 33 & 544.64 \\
        \midrule
        Mittelwert & 657.13 & & 544.14\\
        \bottomrule   
    \end{tabular}
    
    \label{tab:9}
\end{table}
\end{minipage}

\section{Diskussion}

\subsection{Reflexionsgesetz}

Die gemessen Werte bestätigen das Reflexionsgesetz. Die Einfallswinkel sind alle gleich den Ausfallswinkeln. Die Werte sind allerdings nur auf 1$^\circ$ genau ablesbar. Das liegt an der Perspektive und auch der dicke des Laserpunktes. Diese Ableseungenauigkeiten setzen sich auch im weiteren Verlauf des Experimentes fort.

\subsection{Brechungsgesetz}

Mit dem Brechungsgesetz wurde der Brechungsindex für Plexiglas bestimmt. Dieser liegt mit 1.494 nur sehr knapp über dem Literaturwert von 1.49. Die Abweichung beträgt lediglich 0.3\%. Der Fehler ist damit äußerst gering.

\subsection{Planparallele Platten}

Hier ist eine deutliche Abweichung zu erkennen alleine der gemessene Brechungswinkel hat mit 39$^\circ$ weicht von dem berechneten mit 24.91$^\circ$ um 56.56\% ab. Das liegt vermutlich an dem Aufbau des Schirms. Da man diesen ausrichten musste sind dabei wahrscheinlich die größten Fehler entstanden. Dazu kommen dann noch die üblichen Ablesefehler. Damit ist es auch nicht verwunderlich das bei einem Einfallswinkel von 70$^\circ$ der Strahlenversatz um 15.1\% abweicht.

\subsection{Beugung am Gitter}

Bei diesem Teil des Experiments wurden die  Wellenlängen der Laser bestimmt. Dabei beträgt die Wellenlänge von rotem Licht 650nm. Der berechnete wert von 657.13 liegt nur um 1.1\% daneben. Ähnliches ist zu erkennen bei dem grünen Licht. Das Licht sollte eine Wellenlänge von 550nm haben. der berechnete Wert von 544.14nm weicht  nur um 1.06\% ab. I Rahmen der gewöhnlichen Messfehler ist sind diese werte sehr genau bestimmt.
