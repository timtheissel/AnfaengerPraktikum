\section{Auswertung}

\subsection{Bestimmung eines unbekannten Widerstandes mithilfe einer Wheatstoneschen Brückenschaltung}

Es wurde eine Wheatstoneschen Brückenschaltung aufgebaut. Dann wurde eine Spannung angelegt und die Brückenspannung wurde mit dem Oszilloskop visualisiert. Mit dem Potentiometer wurden R$_3$ und R$_4$ so eingestellt, dass die Brückenspannung verschwindet. Der unbekannte Widerstand R$_x$ kann mit der folgenden Formel bestimmt werden.

\begin{displaymath}
    R_x = R_2 \frac{R_3}{R_4}
\end{displaymath}

Der unbekannte Widerstnad wurde für 2 verschiedene R$_2$ bestimmt. Dabei wurden folgende Werte für R$_3$ und R$_4$ am Potentiometer eingestellt.

\begin{minipage}{\linewidth}
    \begin{table}[H]
        \centering
    \captionof{table}{}
    \begin{tabular}{llll}
        \toprule
        R$_2$ [$\Omega$]  & R$_3$ [$\Omega$] &  R$_4$ [$\Omega$]   \\
        \midrule
        664 & 263 & 737 \\
        1000 & 192.5 & 807.5 \\
        \bottomrule   
    \end{tabular}
\end{table}
\end{minipage}

Damit berechnet sich R$_x$ einmal zu (236.9$\pm$1.3)$\Omega$ und einmal zu (238.4$\pm$1.3)$\Omega$.

\subsection{Kapazitätsmessbrücke}

Im nächsten Teil wurde eine Kapazitätsmessbrücke aufgebaut, um die Kapazität eines Kondensators zu messen. Dabei werden diesmal mit 2 Potentiometern die Widerstände R$_2$, R$_3$ und R$_4$ eingestellt, sodass die Brückenspannung verschwindet. Anschließend können dann ein unbekannter Widerstand und eine unbekannte Kapazität bestimmt werden. Dies geschieht mit den folgenden Formeln:

\begin{align*}
    R_x = R_2 \frac{R_3}{R_4}\\
    C_x = C_2 \frac{R_4}{R_3}
\end{align*}

Dabei ist C$_2$= 750nF eine bekannte Kapazität. Mit den Potentiometern wurden R$_2$= 269$\Omega$, R$_3$= 630$\Omega$  und R$_4$= 370$\Omega$ bestimmt. Mit den angegebenen Formeln ergibt sich R$_x$= (458$\pm$14)$\Omega$ und C$_x$= (440.5$\pm$0.9)$\Omega$.

\subsection{Induktivitätsmessbrücke}

Die Messung bei der Induktivitätsmessbrücke verläuft analog zu den Vorherigen. Im Vergleich zur Kapazitätsmessung wurden hier lediglich die Kondensatoren durch Spulen ersetzt und die Frequenz des Wechselstroms wurde um 1000Hz erhöht.

Bei der Messung war L$_2$= 20.1mH, R$_2$ ergab sich zu 228$\Omega$, R$_3$ ergab sich zu 592$\Omega$ und daraus wurde R$_4$= 408$\Omega$ bestimmt.
Mit den Formeln:
\begin{align*}
    R_x = R_2 \frac{R_3}{R_4}\\
    L_x = L_2 \frac{R_3}{R_4}
\end{align*}

lassen sich nun die gesuchte Induktivität und der Widerstand bestimmen. Es entstehen folgende Werte:

\begin{align*}
    R_x = (331\pm 10)\Omega \\
    L_x = (29.16\pm 0.06)\text{mH}
\end{align*}

\subsection{Induktivitätsbestimmung mit der Maxwell-Brücke}

Bei der Messung mit der Maxwell-Brücke wurde die selbe Spule nocheinmal ausgemessen. Es entstehen folgende Werte:

\begin{align*}
    R_2 = 1000\Omega \\
    R_3 = 73\Omega \\
    R_4 = 192\Omega \\
    C_4 = 750 nF
\end{align*}

Mit diesen Werten können die unbekannte Induktivität I$_x$ und der Widerstand R$_x$ bestimmt werden.

\begin{align*}
    R_x = R_2 \frac{R_3}{R_4}\\
    L_x = R_2 * R_3 * C_4
\end{align*}

Die Rechnung liefert folgende Werte:

\begin{align*}
    R_x = (380\pm 16)\Omega \\
    L_x = (54.7\pm 0.16)\text{mH}
\end{align*}

\subsection{Frequenzabhängigkeit der Spannung bei einer Wien-Robinson-Brücke}

Um die Frequenzabhängigkeit einer Wien-Robinson-Brücke zu bestimmen wurden bei verschiedenen Frequenzen die Spannungsamplituden gemessen. Bei der Messung wurde eine Wien-Robinson-Brücke mit R= 1000$\Omega$, R'= 332$\Omega$, C= 295 nF und U$_S$= 10V verwendet. Dabei wurden folgende Messwerte aufgenommen.

\begin{minipage}{\linewidth}
    \begin{table}[H]
        \centering
    \captionof{table}{Frequenzabhängigkeit der Spannung bei einer Wien-Robinson-Brücke}
    \begin{tabular}{ll}
        \toprule
        v [Hz]  & A [V] \\
        \midrule
        20     & 3    \\
        50     & 3.2  \\
        100    & 2.8  \\
        150    & 2.4  \\
        200    & 2.0  \\
        250    & 1.6  \\
        300    & 1.2  \\
        350    & 0.9  \\
        400    & 0.68 \\
        450    & 0.44 \\
        500    & 0.17 \\
        520    & 0.09 \\
        540    & 0.01 \\
        550    & 0.05 \\
        600    & 0.24 \\
        700    & 0.6  \\
        800    & 0.9  \\
        900    & 1.1  \\
        1000   & 1.3  \\
        3000   & 2.7  \\
        5000   & 3    \\
        7000   & 3    \\
        9000   & 3    \\
        10000  & 3    \\
        12000  & 2.8  \\
        14000  & 2.8  \\
        16000  & 2.7  \\
        18000  & 2.5  \\
        19000  & 2.4  \\
        20000  & 2.4  \\
        \bottomrule   
    \end{tabular}
\end{table}
\end{minipage}

Eine halblogarithmische Darstellung dieser Werte sieht folgendermaßen aus:

\begin{figure}[H]
    \centering
    \includegraphics[height=8cm]{"e.png"}
\end{figure}

Die Frequenz bei der die Brückenspannung verschwindet ($v_0$), liegt laut den Messwerten bei ungefähr 540Hz. Dieser Wert lässt sich allerdings auch berechnen. Das geschieht mit folgender Formel:

\begin{displaymath}
    \omega_0 = \frac{1}{RC}
    v_0 = \frac{\omega_0}{2\pi}
\end{displaymath}

Dabei ergeben sich folgende Werte:

\begin{align*}
    \omega_0= 3.389.38Hz
    v_0= 539.51Hz
\end{align*}

\subsection{Klirrfaktor-Messung}

Für den Klirrfaktor wird angenommen, dass die Summe der Oberwellen lediglich der Wert für die zweite Obersumme ist. Dies hat zur Folge, dass lediglich U$_1$ und U$_2$ bestimmt werden müssen. U$_1$ ist dabei der Wert von U$_S$ bei v$_0$. Dieser beträgt 10V. Für U$_2$ wird folgende Formel verwendet:

\begin{displaymath}
    U_2 = \frac{0.01V}{\sqrt{\frac{(2^2-1)^2}{9*((1-2^2)^2)+9*2^2}}}
\end{displaymath}

Anschließend wird folgendermaßen der Klirrfaktor bestimmt:

\begin{displaymath}
    k=\frac{U_2}{U_1}
\end{displaymath}

Der Klirrfaktor beträgt: $k= 6.71*10^{-3}$.

\section{Diskussion}

Die Wheatstonesche Brückenschaltung hat bei der Widerstandsbestimmung sehr genaue Werte geliefert. Die berechneten Werte liegen sehr nah an dem tatsächlichen Wert von 239$\Omega$. Genau wie die Kapazitätsbrücke. Diese liefert auch Werte, die nah an den Literaturwerten von 464,9$\Omega$ und 433,71nF liegen. Bei den weiteren Messungen fällt auf, dass die Induktivitätsmessbrücke deutlich schlechtere Ergebnisse liefert, als die Maxwell-Brücke. Die berechnete Induktivität bei der Maxwell-Brücke ist nur um 5mH größer als der Literaturwert von 49,82mH. Bei der Induktivitätsmessbrücke ist der Wert um 20mH kleiner als der Literaturwert. Bei der Berechnung des Klirrfaktors fällt auf, dass dieser sehr klein ist. Die Messung der Frequenzabhängigkeit hat gezeigt, dass die Messwerte sehr nah an den theoretischen Werten liegen. Dies ist auch anschaulich in der Grafik zu erkennen.