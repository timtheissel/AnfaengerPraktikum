\documentclass[titlepage=firstcover, captions=tableheading]{scrartcl}
\usepackage{microtype}
\usepackage{amsmath}
\usepackage{polyglossia}
\usepackage{graphicx}
\usepackage{booktabs}
\usepackage{siunitx}
\usepackage{hyperref}
\usepackage{caption}
\usepackage{float}
\usepackage{parskip}
\setdefaultlanguage{german}
\title{106 gekoppeltes Pendel}
\author{
Connor Magnus Böckmann \\ email: \href{mailto:connormagnus.boeckmann@tu-dortmund.de}{connormagnus.boeckmann@tu-dortmund.de}
\and Tim Theissel \\ email: \href{mailto:tim.theissel@tu-dortmund.de}{tim.theissel@tu-dortmund.de}  
}
\begin{document}
\maketitle
\newpage
\tableofcontents
\newpage
\section{Zielsetzung}
Das Ziel des Versuchs ist die Bestimmung der Schwingungs- und Schwebungsdauer bei gleichsinnigen-, gegensinnigen-, sowie gekoppelten Schwingungen.
\section{Theoretische Grundlagen}
\subsection{Einfaches Fadenpendel}
Zur Betrachtung zweier gekoppelter Pendel bedarf es zuerst der Betrachtung eines einfachen Pendels der Länge l und der Masse m. Außerdem sei es reibungsfrei aufgehängt. Bei Auslenkung des Pendels wirkt die Gewichtskraft $\vec{F}\textsubscript{g} = m\cdot\vec{a}$ als Rückstellkraft der Bewegung entgegen. Dadurch wird ein Drehmoment M = D\textsubscript{p}\cdot\Phi  auf das Pendel mit der Winkelrichtgröße D\textsubscript{p} und der Auslenkung \Phi   aus der Ruhelage. Die Bewegungsgleichung für ein einzelnes, reibungsfreies Pendel unter Annahme der Kleinwinkelnäherung (sin\theta=\theta) ergibt sich somit zu 
\begin{equation}
    J\cdot\ddot{\Phi}+D\textsubscript{p}\cdot\Phi=0 \label{(1)}
\end{equation}
mit dem Trägheitsmoment J des Pendels. Gelöst wird die Differentialgleichung durch eine harmonische Schwingung. Die Schwingungsfrequenz des Einzelpendels ergibt sich dabei zu  
\begin{displaymath}
    \omega=\sqrt{\frac{D_p}{J}}=\sqrt{\frac{g}{l}}
\end{displaymath}
Aus der Formel lässt sich bereits erkennen, dass die Schwingungsdauer vollkommen unabhängig von der Masse m des Pendels und dem Auslenkungswinkel \Phi ist, vorausgesetzt die Auslenkung genügt der Kleinwinkelnäherung.
Werden zwei Pendel gekoppelt durch eine Feder, wirkt auf das Pendel ein weiteres Drehmoment
\begin{align}
    M_1=D_F(\Phi_2-\Phi_1) \label{(2)}\\
    M_2=D_F(\Phi_1-\Phi_2) \label{(3)}
\end{align}
Die Bewegungsgleichungen für das gekoppelte System werden durch 
\begin{align}
    J\ddot{\Phi_1}+D\Phi_1=D_F(\Phi_2-\Phi_1) \notag\\
    J\ddot{\Phi_2}+D\Phi_2=D_F(\Phi_1-\Phi_2) \notag
\end{align} 
beschrieben. Dabei ist erkennbar, dass die linke Seite der Gleichungen, die vom Einzelpendel bekannte Differentialgleichung aus \ref*{(1)} dargestellt wird. Erweitert wird die rechte Seite durch das Drehmoment aus \ref*{(2)} und \ref*{(3)}, welches durch die Kopplung über die Feder dazu kommt. Gelöst wird das Differentialgleichungssystem erneut durch harmonische Schwingungen mit den Kreisfrequenzen \omega\textsubscript{1} und \omega\textsubscript{2}. \alpha\textsubscript{1} und \alpha\textsubscript{2} bezeichnen dabei die Auslenkungswinkel der Pendel aus der Ruhelage.
\newpage
\subsection{Schwingungsarten}
Es werden verschiedene Arten von Schwingungen bei gekoppelten Pendeln unterschieden, je nach dem wie die Anfangsbedingungen \alpha(t=0) und $\dot{\alpha}(t=0)$ gewählt werden.\\\\
\noindent Gleichsinnige Schwingung für \alpha\textsubscript{1}=\alpha\textsubscript{2}:\\
Beide Pendel werden um den selben Winkel \alpha aus ihrer Ruhelage in die selbe Richtung ausgelenkt, also \alpha\textsubscript{1}=\alpha\textsubscript{2}. Da die Feder dabei weder gestreckt noch gestaucht wird, übt sie dabei keine Kraft auf die Pendel aus. Beide Pendel schwingen also, als wären sie nicht über die Feder verbunden. Die rücktreibende Kraft ist also nur die Gravitationskraft. Genau wie beim Einzelpendel schwingen beide Pendel mit der Schwingungsfrequenz
\begin{displaymath}
    \omega_+=\sqrt{\frac{g}{l}}
\end{displaymath}
Die Schwingungsdauer bei einer gleichsinnigen Schwingung beträgt dann 
\begin{equation}\label{T+}
    T_+ =2\pi\sqrt{\frac{l}{g}}
\end{equation}
\\\\
\noindent Gegensinnige Schwingung für \alpha\textsubscript{1}=-\alpha\textsubscript{2}:\\
Beide Pendel werden um betragsmäßig gleiche Winkel \alpha\textsubscript{1}=-\alpha\textsubscript{2} entgegengesetzt ausgelenkt. Auf beide Pendel wirkt dabei die gleiche, aber entgegengesetzte Kraft durch die Kopplungsfeder. Die entstehende Schwingung ist daher symmetrisch. Die Schwingungsfrequenz ergibt sich dabei zu 
\begin{displaymath}
    \omega_-=\sqrt{\frac{g}{l}+\frac{2K}{l}}
\end{displaymath}
Die Schwingungsdauer ist folglich
\begin{equation}\label{T-}
    T_-=2\pi\sqrt{\frac{l}{g+2K}}
\end{equation}
Die Kopplungskonstante der Feder wird K genannt.
\\\\
Gekoppelte Schwingung für \alpha\textsubscript{1}=0, \alpha\textsubscript{2}\neq0:\\
Bei der gekoppelten Schwingung ist eines der Pendel zu Beginn in seiner Ruhelage, wobei das andere um den Winkel \alpha  ausgelenkt wird. Wird das ausgelenkte Pendel schwingen gelassen, fängt es an seine Energie über die Feder an das andere Pendel zu übertragen. Das andere Pendel beginnt zu schwingen mit steigender Amplitude. Die Amplitude erreicht ihr Maximum in dem Moment in dem das erste Pendel wieder in Ruhe ist. Die Energie wird vollständig übertragen. Der Prozess wiederholt sich immer wieder. Die Schwebungsdauer T\textsubscript{S} -die Zeit zwischen zwei Stillständen eines Pendels- berechnet sich zu 
\begin{equation}\label{TS}
    T_S=\frac{T_+*T_-}{T_+-T_-}
\end{equation}
mit der Schwebungsfrequenz
\begin{displaymath}
    \omega_S=\omega_+-\omega_-
\end{displaymath}
mit der Schwingungsdauer der gleichsinnigen Schwingung T\textsubscript{+} und der Schwingungsdauer der gegensinnigen Schwingung T\textsubscript{-}. \\
\subsection{Kopplungskonstante K}
\noindent Die Kopplungskonstante K ist ein Maß für die Kopplung der Pendel durch die Feder und berechnet sich zu 
\begin{equation}\label{K}
    K=\frac{\omega_-^2-\omega_+^2}{\omega_-^2+\omega_+^2}=\frac{T_+^2-T_-^2}{T_+^2+T_-^2}
\end{equation}
\subsection{Weitere Formeln}
\noindent Arithmetischer Mittelwert:\\
\begin{equation} \label{Mittelwert}  
    x_{arithm}=\frac{1}{N}\cdot \sum \limits_{i=1}^{N}x_i
\end{equation}
Dabei ist N die Anzahl aufgenommener Messwerte und x\textsubscript{i} die Messwerte.
\section{Aufbau des Versuchs}
\noindent Der Aufbau des Versuchs besteht aus zwei Stabpendeln. Die Massen (m=1kg) sind dabei über die Länge des Stabes verschiebbar, um verschiedene Pendellängen einstellbar zu machen und genau gleich lange Pendel zu ermöglichen. Die Pendel sind dabei reibungsarm auf einer Spitzenlagerung gelagert. Das bedeutet, dass am oberen Ende eines jeden Pendels eine Spitze in einer keilförmigen Nut liegt, was die Kontaktfläche und dadurch die Reibung minimiert. Beide Pendel sind über eine Kopplungsfeder mit Kopplungskonstante K verbunden. Diese Feder kann aber auch entfernt werden. Die Schwingungsdauern werden händisch mit einer Stoppuhr gemessen. Ebenso erfolgt die Auslenkung der Pendel nach Augenmaß. 

\section{Messwerte}
In diesem Abschnitt werden einige Messreihen aufgestellt.
Dabei werden die Schingungsdauern (T) und, falls vorhanden, auch die Schwebungsdauern, für alle bereits genannten Schwingungsarten,
jewils 10 mal gemessen.

\noindent Um den Messfehler zu verringern, wurden bei der Durchführung des Versuchs die verschiedenen Schwingungsdauern
nicht für eine Schwingung sondern für 5 Schwingungen gemessen. In den Tabellen zu den Messwerten werden daher für die Schwingungsdauern
immer 2 Werte pro Schwingungsdauer angegeben. Der eine bezieht sich auf die gemessene Zeit (t) und 
der andere auf die tatsächliche Schwingungsdauer (T), welche nur ein fünftel der gemessen Zeit ist.
In den Tabellen, in denen Werte verschiedener Pendellängen dargestellt sind, geben die Indicees der Zeiten in der oberen Zeile die Pendellängen eindeutig an.
Außerdem sind in den Tabellen auch Mittelwerte angegeben. Diese stehen dann jeweils in der gleichen Spalte wie die Werte, auf die sich der Mittelwert bezieht.

\noindent Dies sind alle Tabellen einmal hintereinander dargestellt.



    \begin{minipage}{\linewidth}
        \centering
    \captionof{table}{1m Pendellänge, Pendel nicht durch Feder verbunden}
    \begin{tabular}{llll}
        \toprule
        t\textsubscript{1} (s) & T\textsubscript{1} (s) & t\textsubscript{2} (s) & T\textsubscript{2} (s) \\
        \midrule
        9.9400  &    1.9880  & 9.4000 & 1.8800 \\
        9.3400  &    1.8680  & 9.5700 & 1.9140 \\
        9.2600  &    1.8520  & 9.3800 & 1.8760 \\
        9.2000  &    1.8400  & 9.3300 & 1.8660 \\
        9.4500  &    1.8900  & 9.6000 & 1.9200 \\
        9.5500  &    1.9100  & 9.4500 & 1.8900 \\
        9.4400  &    1.8880  & 9.7700 & 1.9540 \\
        9.5100  &    1.9020  & 9.8700 & 1.9740 \\
        9.3900  &    1.8780  & 9.9100 & 1.9820 \\
        9.5000  &    1.9000  & 9.8900 & 1.9780 \\
        \midrule
        Mittelwerte:\\
        9.4580 & 1.8916 & 9.6170 & 1.9234  \\
        \bottomrule
        
    \end{tabular}
    \label{tab:1}
\end{minipage}

\begin{minipage}{\linewidth}
    \centering
    \captionof{table}{0,5m Pendellänge, Pendel nicht durch Feder verbunden}
    \begin{tabular}{llll}
        \toprule
        t\textsubscript{1} (s) & T\textsubscript{1} (s) & t\textsubscript{2} (s) & T\textsubscript{2} (s) \\
        \midrule
        7.21 & 1.442 & 7.26 & 1.452\\
        7.18 & 1.436 & 7.39 & 1.478\\
        7.23 & 1.446 & 7.25 & 1.450\\
        7.17 & 1.434 & 7.19 & 1.438\\
        7.3  & 1.460 & 7.26 & 1.452\\
        7.25 & 1.450 & 7.34 & 1.468\\
        7.22 & 1.444 & 7.19 & 1.438\\
        7.3  & 1.460 & 7.3  & 1.460\\
        7.25 & 1.450 & 7.24 & 1.448\\
        7.41 & 1.482 & 7.15 & 1.430\\
        \midrule
        Mittelwerte:\\
        7.252 & 1.4504 & 7.257 & 1.4514  \\
        \bottomrule
        
    \end{tabular}
    \label{tab:2}
\end{minipage}
\leavevmode
\newline
\vspace*{1 cm}
\newline



\begin{minipage}{\linewidth}
    \centering
    \captionof{table}{Gleichphasige Schwingung bei verschidenen Pendellängen}
    \begin{tabular}{lllr}
        \toprule
        t\textsubscript{+ (1m)} (s) & T\textsubscript{+ (1m)} (s) 
        & t\textsubscript{+ (0,5m)} (s) & T\textsubscript{+ (0,5m)} (s) \\
        \midrule
        9.470 & 1.8940 & 7.2900 & 1.4580 \\
        9.670 & 1.9340 & 7.1800 & 1.4360 \\
        9.630 & 1.9260 & 7.0400 & 1.4080 \\
        9.460 & 1.8920 & 7.1900 & 1.4380 \\
        9.710 & 1.9420 & 7.3900 & 1.4780 \\
        9.820 & 1.9640 & 7.4000 & 1.4800 \\
        9.370 & 1.8740 & 7.3000 & 1.4600 \\
        9.760 & 1.9520 & 7.2500 & 1.4500 \\
        9.830 & 1.9660 & 7.1800 & 1.4360 \\
        9.850 & 1.9700 & 7.3000 & 1.4600 \\
        \midrule
        Mittelwerte:\\
        9.657 & 1.9314 & 7.2520 & 1.4504\\
        
        \bottomrule
        
    \end{tabular}
    \label{tab:3}
\end{minipage}



\begin{minipage}{\linewidth}
    \centering
    \captionof{table}{Gegenphasige Schwingung bei verschidenen Pendellängen}
    \begin{tabular}{lllr}
        \toprule 
        t\textsubscript{- (1m)} (s) & T\textsubscript{- (1m)} (s) 
        & t\textsubscript{- (0,5m)} (s) & T\textsubscript{- (0,5m)} (s) \\
        \midrule
        8.87 & 1.774 & 5.97 & 1.194 \\
        8.45 & 1.690 & 5.90 & 1.180 \\
        8.64 & 1.728 & 5.92 & 1.184 \\
        8.52 & 1.704 & 5.86 & 1.172 \\
        8.98 & 1.796 & 5.85 & 1.170 \\
        8.55 & 1.710 & 5.93 & 1.186 \\
        8.66 & 1.732 & 5.93 & 1.186 \\
        8.75 & 1.750 & 5.92 & 1.184 \\
        8.85 & 1.770 & 5.84 & 1.168 \\
        8.66 & 1.732 & 5.91 & 1.182 \\
        \midrule
        Mittelwerte:\\
        8.693 & 1.7386 & 5.31 & 1.062\\
        
        \bottomrule
        
    \end{tabular}
\end{minipage}

\leavevmode
\newline
\vspace*{1 cm}
\newline

\begin{minipage}{\linewidth}
    \centering
    \captionof{table}{gekoppelte Schwingung verschiedener Pendellängen}
    \begin{tabular}{lllllr}
        \toprule 
        T\textsubscript{S (1m)} & 
        T\textsubscript{S (0,5m)} &
        t\textsubscript{K (1m)} (s) &
        T\textsubscript{(1m)} & 
        t\textsubscript{K (0,5m)} (s)
        T\textsubscript{(1m)} & \\
        \midrule
        14.42 &  9.83 & 1.966 & 6.31 & 6.30 & 1.26 \\
        14.83 &  9.91 & 1.982 & 6.08 & 7.00 & 1.60 \\
        14.8  & 10.14 & 2.028 & 5.90 & 5.50 & 1.10 \\
        14.42 &  9.64 & 1.928 & 6.41 & 6.58 & 1.316\\
        14.88 &  9.62 & 1.924 & 5.98 & 6.41 & 1.282\\
        14.66 &  9.37 & 1.874 & 5.99 & 5.83 & 1.166\\
        14.40 &  9.74 & 1.948 & 6.08 & 6.30 & 1.26\\
        14.03 &  9.37 & 1.874 & 5.97 & 6.53 & 1.306\\
        15.03 &  9.23 & 1.846 & 6.26 & 6.56 & 1.312\\
        14.16 &  9.31 & 1.862 & 6.01 & 5.50 & 1.1\\
        \midrule
        Mittelwerte:\\
        14.563 & 9.616 & 1.9232 & 6.099 & 6.251 & 1.2502\\
        
        \bottomrule
        
    \end{tabular}
    \label{tab:4}
\end{minipage}

\pagebreak

\section{Auswertung}

Zu beginn dieser Auswertung gilt es, die Schwingungsfrequenzen und den Kopplungsgrad zu bestimmen.
Um von den Schwingungsdauern/der Schwebungsdauer auf die Frequenzen zu kommen wird Folgende Gleichung benötigt:
\begin{displaymath}
    \omega = \frac{2\pi}{T}
\end{displaymath}

Dabei ist mit T jeweils die Schwingungsdauer der zu berechnenden Frequenz gemeint.
Also für \omega\textsubscript{+} auch T\textsubscript{+} usw.

Die Kopplungskonstante K wird nach \ref{K} berechnet.

\begin{minipage}{\linewidth}
    \centering
    \captionof{table}{Frequenzen der drei Schwingungsarten und die Kopplungskonstante}
    \begin{tabular}{lllllr}
        \toprule 
        T\textsubscript{+} & \omega\textsubscript{+} & T\textsubscript{-} & \omega\textsubscript{-} & T\textsubscript{S} & \omega\textsubscript{S} \\
        \midrule 
        1.8940 & 3.317 & 1.774 & 3.542 & 14.42 & 0.436 \\ 
        1.9340 & 3.249 & 1.690 & 3.718 & 14.83 & 0.424 \\ 
        1.9260 & 3.262 & 1.728 & 3.636 & 14.80 & 0.425 \\ 
        1.8920 & 3.321 & 1.704 & 3.687 & 14.42 & 0.436 \\ 
        1.9420 & 3.235 & 1.796 & 3.498 & 14.88 & 0.422 \\ 
        1.9640 & 3.199 & 1.710 & 3.674 & 14.66 & 0.429 \\ 
        1.8740 & 3.353 & 1.732 & 3.628 & 14.40 & 0.436 \\ 
        1.9520 & 3.219 & 1.750 & 3.590 & 14.03 & 0.448 \\ 
        1.9660 & 3.196 & 1.770 & 3.550 & 15.03 & 0.418 \\ 
        1.9700 & 3.189 & 1.732 & 3.628 & 14.16 & 0.444 \\ 
        \midrule
        Mittelwerte \\
               & 3.254 &       & 3.615 &       & 0.432 \\
        \bottomrule
        
    \end{tabular}
\end{minipage}

Die Kopplungskonstante ergibt sich nach \ref{K} zu 0.1048 Hz.
Mit \omega\textsubscript{+}=3.254 \omega\textsubscript{-}=3.615.

\subsection{Aufgabe 1}

Hier werden die Schwingungsdauern T\textsubscript{1} und T\textsubscript{2} bestimmt.

Dazu werden die Messwerte aus \ref{tab:1} und \ref{tab:2} verwendet.

Die gemessenen Zeiten (t) wurden durch 5 geteilt, um die datsächliche Schwingungsdauer zu erhalten.
Die Schwingungsdauern T\textsubscript{1} und T\textsubscript{2} ergeben sich dann jeweils als Mittelwert der 10 gemessenen Schwingungsdauern.
Die Mittelwerte wurden nach der Formel \ref{Mittelwert} berechnet.

Das Ergebnis ist dann:

\begin{center}
    \begin{tabular}{lll}
        \toprule
        Pendellänge (m) & T\textsubscript{1} (s) & T\textsubscript{2} (s)\\
        \midrule 
        1 &1.8916 & 1.9234 \\
        0.5 & 1.4504 & 1.4514 \\
        \bottomrule
    \end{tabular}
\end{center}

\subsection{Aufgabe 2}

Die Schwingungsdauer T\textsubscript{+} ergibt sich nach \ref{T+} zu: 

\begin{center}
    \begin{tabular}{ll}
        \toprule
        Pendellänge (m) & T\textsubscript{+} (s) \\
        \midrule 
        1 & 2.006 \\
        0.5 & 1.419 \\
        \bottomrule
    \end{tabular}
\end{center}
Mit $g=9.81\frac{m}{s²}$ und $l=1m$.

\subsection{Aufgabe 3}

T\textsubscript{-} lässt sich nach \ref{T-} berechnen.
Mit K = 0.1048 folgt:
\begin{center}
    \begin{tabular}{ll}
        \toprule
        Pendellänge (m) & T\textsubscript{-} (s) \\
        \midrule 
        1 & 1.985 \\
        0.5 & 1.404 \\
        \bottomrule
    \end{tabular}
\end{center}

\subsection{Aufgabe 4}

Die Schwingungs- sowie Schwebungsdauern sind gemessen worden 10 mal für 2 Pendellängen gemessen worden und stehen in \ref{tab:4}.

Als Vergleichswert lässt sich die Schwebungsdauer auch berechnen nach \ref{TS}.
Mit den in Aufgabe 2 berechneten Schwingungsdauern ergibt sich für die Schwebungsdauer 189,61s für 1m Pendellänge.
Für das 0,5m lange Pendel beträgt die Schwebungsdauer 132.82s.


\end{document}