\documentclass[titlepage=firstcover, captions=tableheading]{scrartcl}
\usepackage{microtype}
\usepackage{amsmath}
\usepackage{polyglossia}
\usepackage{graphicx}
\usepackage{booktabs}
\usepackage{siunitx}
\usepackage{hyperref}
\usepackage{caption}
\usepackage{float}
\usepackage{parskip}
\setdefaultlanguage{german}
\title{106 gekoppeltes Pendel}
\author{
Connor Magnus Böckmann \\ email: \href{mailto:connormagnus.boeckmann@tu-dortmund.de}{connormagnus.boeckmann@tu-dortmund.de}
\and Tim Theissel \\ email: \href{mailto:tim.theissel@tu-dortmund.de}{tim.theissel@tu-dortmund.de}  
}
\begin{document}
\maketitle
\newpage
\tableofcontents
\newpage
\section{Zielsetzung}
Das Ziel des Versuchs ist die Bestimmung der Schwingungs- und Schwebungsdauer bei gleichsinnigen-, gegensinnigen-, sowie gekoppelten Schwingungen.
\section{Theoretische Grundlagen}
\subsection{Einfaches Fadenpendel}
Zur Betrachtung zweier gekoppelter Pendel bedarf es zuerst der Betrachtung eines einfachen Pendels der Länge l und der Masse m. 
Außerdem sei es reibungsfrei aufgehängt. 
Bei Auslenkung des Pendels wirkt die Gewichtskraft $\vec{F}\textsubscript{g} = m\cdot\vec{a}$ als Rückstellkraft der Bewegung entgegen. 
Dadurch wird ein Drehmoment M = D\textsubscript{p}\cdot\Phi \; auf das Pendel mit der Winkelrichtgröße D\textsubscript{p} und der Auslenkung \Phi \;  aus der Ruhelage. 
Die Bewegungsgleichung für ein einzelnes, reibungsfreies Pendel unter Annahme der Kleinwinkelnäherung (sin\theta=\theta) ergibt sich somit zu 
\begin{equation}
    J\cdot\ddot{\Phi}+D\textsubscript{p}\cdot\Phi=0 \label{(1)}
\end{equation}
mit dem Trägheitsmoment J des Pendels. Gelöst wird die Differentialgleichung durch eine harmonische Schwingung. Die Schwingungsfrequenz des Einzelpendels ergibt sich dabei zu  
\begin{displaymath}
    \omega=\sqrt{\frac{D_p}{J}}=\sqrt{\frac{g}{l}}
\end{displaymath}
Aus der Formel lässt sich bereits erkennen, dass die Schwingungsdauer vollkommen unabhängig von der Masse m des Pendels und dem Auslenkungswinkel \Phi \; ist, vorausgesetzt die Auslenkung genügt der Kleinwinkelnäherung.
Werden zwei Pendel gekoppelt durch eine Feder, wirkt auf das Pendel ein weiteres Drehmoment
\begin{align}
    M_1=D_F(\Phi_2-\Phi_1) \label{(2)}\\
    M_2=D_F(\Phi_1-\Phi_2) \label{(3)}
\end{align}
Die Bewegungsgleichungen für das gekoppelte System werden durch 
\begin{align}
    J\ddot{\Phi_1}+D\Phi_1=D_F(\Phi_2-\Phi_1) \notag\\
    J\ddot{\Phi_2}+D\Phi_2=D_F(\Phi_1-\Phi_2) \notag
\end{align} 
beschrieben. Dabei ist erkennbar, dass die linke Seite der Gleichungen, die vom Einzelpendel bekannte Differentialgleichung aus \ref*{(1)} dargestellt wird. Erweitert wird die rechte Seite durch das Drehmoment aus \ref*{(2)} und \ref*{(3)}, welches durch die Kopplung über die Feder dazu kommt. Gelöst wird das Differentialgleichungssystem erneut durch harmonische Schwingungen mit den Kreisfrequenzen \omega\textsubscript{1} und \omega\textsubscript{2}. \alpha\textsubscript{1} und \alpha\textsubscript{2} bezeichnen dabei die Auslenkungswinkel der Pendel aus der Ruhelage.
\newpage
\subsection{Schwingungsarten}
Es werden verschiedene Arten von Schwingungen bei gekoppelten Pendeln unterschieden, je nach dem wie die Anfangsbedingungen \alpha(t=0) und $\dot{\alpha}(t=0)$ gewählt werden.\\\\
\noindent Gleichsinnige Schwingung für \alpha\textsubscript{1}=\alpha\textsubscript{2}:\\
Beide Pendel werden um den selben Winkel \alpha \; aus ihrer Ruhelage in die selbe Richtung ausgelenkt, also \alpha\textsubscript{1}=\alpha\textsubscript{2}. Da die Feder dabei weder gestreckt noch gestaucht wird, übt sie dabei keine Kraft auf die Pendel aus. Beide Pendel schwingen also, als wären sie nicht über die Feder verbunden. Die rücktreibende Kraft ist also nur die Gravitationskraft. Genau wie beim Einzelpendel schwingen beide Pendel mit der Schwingungsfrequenz
\begin{equation}\label{w+}
    \omega_+=\sqrt{\frac{g}{l}}
\end{equation}
Die Schwingungsdauer bei einer gleichsinnigen Schwingung beträgt dann 
\begin{equation}\label{T+}
    T_+ =2\pi\sqrt{\frac{l}{g}}
\end{equation}
\\\\
\noindent Gegensinnige Schwingung für \alpha\textsubscript{1}=-\alpha\textsubscript{2}:\\
Beide Pendel werden um betragsmäßig gleiche Winkel \alpha\textsubscript{1}=-\alpha\textsubscript{2} entgegengesetzt ausgelenkt. 
Auf beide Pendel wirkt dabei die gleiche, aber entgegengesetzte Kraft durch die Kopplungsfeder. Die entstehende Schwingung ist daher symmetrisch. Die Schwingungsfrequenz ergibt sich dabei zu 
\begin{equation}\label{w-}
    \omega_-=\sqrt{\frac{g}{l}+\frac{2K}{l}}
\end{equation}
Die Schwingungsdauer ist folglich
\begin{equation}\label{T-}
    T_-=2\pi\sqrt{\frac{l}{g+2K}}
\end{equation}
Die Kopplungskonstante der Feder wird K genannt.
\\\\
Gekoppelte Schwingung für \alpha\textsubscript{1}=0, \alpha\textsubscript{2}\neq0:\\
Bei der gekoppelten Schwingung ist eines der Pendel zu Beginn in seiner Ruhelage, wobei das andere um den Winkel \alpha \; ausgelenkt wird. Wird das ausgelenkte Pendel schwingen gelassen, fängt es an seine Energie über die Feder an das andere Pendel zu übertragen. Das andere Pendel beginnt zu schwingen mit steigender Amplitude. Die Amplitude erreicht ihr Maximum in dem Moment in dem das erste Pendel wieder in Ruhe ist. Die Energie wird vollständig übertragen. Der Prozess wiederholt sich immer wieder. Die Schwebungsdauer T\textsubscript{S} -die Zeit zwischen zwei Stillständen eines Pendels- berechnet sich zu 
\begin{equation}\label{TS}
    T_S=\frac{T_+*T_-}{T_+-T_-}
\end{equation}
mit der Schwebungsfrequenz
\begin{displaymath}
    \omega_S=\omega_+-\omega_-
\end{displaymath}
mit der Schwingungsdauer der gleichsinnigen Schwingung T\textsubscript{+} und der Schwingungsdauer der gegensinnigen Schwingung T\textsubscript{-}. \\
\subsection{Kopplungskonstante K}
\noindent Die Kopplungskonstante K ist ein Maß für die Kopplung der Pendel durch die Feder und berechnet sich zu 
\begin{equation}\label{K}
    K=\frac{\omega_-^2-\omega_+^2}{\omega_-^2+\omega_+^2}=\frac{T_+^2-T_-^2}{T_+^2+T_-^2}
\end{equation}

\section{Aufbau des Versuchs}
\noindent Der Aufbau des Versuchs besteht aus zwei Stabpendeln. Die Massen (m=1kg) sind dabei über die Länge des Stabes verschiebbar, um verschiedene Pendellängen einstellbar zu machen und genau gleich lange Pendel zu ermöglichen. Die Pendel sind dabei reibungsarm auf einer Spitzenlagerung gelagert. Das bedeutet, dass am oberen Ende eines jeden Pendels eine Spitze in einer keilförmigen Nut liegt, was die Kontaktfläche und dadurch die Reibung minimiert. Beide Pendel sind über eine Kopplungsfeder mit Kopplungskonstante K verbunden. Diese Feder kann aber auch entfernt werden. Die Schwingungsdauern werden händisch mit einer Stoppuhr gemessen. Ebenso erfolgt die Auslenkung der Pendel nach Augenmaß. 

\section{Auswertung}

\subsection{Aufgabe 1}

Ziel der Aufgabe ist die Bestimmung der Schwingungsdauern T\textsubscript{1} und T\textsubscript{2}.
Dazu werden gemessenen Zeiten (t) durch 5 geteilt, um so die gemessenen Schwingungsdauern zu erhalten und 
die Schwingungsdauern T\textsubscript{1} und T\textsubscript{2} ergeben sich dann jeweils als Mittelwert der 10 gemessenen Schwingungsdauern.
Die Mittelwerte werden nach der Formel \ref{Mittelwert} berechnet.

Allerdings muss in diesem Fall noch die Standardabweichung bestimmt werden, da die gemessenen Werte Unsicherheiten besitzen.
Diese werden nach der Formel \ref{Standardabweichung}


Es wurden die folgenden Messwerte verwendet: 

\begin{minipage}{\linewidth}
\centering
    \captionof{table}{1m Pendellänge, Pendel nicht durch Feder verbunden}
    \centering
    \begin{tabular}{llll}
    \toprule
    t\textsubscript{1} (s) & T\textsubscript{1} (s) & t\textsubscript{2} (s) & T\textsubscript{2} (s) \\
    \midrule
    9.9400  &    1.9880  & 9.4000 & 1.8800 \\
    9.3400  &    1.8680  & 9.5700 & 1.9140 \\
    9.2600  &    1.8520  & 9.3800 & 1.8760 \\
    9.2000  &    1.8400  & 9.3300 & 1.8660 \\
    9.4500  &    1.8900  & 9.6000 & 1.9200 \\
    9.5500  &    1.9100  & 9.4500 & 1.8900 \\
    9.4400  &    1.8880  & 9.7700 & 1.9540 \\
    9.5100  &    1.9020  & 9.8700 & 1.9740 \\
    9.3900  &    1.8780  & 9.9100 & 1.9820 \\
    9.5000  &    1.9000  & 9.8900 & 1.9780 \\
    \midrule
    Mittelwerte:\\
    9.4580 & 1.8916 & 9.6170 & 1.9234  \\
    \bottomrule
    
\end{tabular}
\label{tab:1}
\leavevmode
\newline
\vspace*{1 cm}
\newline
\end{minipage}

\begin{minipage}{\linewidth}
    \centering
    \captionof{table}{0,5m Pendellänge, Pendel nicht durch Feder verbunden}
    \begin{tabular}{llll}
        \toprule
        t\textsubscript{1} (s) & T\textsubscript{1} (s) & t\textsubscript{2} (s) & T\textsubscript{2} (s) \\
        \midrule
        7.21 & 1.442 & 7.26 & 1.452\\
        7.18 & 1.436 & 7.39 & 1.478\\
        7.23 & 1.446 & 7.25 & 1.450\\
        7.17 & 1.434 & 7.19 & 1.438\\
        7.3  & 1.460 & 7.26 & 1.452\\
        7.25 & 1.450 & 7.34 & 1.468\\
        7.22 & 1.444 & 7.19 & 1.438\\
        7.3  & 1.460 & 7.3  & 1.460\\
        7.25 & 1.450 & 7.24 & 1.448\\
        7.41 & 1.482 & 7.15 & 1.430\\
        \midrule
        Mittelwerte:\\
        7.252 & 1.4504 & 7.257 & 1.4514  \\
        \bottomrule
        
    \end{tabular}
    \label{tab:2}
\end{minipage}

\pagebreak

Daraus ergibt sich als Ergebnis:
\begin{center}
    \begin{tabular}{ll @{$\pm$} l l@{$\pm$}l }
        \toprule
        Pendellänge (m) & T\textsubscript{1} (s) & \sigma\textsubscript{1} & T\textsubscript{2} (s) & \sigma\textsubscript{2}\\
        \midrule 
        1 &1.8916 & 0.0802 & 1.9234 & 0.1292\\
        0.5 & 1.4504 & 0.0299 & 1.4514 & 0.0379\\
        \bottomrule
    \end{tabular}
\end{center}

\subsection{Aufgabe 2}

Analog zu Aufgabe 1 lässt sich auch die die Schwingungsdauer für eine gleichphasige Schwingung der gekoppelten Pendel bestimmen.
Diesesmal wird die Formel \ref{T+}
Auch wird für die Standardabweichung die Formel \ref{Standardabweichung} und für den Mittelwert die Formel \ref{Mittelwert} verwendet.

\begin{minipage}{\linewidth}
    \centering
    \captionof{table}{Gleichphasige Schwingung bei verschidenen Pendellängen}
    \begin{tabular}{lllr}
        \toprule
        t\textsubscript{+ (1m)} (s) & T\textsubscript{+ (1m)} (s) 
        & t\textsubscript{+ (0,5m)} (s) & T\textsubscript{+ (0,5m)} (s) \\
        \midrule
        9.470 & 1.8940 & 7.2900 & 1.4580 \\
        9.670 & 1.9340 & 7.1800 & 1.4360 \\
        9.630 & 1.9260 & 7.0400 & 1.4080 \\
        9.460 & 1.8920 & 7.1900 & 1.4380 \\
        9.710 & 1.9420 & 7.3900 & 1.4780 \\
        9.820 & 1.9640 & 7.4000 & 1.4800 \\
        9.370 & 1.8740 & 7.3000 & 1.4600 \\
        9.760 & 1.9520 & 7.2500 & 1.4500 \\
        9.830 & 1.9660 & 7.1800 & 1.4360 \\
        9.850 & 1.9700 & 7.3000 & 1.4600 \\
        \midrule
        Mittelwerte:\\
        9.657 & 1.9314 & 7.2520 & 1.4504\\
        
        \bottomrule
        
    \end{tabular}
    \label{tab:3}
\end{minipage}

Mit den Messwerten aus der obigen Tabelle ergibt sich für T\textsubscript{+}:

\begin{center}
    \begin{tabular}{ll@{$\pm$}l}
        \toprule
        Pendellänge (m) & T\textsubscript{+} (s) & \sigma\\
        \midrule 
        1 & 1.9314 & 0.0958 \\
        0.5 & 1.4504 & 0.0645 \\
        \bottomrule
    \end{tabular}
\end{center}

\subsection{Aufgabe 3}

Auch die gegenphasige Schwingungsdauer T\textsubscript{-} lässt sich auf die gleiche Weise mit \ref{T-} bestimmen.
Wieder wurden die Formeln \ref{Mittelwert} und \ref{Standardabweichung} verwendet.

Die Messwerte sind: 


\begin{minipage}{\linewidth}
    \centering
    \captionof{table}{Gegenphasige Schwingung bei verschiedenen Pendellängen}
    \begin{tabular}{lllr}
        \toprule 
        t\textsubscript{- (1m)} (s) & T\textsubscript{- (1m)} (s) 
        & t\textsubscript{- (0,5m)} (s) & T\textsubscript{- (0,5m)} (s) \\
        \midrule
        8.870 & 1.7740 & 5.97 & 1.1940 \\
        8.450 & 1.6900 & 5.90 & 1.1800 \\
        8.640 & 1.7280 & 5.92 & 1.1840 \\
        8.520 & 1.7040 & 5.86 & 1.1720 \\
        8.980 & 1.7960 & 5.85 & 1.1700 \\
        8.550 & 1.7100 & 5.93 & 1.1860 \\
        8.660 & 1.7320 & 5.93 & 1.1860 \\
        8.750 & 1.7500 & 5.92 & 1.1840 \\
        8.850 & 1.7700 & 5.84 & 1.1680 \\
        8.660 & 1.7320 & 5.91 & 1.1820 \\
        \midrule
        Mittelwerte:\\
        8.693 & 1.7386 & 5.31 & 1.1806\\
        
        \bottomrule
        
    \end{tabular}
\end{minipage}

Daraus lässt sich T\textsubscript{-} berechnen zu:

\begin{center}
    \begin{tabular}{ll@{$\pm$}l}
        \toprule
        Pendellänge (m) & T\textsubscript{-} (s) & \sigma\\
        \midrule 
        1 & 1.7386 & 0.1010 \\
        0.5 & 1.1806 & 0.0246 \\
        \bottomrule
    \end{tabular}
\end{center}

\subsection{Aufgabe 4}

In dieser Aufgabe sollen die Schwingungsdauern T\textsubscript{K} und Schwebungsdauern T\textsubscript{S} für eine gekoppelte Schwingung bestimmt werden.
T\textsubscript{K} lässt sich, nach dem in Aufgabe 1 vorgestellten Verfahren, bestimmen.
Für T\textsubscript{S} gilt es zu beachten, dass die gemessenen Schwebungsdauern nicht durch 5 geteilt werden müssen, da diese in einer Schwebung gemessen wurden und nicht in 5.
Für T\textsubscript{K} und T\textsubscript{S} müssen auch hier wieder Mittelwert und Standardabweichung nach \ref{Mittelwert} und \ref{Standardabweichung} berechnet werden.

\begin{minipage}{\linewidth}
    \centering
    \captionof{table}{gekoppelte Schwingung verschiedener Pendellängen}
    \begin{tabular}{lllllr}
        \toprule 
        T\textsubscript{S (1m)} (s) & 
        t\textsubscript{K (1m)} (s) &
        T\textsubscript{(1m)} (s) & 
        T\textsubscript{S (0,5m)} (s) &
        t\textsubscript{K (0,5m)} (s) &
        T\textsubscript{(0.5m)} (s)\\
        \midrule
        14.420 &  9.830 & 1.9660 & 6.310 & 6.300 & 1.2600 \\
        14.830 &  9.910 & 1.9820 & 6.080 & 7.000 & 1.6000 \\
        14.800 & 10.140 & 2.0280 & 5.900 & 5.500 & 1.1000 \\
        14.420 &  9.640 & 1.9280 & 6.410 & 6.580 & 1.3160 \\
        14.880 &  9.620 & 1.9240 & 5.980 & 6.410 & 1.2820 \\
        14.660 &  9.370 & 1.8740 & 5.990 & 5.830 & 1.1660 \\
        14.400 &  9.740 & 1.9480 & 6.080 & 6.300 & 1.2600 \\
        14.030 &  9.370 & 1.8740 & 5.970 & 6.530 & 1.3060 \\
        15.030 &  9.230 & 1.8460 & 6.260 & 6.560 & 1.3120 \\
        14.160 &  9.310 & 1.8620 & 6.010 & 5.500 & 1.1000 \\
        \midrule
        Mittelwerte:\\
        14.563 & 9.616 & 1.9232 & 6.099 & 6.251 & 1.2502\\
        
        \bottomrule
        
    \end{tabular}
    \label{tab:4}
\end{minipage}

Als Ergebnis ergeben sich folgende Werte:

\begin{center}
    \begin{tabular}{
        l
        l@{$\pm$}l
        l@{$\pm$}l
        }
        \toprule
        Pendellänge (m) & T\textsubscript{S} (s) & \sigma & T\textsubscript{K} (s) & \sigma \\
        \midrule 
        1 & 14.563 & 0.9743 & 1.9232 & 0.1722\\
        0.5 & 6.099 & 0.4627 & 1.2502 & 0.4324 \\
        \bottomrule
    \end{tabular}
\end{center}

\subsection{Aufgabe 5}

Der Kopplungsgrad K kann aus den Schwingungsdauern nach Gleichung \ref{K} brechnet werden.
Da diese Formel allerdings bedeutet, dass der Kopplungsgrad durch fehlerbehaftete Größen berechnet wird, muss hier der Gauß-Fehler nach \ref{Gauß} bestimmt werden.
Mit den Schwingungsdauern aus Aufgabe 2 und 3 erhält man folgende Werte für den Kopplungsgrad:

\begin{center}
    \begin{tabular}{ll@{$\pm$}l}
        \toprule
        Pendellänge (m) & K & Gauß-Fehler\\
        \midrule 
        1 & 0.1048 & 0.0755 \\
        0.5 & 0.2030 & 0.0471 \\
        \bottomrule
    \end{tabular}
\end{center}

Der Kopplungsgrad lässt sich auch durch die Frequenzen bestimmen.
Dafür müssen aus den Schwingungsdauern noch die Frequenzen bestimmt werden.
Um die Frequenzen aus den Schwingungsdauern zu berechnen lohnt sich ein Vergleich zwischen \ref{w+} und \ref{T+}, sowie \ref{w-} und \ref{T-}.

\noindent Es fällt auf, dass gilt:
\begin{align}
    T_{+/-} = \frac{2\pi}{\omega_{+/-}}\\
    \omega_{+/-} = \frac{2\pi}{T_{+/-}}
\end{align}

Der Gauß-Fehler lässt sich nach \ref{Gauß} berechnen.
Mit den Mittelwerten der Schwingungsdauern und deren Unsicherheit aus Aufgabe 2 und Aufgabe 3 ergibt sich für die Frequenzen:

\begin{center}
    \begin{tabular}{ll@{$\pm$}ll@{$\pm$}l}
        \toprule
        Pendellänge (m) & \omega\textsubscript{+} & Gauß-Fehler & \omega\textsubscript{-} & Gauß-Fehler\\
        \midrule 
        1 & 3.2532 & 0.1614 & 3.6139 & 0.2099\\
        0.5 & 4.3320 & 0.1926 & 5.3220 & 0.4553\\
        \bottomrule
    \end{tabular}
\end{center}

Werden diese Werte nun in \ref{K} eingesetzt ist K aus \omega :

\begin{center}
    \begin{tabular}{ll@{$\pm$}l}
        \toprule
        Pendellänge (m) & K & Gauß-Fehler\\
        \midrule 
        1 & 0.1048 & 4.8516 \\
        0.5 & 0.2030 & 3.2161 \\
        \bottomrule
    \end{tabular}
\end{center}

\subsection{Aufgabe 6}

Die Schwebungsdauer T\textsubscript{S} lässt sich nach \ref{TS} berechnen.
Da T\textsubscript{+} und T\textsubscript{-} fehlerbehaftete Größen sind, muss auch hier der Gauß-Fehler berechnet werden.
Die Werte für die Schwingungsdauern sind:

\begin{center}
    \begin{tabular}{ll @{$\pm$} l l@{$\pm$}l }
        \toprule
        Pendellänge (m) & T\textsubscript{+} (s) & \sigma & T\textsubscript{-} (s) & \sigma \\
        \midrule 
        1 & 1.9314 & 0.0958 & 1.7386 & 0.1010\\
        0.5 & 1.4504 & 0.0645 & 1.1806 & 0.0246\\
        \bottomrule
    \end{tabular}
\end{center}

Für die Schwebungsdauer ergibt sich dann:

\begin{center}
    \begin{tabular}{ll@{$\pm$}l}
        \toprule
        Pendellänge (m) & T\textsubscript{S} & Gauß-Fehler\\
        \midrule 
        1 & 17.4167 & 12.7836 \\
        0.5 & 6.3467 & 1.4250 \\
        \bottomrule
    \end{tabular}
\end{center}
 
\subsection{Fehlerrechnung}
\noindent Arithmetischer Mittelwert:\\
\begin{equation} \label{Mittelwert}  
    x_{arithm}=\frac{1}{N}\cdot \sum \limits_{k=1}^{N}x_k
\end{equation}
Dabei ist N die Anzahl aufgenommener Messwerte und x\textsubscript{k} sind die Messwerte.

Standardabweichung: \\
\begin{equation}\label{Standardabweichung}
    \sigma = \sqrt{\frac{1}{N(N-1)} \sum_{k=1}^{n} (x_k - \bar{x})^2}
\end{equation}
Hier ist N wider die Anzahl aufgenommener Messwerte, x\textsubscript{k} sind die Messwerte und $\bar{x}$ ist der arithmetische Mittelwert.   

\begin{equation} \label{Gauß}
    \Delta f = \sqrt{\left(\frac{(\partial f)}{(\partial x)}\right)^2 (\Delta x)² +
                     \left(\frac{(\partial f)}{(\partial y)}\right)^2 (\Delta y)² + ... +
                     \left(\frac{(\partial f)}{(\partial z)}\right)^2 (\Delta z)²
    }
\end{equation}

\section{Diskussion}

\subsection{Vergleich Aufgabe 6}

Für einen Vergleich der Messwerte mit den berechneten Schwebungsdauern sind hier nocheinmal die zu vergleichenden Werte dargestellt:

\begin{center}
    \begin{tabular}{ll@{$\pm$}ll@{$\pm$}l}
        \toprule
        Pendellänge (m) & T\textsubscript{S} (berechnet) & Gauß-Fehler & T\textsubscript{S} (gemessen) & \sigma\\
        \midrule 
        1   & 17.4167 & 12.7836 & 14.563 & 0.9743 \\
        0.5 &  6.3467 &  1.4250 &  6.099 & 0.4627 \\
        \bottomrule
    \end{tabular}
\end{center}

\subsection{Schwebungsdauer}

Die gemessenen Werte liegen nah zwar an den berechneten Werten, jedoch ist eine Abweichung festzustellen.
Diese ist darauf zurückzuführen, dass bei diesem Versuch die Schwebungsdauer mit bloßem Auge gemessen wurde.
Ebenso wurde auch die zum Start benötigte Auslenkung mit der Hand bis zu einem Punkt ausgeführt.
Auch dabei ist es nicht möglich immer genau den gleichen Punkt zu treffen.
Eine weitere Schwierigkeit dieses Vergleichs sind die Gauß-Fehler der berechneten Werte.
Die Gauß-Fehler sind so groß, weil die Werte aus vier (zwei verschiedenen) fehlerbehafteten Größen bestimmt wurden 
und sich die Unsicherheit der gemessenen Schwingungsdauern so vervielfältigt.
Die berechneten Schwebungsdauern an sich liegen dabei sehr nah an den gemessenen Schwebungsdauern aber die Gauß-Fehler lassen,
durch den großen Bereich, den sie für die berechneten Werte zulassen,
kaum eine qualifizierte Aussage über die Abweicheng dieser Schwebungsdauern zu.

\subsection{Allgemeine Probleme des Versuchs und Verbesserungsvorschläge}

Um alle Messwerte genauer zu bestimmen könnte bei der Durchführung des Versuchs auf Lichtschranken zurückgegriffen werden. 
Dann stoppt die Zeit immer genau zu dem Zeitpunkt an dem gestoppt werden soll und die Reaktionszeit bis zum Stoppen ist geringer.
Ebenso könnte eine mechanische Sperre anbringen, die es ermöglicht präzise die Pendel auszulenken. 
Dies würde es ermöglichen, eine nahezu perfekt gleichphasige oder nahezu perfekt gegenphasige Schwingung zu erhalten. 
Bei der Auslenkung mit der Hand kommt es häufig zu Schwingungen, die in gekoppelten Schwingungen enden, anstatt gleichphasig oder gegenphasig zu bleiben.
Es entsteht also ein Messfehler, weil es nahezu nicht möglich ist mit der Hand die Pendel so exakt auszulenken.

Ein weiteres Problem dieses Versuchs liegt bei den berechneten Werten. Diese werden nahezu ausschließlich aus den fehlerbehafteten Messwerten berechnet,
was dazu führt, dass die Gauß-Fehler bei einigen Werten sehr hoch ausfallen.

\section{Literatur}

Anleitung V106 Gekoppelte Pendel TU Dortmund
%https://moodle.tu-dortmund.de/pluginfile.php/1368482/mod_resource/content/2/V106.pdf
Formelsammlung zur Berechnung von Messunsicherheiten
%https://moodle.tu-dortmund.de/pluginfile.php/1277100/mod_resource/content/1/FehlerFormeln.pdf

\section{Originalwerte}

In diesem Kapitel sind alle, bei dem Versuch aufgenommenen unbearbeiteten, Messwerte:

\begin{minipage}{\linewidth}
    \centering
\captionof{table}{Gemessene Zeiten für alle Schwingungsarten bei einer Pendellänge von einem Meter}
\begin{tabular}{llllll}
    \toprule
    t\textsubscript{1} (s) & t\textsubscript{2} & t\textsubscript{+} & t\textsubscript{-} & T\textsubscript{S} & t\textsubscript{K}\\
    \midrule
    9.9400 & 9.4000 & 9.47 & 8.87 & 14.42 & 9.83 \\
    9.3400 & 9.5700 & 9.67 & 8.45 & 14.83 & 9.91 \\
    9.2600 & 9.3800 & 9.63 & 8.64 & 14.80 & 10.14 \\
    9.2000 & 9.3300 & 9.46 & 8.52 & 14.42 & 9.64 \\
    9.4500 & 9.6000 & 9.71 & 8.98 & 14.88 & 9.62 \\
    9.5500 & 9.4500 & 9.82 & 8.55 & 14.66 & 9.37 \\
    9.4400 & 9.7700 & 9.37 & 8.66 & 14.40 & 9.74 \\
    9.5100 & 9.8700 & 9.76 & 8.75 & 14.03 & 9.37 \\
    9.3900 & 9.9100 & 9.83 & 8.85 & 15.03 & 9.23 \\
    9.5000 & 9.8900 & 9.85 & 8.66 & 14.16 & 9.31  \\
    \bottomrule
    
\end{tabular}
\end{minipage}

\begin{minipage}{\linewidth}
    \centering
\captionof{table}{Gemessene Zeiten für alle Schwingungsarten bei einer Pendellänge von einem halben Meter}
\begin{tabular}{llllll}
    \toprule
    t\textsubscript{1} (s) & t\textsubscript{2} & t\textsubscript{+} & t\textsubscript{-} & T\textsubscript{S} & t\textsubscript{K}\\
    \midrule
    7.21 & 	7.26 & 7.29  & 5.97 & 6.31 & 6.3  \\
    7.18 & 	7.39 & 7.18  & 5.9 &  6.08 & 7.0 \\
    7.23 & 	7.25 & 7.04  & 5.92 & 5.9  & 5.5 \\
    7.17 & 	7.19 & 7.19  & 5.86 & 6.41 & 6.58  \\
    7.3  &  7.26 & 7.39  & 5.85 & 5.98 & 6.41  \\
    7.25 & 	7.34 & 7.4  &  5.93 & 5.99 & 5.83 \\
    7.22 & 	7.19 & 7.3  &  5.93 & 6.08 & 6.3 \\
    7.3  &  7.3  & 7.25  & 5.92 & 5.97 & 6.53  \\
    7.25 & 	7.24 & 7.18  & 5.84 & 6.26 & 6.56  \\
    7.41 & 	7.15 & 7.3  &  5.91 & 6.01 & 5.5 \\
    \bottomrule
    
\end{tabular}
\end{minipage}


\end{document}