\documentclass[titlepage=firstcover, captions=tableheading]{scrartcl}
\usepackage{microtype}
\usepackage{amsmath}
\usepackage{polyglossia}
\usepackage{graphicx}
\usepackage{booktabs}
\usepackage{siunitx}
\usepackage{hyperref}
\usepackage{caption}
\usepackage{float}
\setdefaultlanguage{german}
\title{106 gekoppeltes Pendel}
\author{
Connor Magnus Böckmann \\ email: \href{mailto:connormagnus.boeckmann@tu-dortmund.de}{connormagnus.boeckmann@tu-dortmund.de}
\and Tim Theissel \\ email: \href{mailto:tim.theissel@tu-dortmund.de}{tim.theissel@tu-dortmund.de}  
}
\begin{document}
\maketitle
\newpage
\tableofcontents
\newpage
\section{Zielsetzung}
Das Ziel des Versuchs ist die Bestimmung der Schwingungs- und Schwebungsdauer bei gleichsinnigen-, gegensinnigen-, sowie gekoppelten Schwingungen.
\section{Theoretische Grundlagen}
Zur Betrachtung zweier gekoppelter Pendel bedarf es zuerst der Betrachtung eines einfachen Pendels der Länge l und der Masse m. Außerdem sei es reibungsfrei aufgehängt. Bei Auslenkung des Pendels wirkt die Gewichtskraft $\vec{F}\textsubscript{g} = m\cdot\vec{a}$ als Rückstellkraft der Bewegung entgegen. Dadurch wird ein Drehmoment M = D\textsubscript{p}\cdot\Phi  auf das Pendel mit der Winkelrichtgröße D\textsubscript{p} und der Auslenkung \Phi   aus der Ruhelage. Die Bewegungsgleichung für ein einzelnes, reibungsfreies Pendel unter Annahme der Kleinwinkelnäherung (sin\theta=\theta) ergibt sich somit zu 
\begin{displaymath}
    J\cdot\ddot{\Phi}+D\textsubscript{p}\cdot\Phi=0
\end{displaymath}
mit dem Trägheitsmoment J des Pendels. Gelöst wird die Differentialgleichung durch eine harmonische Schwingung. Die Schwingungsfrequenz ergibt sich dabei zu  
\begin{displaymath}
    \omega=\sqrt{\frac{D_p}{J}}=\sqrt{\frac{g}{l}}
\end{displaymath}
Aus der Formel lässt sich bereits erkennen, dass die Schwingungsdauer vollkommen unabhängig von der Masse m des Pendels und dem Auslenkungswinkel \Phi ist, vorausgesetzt die Auslenkung genügt der Kleinwinkelnäherung.
\end{document}