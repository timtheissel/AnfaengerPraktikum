\documentclass[titlepage=firstcover, captions=tableheading]{scrartcl}
\usepackage{microtype}
\usepackage{amsmath}
\usepackage{polyglossia}
\usepackage{graphicx}
\usepackage{booktabs}
\usepackage{siunitx}
\usepackage{hyperref}
\usepackage{caption}
\usepackage{float}
\usepackage{parskip}
\begin{document}


\section{Messwerte}
In diesem Abschnitt werden einige Messreihen aufgestellt.
Dabei werden die Schingungsdauern (T) und, falls vorhanden, auch die Schwebungsdauern, für alle bereits genannten Schwingungsarten,
jewils 10 mal gemessen.

\noindent Um den Messfehler zu verringern, wurden bei der Durchführung des Versuchs die verschiedenen Schwingungsdauern
nicht für eine Schwingung sondern für 5 Schwingungen gemessen. In den Tabellen zu den Messwerten werden daher für die Schwingungsdauern
immer 2 Werte pro Schwingungsdauer angegeben. Der eine bezieht sich auf die gemessene Zeit (t) und 
der andere auf die tatsächliche Schwingungsdauer (T), welche nur ein fünftel der gemessen Zeit ist.
In den Tabellen, in denen Werte verschiedener Pendellängen dargestellt sind, geben die Indicees der Zeiten in der oberen Zeile die Pendellängen eindeutig an.
Außerdem sind in den Tabellen auch Mittelwerte angegeben. Diese stehen dann jeweils in der gleichen Spalte wie die Werte, auf die sich der Mittelwert bezieht.

\noindent Dies sind alle Tabellen einmal hintereinander dargestellt.



    \begin{minipage}{\linewidth}
        \centering
    \captionof{table}{1m Pendellänge, Pendel nicht durch Feder verbunden}
    \begin{tabular}{llll}
        \toprule
        t\textsubscript{1} (s) & T\textsubscript{1} (s) & t\textsubscript{2} (s) & T\textsubscript{2} (s) \\
        \midrule
        9.9400  &    1.9880  & 9.4000 & 1.8800 \\
        9.3400  &    1.8680  & 9.5700 & 1.9140 \\
        9.2600  &    1.8520  & 9.3800 & 1.8760 \\
        9.2000  &    1.8400  & 9.3300 & 1.8660 \\
        9.4500  &    1.8900  & 9.6000 & 1.9200 \\
        9.5500  &    1.9100  & 9.4500 & 1.8900 \\
        9.4400  &    1.8880  & 9.7700 & 1.9540 \\
        9.5100  &    1.9020  & 9.8700 & 1.9740 \\
        9.3900  &    1.8780  & 9.9100 & 1.9820 \\
        9.5000  &    1.9000  & 9.8900 & 1.9780 \\
        \midrule
        Mittelwerte:\\
        9.4580 & 1.8916 & 9.6170 & 1.9234  \\
        \bottomrule
        
    \end{tabular}
\end{minipage}

\begin{minipage}{\linewidth}
    \centering
    \captionof{table}{0,5m Pendellänge, Pendel nicht durch Feder verbunden}
    \begin{tabular}{llll}
        \toprule
        t\textsubscript{1} (s) & T\textsubscript{1} (s) & t\textsubscript{2} (s) & T\textsubscript{2} (s) \\
        \midrule
        7.21 & 1.442 & 7.26 & 1.452\\
        7.18 & 1.436 & 7.39 & 1.478\\
        7.23 & 1.446 & 7.25 & 1.450\\
        7.17 & 1.434 & 7.19 & 1.438\\
        7.3  & 1.460 & 7.26 & 1.452\\
        7.25 & 1.450 & 7.34 & 1.468\\
        7.22 & 1.444 & 7.19 & 1.438\\
        7.3  & 1.460 & 7.3  & 1.460\\
        7.25 & 1.450 & 7.24 & 1.448\\
        7.41 & 1.482 & 7.15 & 1.430\\
        \midrule
        Mittelwerte:\\
        7.252 & 1.4504 & 7.257 & 1.4514  \\
        \bottomrule
        
    \end{tabular}
\end{minipage}
\leavevmode
\newline
\vspace*{1 cm}
\newline



\begin{minipage}{\linewidth}
    \centering
    \captionof{table}{Gleichphasige Schwingung bei verschidenen Pendellängen}
    \begin{tabular}{lllr}
        \toprule
        t\textsubscript{+ (1m)} (s) & T\textsubscript{+ (1m)} (s) 
        & t\textsubscript{+ (0,5m)} (s) & T\textsubscript{+ (0,5m)} (s) \\
        \midrule
        9.470 & 1.8940 & 7.2900 & 1.4580 \\
        9.670 & 1.9340 & 7.1800 & 1.4360 \\
        9.630 & 1.9260 & 7.0400 & 1.4080 \\
        9.460 & 1.8920 & 7.1900 & 1.4380 \\
        9.710 & 1.9420 & 7.3900 & 1.4780 \\
        9.820 & 1.9640 & 7.4000 & 1.4800 \\
        9.370 & 1.8740 & 7.3000 & 1.4600 \\
        9.760 & 1.9520 & 7.2500 & 1.4500 \\
        9.830 & 1.9660 & 7.1800 & 1.4360 \\
        9.850 & 1.9700 & 7.3000 & 1.4600 \\
        \midrule
        Mittelwerte:\\
        9.657 & 1.9314 & 7.2520 & 1.4504\\
        
        \bottomrule
        
    \end{tabular}
\end{minipage}



\begin{minipage}{\linewidth}
    \centering
    \captionof{table}{Gegenphasige Schwingung bei verschidenen Pendellängen}
    \begin{tabular}{lllr}
        \toprule 
        t\textsubscript{- (1m)} (s) & T\textsubscript{- (1m)} (s) 
        & t\textsubscript{- (0,5m)} (s) & T\textsubscript{- (0,5m)} (s) \\
        \midrule
        8.87 & 1.774 & 5.97 & 1.194 \\
        8.45 & 1.690 & 5.90 & 1.180 \\
        8.64 & 1.728 & 5.92 & 1.184 \\
        8.52 & 1.704 & 5.86 & 1.172 \\
        8.98 & 1.796 & 5.85 & 1.170 \\
        8.55 & 1.710 & 5.93 & 1.186 \\
        8.66 & 1.732 & 5.93 & 1.186 \\
        8.75 & 1.750 & 5.92 & 1.184 \\
        8.85 & 1.770 & 5.84 & 1.168 \\
        8.66 & 1.732 & 5.91 & 1.182 \\
        \midrule
        Mittelwerte:\\
        8.693 & 1.7386 & 5.31 & 1.062\\
        
        \bottomrule
        
    \end{tabular}
\end{minipage}

\leavevmode
\newline
\vspace*{1 cm}
\newline

\begin{minipage}{\linewidth}
    \centering
    \captionof{table}{gekoppelte Schwingung verschiedener Pendellängen}
    \begin{tabular}{lllllr}
        \toprule 
        T\textsubscript{S (1m)} & 
        T\textsubscript{S (0,5m)} &
        t\textsubscript{K (1m)} (s) &
        T\textsubscript{(1m)} & 
        t\textsubscript{K (0,5m)} (s)
        T\textsubscript{(1m)} & \\
        \midrule
        14.42 &  9.83 & 1.966 & 6.31 & 6.30 & 1.26 \\
        14.83 &  9.91 & 1.982 & 6.08 & 7.00 & 1.60 \\
        14.8  & 10.14 & 2.028 & 5.90 & 5.50 & 1.10 \\
        14.42 &  9.64 & 1.928 & 6.41 & 6.58 & 1.316\\
        14.88 &  9.62 & 1.924 & 5.98 & 6.41 & 1.282\\
        14.66 &  9.37 & 1.874 & 5.99 & 5.83 & 1.166\\
        14.40 &  9.74 & 1.948 & 6.08 & 6.30 & 1.26\\
        14.03 &  9.37 & 1.874 & 5.97 & 6.53 & 1.306\\
        15.03 &  9.23 & 1.846 & 6.26 & 6.56 & 1.312\\
        14.16 &  9.31 & 1.862 & 6.01 & 5.50 & 1.1\\
        \midrule
        Mittelwerte:\\
        14.563 & 9.616 & 1.9232 & 6.099 & 6.251 & 1.2502\\
        
        \bottomrule
        
    \end{tabular}
\end{minipage}
\end{document}