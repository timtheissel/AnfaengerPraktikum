\documentclass[titlepage=firstcover, captions=tableheading]{scrartcl}
\usepackage{microtype}
\usepackage{amsmath}
\usepackage{polyglossia}
\usepackage{graphicx}
\usepackage{booktabs}
\usepackage{siunitx}
\usepackage{hyperref}
\usepackage{caption}
\usepackage{float}
\setdefaultlanguage{german}
\title{V354 Gekoppelte und erzwungene Schwingungen}
\author{
Connor Magnus Böckmann \\ email: \href{mailto:connormagnus.boeckmann@tu-dortmund.de}{connormagnus.boeckmann@tu-dortmund.de}
\and Tim Theissel \\ email: \href{mailto:tim.theissel@tu-dortmund.de}{tim.theissel@tu-dortmund.de}}
\begin{document}
\maketitle
\newpage
\tableofcontents
\newpage
\section{Zielsetzung}
Das Ziel des Versuchs ist die Untersuchung eines RCL-Schwingkreises. Dabei soll der reale gedämpfte Schwingkreis anhand der Zeitabhängigkeit der Amplituden, des effektiven Widerstandes, dem aperiodischen Grenzfall und die Frequenzabhängigkeiten der Kondensatorspannung und ihrer Phase zur Erregerspannung. 
\section{Theoretische Grundlagen}
Beim in diesem Versuch untersuchten Schwingkreis handelkt es sich um einen RCL-Schwingkreis. Dieser enthält, verglichen mit dem RC-Kreis, neben dem Kondensator und dem Widerstand noch eine Spule mit Induktivität L. Dadurch ergibt sich ein zweiter Energiespeicher in Form der Spule. Diese ermöglicht ein Pendeln der Energie zwischen den beiden Energiespeichern, Spule und Kondensator. Das RCL-System schwingt also periodisch hin und her. Ohne Energieverbraucher, zum Beispiel den Widerstand, würde das System unendlich lange ungedämpft schwingen. Da der Widerstand aber elektrische Energie in Wärme umwandelt, geht darüber Energie verloren und der Schwingkreis wird gedämpft. Besonders von Interesse ist dabei das Zeitgesetz.
\subsection{Differentialgleichung für eine gedämpfte Schwingung-Herleitung und Lösung}
Vereinfacht lässt sich der RCl-Schwingkreis darstellen wie in Abb. 1. Daraus lässt sich erkennen, dass die zweite Kirchhoffsche Regel hier verwendet werden kann.
\begin{figure}[H]
    \centering
    \includegraphics{"Schaltkreis_RCL.png"}
    \caption{RCL-Schwingkreis}
\end{figure}
Mit der zweiten Kirchhoffschen Regel lässt sich die Spannung im System darstellen:
\begin{align}
    U_R(t)+U_C(t)+U_L(t)&=0\nonumber\\
    U_R(t)&=R\cdot I(t)\nonumber\\
    U_C(t)&=\frac{Q(t)}{C} \text{ (Q(t)=Ladung des Kondensators)}\nonumber\\
    U_L(t)&=L\frac{dI}{dt} \text{ (Induktionsgesetz)}\nonumber\\\nonumber
\end{align}
Es folgt also daraus:
\begin{equation}
  L\frac{dI}{dt}+RI+Q\frac{Q}{C}=0  \nonumber
\end{equation}
Die gesuchte Schwingungsgleichung ergibt sich dann mit 
\begin{equation}
    I=\frac{dQ}{dt}\nonumber
\end{equation}
aus der ersten Ableitung zu 
\begin{equation} \label{DGL}
    \frac{d^2I}{dt^2}+\frac{RdI}{Ldt}+\frac{I}{LC}=0
\end{equation}
Geloest wird diese lineare, homogene DGL mit dem Ansatz 
\begin{equation}
    \tau(t)=\xi e^{j\omega t}\nonumber
\end{equation}
Dabei sind \xi und \omega komplexe Zahlen und $j=\sqrt{-1}$.
Daraus erhaelt man 
\begin{equation}
    (-\omega^2+j\frac{R}{L}\omega+\frac{1}{LC})\xi e^{j\omega t} \nonumber
\end{equation}
Die DGL wird nun fuer beliebige \xi und beliebige t erfuellt, solange \omega die charakteristische Gleichung 
\begin{equation}
    \omega^2-j\frac{R\omega}{L}-\frac{1}{LC}=0 \nonumber
\end{equation}
Daher muss \omega einen der beiden Werte 
\begin{equation}
    \omega=j\frac{R}{2L}\pm \sqrt{\frac{1}{LC}-\frac{R^2}{4L^2}}\nonumber
\end{equation}
Alle Loesungen der DGL lassen sich dann ausdruecken als 
\begin{equation}
    \tau=\xi_1e^{j\omega_1t}+\xi_2e^{j\omega_2t}\nonumber
\end{equation}
Ausserdem werden die Abkuerzungen 
\begin{align}
    2\pi\mu:&=\frac{R}{2L}\nonumber\\
    2\pi v:&=\sqrt{\frac{1}{LC}-\frac{R^2}{4L^2}}\nonumber\\
\end{align}
$\tau(t)$ laest sich dann schreiben als
\begin{equation}\label{Klammer}
    \tau(t)=e^{-2\pi\mu t}(\xi_1e^{j2\pi vt}+\xi_2e^{-j2\pi vt})
\end{equation}
Die Loesung haengt nun massgebend davon ab, ob $\frac{1}{LC}$ groesser oder kleiner als $\frac{R^2}{4L^2}$. Dieser Umstand entscheide nun, ob v reell oder imaginaer ist. Daher ist eine Fallunterscheidung notwendig.
\subsection{1. Fall: $\frac{1}{LC}>\frac{R^2}{4L^2}$}
Unter diesen Bedingungen ist dann $\xi_1 =\bar{\xi_2}$, damit die Loesung $\tau(t)$ reell wird.
Das laesst sich ausdruecken durch den Ansatz
\begin{align}
    \xi_1&=\frac{1}{2}A_0e^{j\eta}\nonumber\\
    \xi_2&=\frac{1}{2}A_0e^{-j\eta}\nonumber
\end{align}
Fuer $\tau(t)$ unter Benutzung der Euler-Identitaet gilt
\begin{equation}
    \frac{e^{j\varphi}+e^{-j\varphi}}{2}=\cos\varphi \nonumber
\end{equation}
Daher ergibt sich fuer den geklammerten Ausdruck aus \ref{Klammer} eine reine oszillatorische Funktion.
\begin{equation}
    I(t)=A_0e^{-2\pi\mu t}\cos(2\pi vt+\eta)\nonumber
\end{equation}
Diese Gleichung stellt eine gedaempfte Schwingung dar. Die Schwingung ist also harmonisch, geht aber mit fortschreitender Zeit exponentiell gegen null. Die Schwingungsdauer laesst sich dabei ausdruecken als:
\begin{equation}
    T=\frac{1}{v}=\frac{2\pi}{\sqrt{\frac{1}{LC}-\frac{R^2}{4L^2}}}\nonumber
\end{equation}
Sie naehert sich dem Wert
\begin{equation}
    T_0=\frac{2\pi}{\omega_0}=2\pi\sqrt{LC}\nonumber
\end{equation}
der ungedaempften Schwingung an, wenn $\frac{R^2}{4L^2}$ klein gegen $\frac{1}{LC}$.
Die Amplitudenabnahmmegeschwindigkeit wird charakterisiert durch $2\pi\mu=\frac{R}{2L}$.
Die Amplitude geht dabei nach der Zeit 
\begin{equation}
    T_{ex}=\frac{1}{2\pi\mu}=2\pi\sqrt{LC}\nonumber
\end{equation}
auf den e-ten Teil des Startwerts zurueck. $T_ex$ wird Abklingdauer genannt.
\subsection{2. Fall: Aperiodische Daempfung fuer $\frac{1}{LC}<\frac{R^2}{4L^2}$}
Die Loesung I(t) enthaelt fuer diesen Fall keinen oszillatorischen Anteil mehr. Dieser Fall nennt sich aperiodische Daempfung. Von besonderer Bedeutung ist der Spezialfall des aperiodischen Grenzfalls.
\begin{equation}
    \frac{1}{LC}=\frac{R_ap^2}{4L^2}\nonumber
\end{equation}
Dadurch ist $v=0$ und $I(t)=Ae^{-\frac{Rt}{2L}}=Ae^{-\frac{t}{\sqrt{LC}}}$
Der aperiodische Grenzfall stellt einen Fall ohne Ueberschwingen dar fuer den die Schwingung am schnellsten gegen null geht.
\subsection{Differentialgleichung fuer erzwungene Schwingungen}
Im Folgenden wird das schwingfaehige System durch eine aeussere, periodische Kraft ausgelenkt. Im Fall des RCL-Schwingkreises handelt es sich dabei um eine sinusfoermige Wechselspannung u(t).
\begin{equation}
    u(t)=U_0e^{j\omega t}\nonumber
\end{equation}
Die in den vorigen Abschnitten beschriebene Differentialgleichung \ref{DGL} nimmt hier somit die Form
\begin{equation}
    L\frac{d\xi}{dt}+R\xi+\frac{Q}{C}=U_0e^{j\omega t}\nonumber
\end{equation}
oder 
\begin{equation}\label{DGL 2}
    LC\frac{d^2u_C}{dt^2}+RC\frac{du_C}{dt}+u_C=U_0e^{j\omega t}
\end{equation}
an. 
Q(t) stellt hierbei die Ladung auf dem Kondensator dar, weshalb damit die Spannung folgende ist:
\begin{equation}
    u_C(t)=\frac{Q(t)}{C}\nonumber
\end{equation}
Betrachtet wird im Folgenden vorallem die Amplitude der Kondensatorspannung $u_C(t)$ mit ihrem Phasenunterschied gegenueber der Erregerspannung u(t). Besonderes Augenmerk wird dabei auf ihre Frequenzabhaengigkeit gelegt.\\
Der Ansatz ist
\begin{equation}\label{Ansatz}
    u_C(\omega, t)=\xi(\omega)e^{j\omega t} (\xi komplex) 
\end{equation}
Nun wird \ref{Ansatz} in \ref{DGL 2} eingesetzt und die Bestimmungsgleichung $-LC\omega^2\xi+j\omega RC\xi+\xi=U_0$ nach \xi  aufgeloest.
\begin{equation}
    \xi=\frac{U_0}{1-LC\omega^2+j\omega RC}=\frac{U_0(1-LC\omega^2-j\omega RC)}{(1-LC\omega^2)^2+\omega^2R^2C^2} \nonumber
\end{equation}
Der Betrag betraegt dann 
\begin{equation}\label{Betrag}
    |\xi|=\sqrt{Re^2(\xi)+Im^2(\xi)}=U_0\sqrt{\frac{(1-LC\omega^2)^2+\omega^2R^2C^2}{((1-LC\omega^2)^2)+\omega^2R^2C^2)^2}}
\end{equation}
mit der Phase
\begin{equation}
    \tan \varphi(\omega)=\frac{Im(\xi)}{Re(\xi)}=\frac{-\omega RC}{1-LC\omega^2}\nonumber
\end{equation}
beziehungsweise
\begin{equation}\label{phi}
    \varphi(\omega)=\arctan(\frac{-\omega RC}{1-LC\omega^2})
\end{equation}
Der Betrag der Loesungsfunktion nach \ref{Ansatz} entspricht dabei dem Betrag von \xi. Somit erhaelt man aus \ref{Betrag} als Ergebnis
\begin{equation} \label{Resonanzkurve}
    U_C(\omega)=\frac{U_0}{\sqrt{(1-LC\omega^2)^2+\omega^2R^2C^2}}
\end{equation}
Durch diese Beziehung laesst sich also nun die gesuchte Frequenzabhaengigkeit zwischen der Kondensatorspannung und der Frequenz $\omega$. Dieser Zusammenhang nennt sich Resonanzkurve. Fuer $\omega\rightarrow\infty$ geht $U_C\rightarrow 0$ und fuer $\omega\rightarrow 0$ geht $U_C\rightarrow U_0$. Bei einer bestimmten Frequenz gibt es ein $U_C$, welches groesser als die urspruengliche Erregeramplitude $U_0$ ist. Diese Frequenz nennt sich Resonanzfrequenz und das zugehoerige Phaenomen nennt sich Resonanz. Gegeben wird die Resonanzfrequenz durch
\begin{equation}
    \omega_{res}=\sqrt{\frac{1}{LC}-\frac{R^2}{2L^2}} \nonumber
\end{equation}
Besonderer Betrachtung wird der Fall der schwachen Daempfung, also $\frac{R^2}{2L^2}<<\frac{1}{LC}$, unterzogen. Dort naehert sich die Resonanzfrequenz naemlich der Kreisfrequenz des ungedaempften Schwingkreises an. Die Kondensatorspannung uebertrift die Erregerspannung um den Faktor $\frac{1}{\omega_0RC}$, die so genannte Resonanzueberhoehung oder Guete q eines Schwingkreises.
\begin{equation}\label{Umax}
    U_{c,max}=\frac{1}{\omega_0RC}U_0=\frac{1}{R}\sqrt{\frac{L}{C}}U_0 
\end{equation}
Es laesst sich bereits erkennen, dass die Kondensatorspannung fuer $R\rightarrow 0$ gegen $\infty$ geht. Es handelt sich dabei um die so genannte Resonanzkatastrophe. Ausserdem ist die Breite der Resonanzkurve in \ref{Resonanzkurve} ein Mass fuer die Schaerfe eines Schwingkreises. Charakterisiert wird sie durch die Frequenzen bei denen $U_C$ auf den $\frac{1}{\sqrt{2}}$-sten Bruchteil seines Maximums \ref{Umax} abgefallen ist. Diese werden $\omega_+$ und $\omega_-$ genannt und werden durch 
\begin{equation}
    U_{c,max}=\frac{1}{\omega_0RC}U_0=\frac{1}{R}\sqrt{\frac{L}{C}}U_0 \nonumber
\end{equation}
gegeben. Unter Beachtung, dass $\frac{R^2}{L^2}<<\omega_0^2$ ist, folgt fuer die Breite der Resonanzkurve
\begin{equation}
    \omega_+-\omega_-\approx \frac{R}{L}\nonumber
\end{equation}
Somit besteht zwischen der Guete q und der Breite der Resonanzkurve die folgende Beziehung:
\begin{equation}
    q=\frac{\omega_0}{\omega_+-\omega_-}\nonumber
\end{equation}
Das Verhalten aendert sich fuer starke Daempfung, also $\frac{R^2}{2L^2}>>\frac{1}{LC}$, grundlegend. Es existiert keine Frequenzueberhoehung mehr. Stattdessen geht $U_C$ von $U_0$ aus mit wachsender Frequenz monoton gegen 0. Bei hinreichend hohen Frequenzen faellt $U_C$ proportional zu $\frac{1}{\omega^2}$.\\
Im Folgenden soll ausserdem noch die Frequenzabhaengigkeit der Phase zwischen Erreger- und Kondensatorspannung untersucht werden. Nach dem Ansatz aus \ref{Ansatz} gibt Gleichung \ref{phi} den Zusammenhang zwischen $\varphi$ und $\omega$ wieder. Gleichung \ref{phi} sagt dabei aus, dass bei kleinen Frequenzen Kondensator- und Erregerspannung praktisch in Phase sind. Im Gegensatz dazu verschiebt sich die Kondensatorspannung bei sehr grossen Frequenzen um $\pi$ und hinkt dadurch der Erregerspannung hinterher. 
\subsection{Impedanz eines Schwingkreises}
Ein RCL-Serienschwingkreis kann auch als ein Zweipol aufgefasst werden. 
\begin{figure}[H]
    \centering
    \includegraphics{"Schaltkreis_RCL.png"}
    \caption{Serienschwingkreis als Zweipol}
\end{figure}
An den Enden des Zweipols laesst sich ein Widerstand feststellen. Dieser ist frequenzabhaengig und wird Impedanz. Sie wird auf Grund der haeufig vorhandenen Phasenverschiebung zwischen Strom und Spannung als komplexe Zahl definiert.
\begin{equation}
    z=x+iy \nonumber
\end{equation}
Dabei sind x und y reelle Widerstaende. Sie werden Blindwiderstand (Reaktanz) und Wirkwiderstand genannt. Der Betrag der Impedanz 
\begin{equation}
    |z|=\sqrt{x^2+y^2} \nonumber
\end{equation}
nennt sich Scheinwiderstand. Die Darstellung der Impedanz $z(\omega)$ in der komplexen Zahlenebene wird Ortskurve genannt und laesst sich duch einen Vektor aus dem Ursprung darstellen. Die Laenge des Vektors entspricht dem Scheinwiderstand. Der Winkel $\alpha$ zwischen dem Vektor und der reellen Achse ist genauso gross wie die Phasenverschiebung zwischen Spannung und Strom im Serienschwingkreis als Zweipol.
\begin{figure}[H]
    \centering
    \includegraphics[width=0.45\textwidth]{"RCL_Serienschwingkreis_komplex.png"}
    \caption{Darstellung der Impedanz des Serienschwingkreises als Zweipol in der komplexen Zahlenebene}
\end{figure}
Die Impedanz z errechnet sich nun folgendermassen mit den Widerstandsoperatoren
\begin{align}
    z_C&=-i\frac{1}{C\omega} \text{ (Widerstand der Kapazitaet)} \nonumber \\
    z_L&=iL\omega \text{ (Widerstand der Induktivitaet)}\nonumber \\
    z_R&=R_S \text{ (ohmscher Widerstand)} \nonumber\\
\end{align}
zu 
\begin{equation}
    z_S=R_S+i(L\omega-\frac{1}{C\omega}) \nonumber
\end{equation}
$x_S$ und $y_S$ sind dann
\begin{align}
    x_S&=R_S \nonumber \\
    y_S&=iL\omega-\frac{1}{C\omega} \nonumber \\
    |z_S|&=\sqrt{R_S^2+(L\omega-\frac{1}{C\omega})^2} \label{Scheinwiderstand}
\end{align}
Die Frequenzunabhaengigkeit von $x_S$ sorgt fuer eine Ortskurve, welche immer parallel zur imaginaeren Achse verlaeuft und an der Stelle $R_S$ die reelle Achse schneidet. Der Scheinwiderstand erreicht auf Grund von \ref{Scheinwiderstand} an der Stelle $\omega=\omega_0=\frac{1}{\sqrt{LC}}$ sein Minimum $R_S$. Fuer $\omega\rightarrow 0$ und $\omega\rightarrow\infty$ geht der Widerstand gegen $\infty$. 
\subsection{Parallelschwingkreis als Zweipol}
\begin{figure}[H]
    \centering
    \includegraphics[width=0.45\textwidth]{"RCL_Parallelschwingkreis.png"}
    \caption{Parallelschwingkreis als Zweipol}
\end{figure}
\noindent Die Impedanz eines Parallelschwingkreises soll der Vollstaendigkeit halber auch noch kurz erwaeht werden. Diese errechnet sich dann zu
\begin{equation}\label{Impedanz_parallel}
    z_p=\frac{\frac{1}{R_p}+i(\frac{1}{\omega L}-C\omega)}{\frac{1}{R_p^2}+(\frac{1}{L\omega}-C\omega)^2} 
\end{equation}
und der Scheinwiderstand dementsprechend zu
\begin{equation}
    |z_p|=\frac{1}{\sqrt{\frac{1}{R_p^2}+(\frac{1}{L\omega}-C\omega)^2}}
\end{equation}
$|z_p|$ durchlauft also nun bei $\omega_0=\frac{1}{\sqrt{LC}}$ ein Maximum mit dem Wert $R_p$. $|z_p|$ geht fuer $\omega\rightarrow 0$ gegen 0. Die Abhaengigkeit des Scheinwiderstandes von der Frequenz laesst sich wie folgt darstellen:
\begin{figure}[H]
    \centering
    \includegraphics[width=0.45\textwidth]{"RCL_Parallelschwingkreis_Abhaengigkeit.png"}
    \caption{Abhaengigkeit des Scheinwiderstandes von der Frequenz im Parallelschwingkreis}
\end{figure}
Aus \ref{Impedanz_parallel} ergibt sich fuer die Ortskurve des parallel geschalteten RCL-Schwingkreises ein Kreis mit dem Radius $\frac{1}{2}R_p$. Der Mittelpunkt ist dabei der Punkt $(0,\frac{1}{2}R_p)$, siehe ~\ref{fig:Parallel_Ortskurve}.
\begin{figure}[H]
    \centering
    \includegraphics[width=0.45\textwidth]{"RCL_Parallelschwingkreis_Ortskurve.png"} 
    \caption{Ortskurve eines Parallelschwingkreis}
    \label{fig:Parallel_Ortskurve}
\end{figure}
\end{document}

