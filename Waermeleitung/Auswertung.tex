\section{Auswertung}

\subsection{Statistische Methode}

\subsubsection{Temperaturverläufe}

Zuerst wurden in einer Messung die Temperaturen der, der Heizquelle, ferenen Thermoelemente gemessen. Dabei liegt T1 am breiteren der beiden Messingprobenstäbe und T4 am schmaleren. T5 liegt am Aluminium Stab und T8 am Edelstahlstab. Die Temperaturverläufe der einzelnen Messpunkte sehen dabei wie folgt aus:

\begin{figure}[H]
    \centering
    \includegraphics{1.png}
\end{figure}

\begin{figure}[H]
    \centering
    \includegraphics{2.png}
\end{figure}

\subsubsection{Temperatur nach t=700s}

700s entsprechen bei einer Abtastrate von 5 Werten pro Sekunde den Werten in der 3500. Zeile der Tabelle ?. Diese sind allerdings auch noch einemla hier aufgelistet:
\begin{align*}
    T1 = 53.3 ^\circ C  \\
    T4 = 50.59 ^\circ C \\
    T5 = 55.95 ^\circ C \\
    T8 = 42.03 ^\circ C 
\end{align*}

\subsubsection{Wärmestrom $\frac{\Delta Q}{\Delta t}$}

Anschließend soll für 5 verschiedene Meßzeiten der Wärmestrom $\frac{\Delta Q}{\Delta t}$ bestimmt werden. Dazu wird folgende Gleichung verwendet. 
\begin{displaymath}
    \frac{\Delta Q}{\Delta t} = - \kappa A \frac{\partial T}{\partial x}
\end{displaymath}

Dabei ergeben sich für den Wärmestrom folgende Werte: 
\begin{minipage}{\linewidth}
    \begin{table}[H]
        \centering
    \captionof{table}{Berechnete Wärmeströme für t=140s}
    \begin{tabular}{lll}
        \toprule
        Meßzeit [s] & $\frac{\partial T}{\partial x}$ & Wärmestrom  \\
        \midrule
        Messing (breit) & 5.03 & -289.728 \\
        Messing (schmal) & 6.31 & -212.016 \\
        Aluminium & 3.03 & -290.880 \\
        Edelstahl & 13.63 & -256.512 \\
        \bottomrule   
    \end{tabular}
    
    \label{tab:1}
\end{table}
\end{minipage}

\begin{minipage}{\linewidth}
    \begin{table}[H]
        \centering
    \captionof{table}{Berechnete Wärmeströme für t=280s}
    \begin{tabular}{lll}
        \toprule
        Meßzeit [s] & $\frac{\partial T}{\partial x}$ & Wärmestrom  \\
        \midrule
        Messing (breit) & 3.11 & -179.136 \\
        Messing (schmal) & 4.51 & -151.536 \\
        Aluminium & 1.94 & -186.240 \\
        Edelstahl & 11.93 & -229.056 \\
        \bottomrule   
    \end{tabular}
    
    \label{tab:2}
\end{table}
\end{minipage}

\begin{minipage}{\linewidth}
    \begin{table}[H]
        \centering
    \captionof{table}{Berechnete Wärmeströme für t=420s}
    \begin{tabular}{lll}
        \toprule
        Meßzeit [s] & $\frac{\partial T}{\partial x}$ & Wärmestrom  \\
        \midrule
        Messing (breit) & 2.56 & -147.456 \\
        Messing (schmal) & 4.05 & -136,080 \\
        Aluminium & 1.71 & -164.160 \\
        Edelstahl & 10.86 & -208.512 \\
        \bottomrule   
    \end{tabular}
    
    \label{tab:3}
\end{table}
\end{minipage}

\begin{minipage}{\linewidth}
    \begin{table}[H]
        \centering
    \captionof{table}{Berechnete Wärmeströme für t=420s}
    \begin{tabular}{lll}
        \toprule
        Meßzeit [s] & $\frac{\partial T}{\partial x}$ & Wärmestrom  \\
        \midrule
        Messing (breit) & 2.40 & -138.240 \\
        Messing (schmal) & 3.93 & -132.048 \\
        Aluminium & 1.64 & -157.44 \\
        Edelstahl & 10.36 & -198.912 \\
        \bottomrule   
    \end{tabular}
    
    \label{tab:4}
\end{table}
\end{minipage}

\begin{minipage}{\linewidth}
    \begin{table}[H]
        \centering
    \captionof{table}{Berechnete Wärmeströme für t=420s}
    \begin{tabular}{lll}
        \toprule
        Meßzeit [s] & $\frac{\partial T}{\partial x}$ & Wärmestrom  \\
        \midrule
        Messing (breit) & 2.33 & -134.208 \\
        Messing (schmal) & 3.87 & -130.032 \\
        Aluminium & 1.62 & -155.52 \\
        Edelstahl & 10.00 & -192.000 \\
        \bottomrule   
    \end{tabular}
    
    \label{tab:5}
\end{table}
\end{minipage}

\subsubsection{Temperaturdifferenz}

Zum Ende dieser Methode sind noch einmal die Temperaturdifferenzen T$_2$-T$_1$ und T$_7$-T$_8$ graphisch dargestellt.

\begin{figure}[H]
    \centering
    \includegraphics{3.png}
\end{figure}


\subsection{dynamische Methode}

\subsubsection{Bestimmung der Wärmeleitfähigkeit von:}

\subsubsection{Aluminium}

Wenn die Ergebnisse aus der dynamischen Messung für Aluminium graphisch darstellt entsteht folgendes Diagramm:

\begin{figure}[H]
    \centering
    \includegraphics{alu.png}
\end{figure}

Daraus können die Amplituden und die Phasendifferenz der Wellen abgelesen werden. Die abgelesenen Werte sehen wie folgt aus:

\begin{minipage}{\linewidth}
    \begin{table}[H]
        \centering
    \captionof{table}{Amplituden und Phasendifferenz für Aluminium.}
    \begin{tabular}{llll}
        \toprule
        Phasendifferenz & A$_6$ & A$_5$ & $\kappa$  \\
        \midrule
        10 & 34 & 31 & 262.80 \\
        10 & 40 & 35 & 181.80 \\
        10 & 45 & 41 & 260.78 \\
        10 & 51 & 44 & 164.43 \\
        10 & 54 & 49 & 249.84 \\
        10 & 58 & 52 & 222.31 \\
        10 & 61 & 56 & 283.85 \\
        10 & 64 & 58 & 246.60 \\
        10 & 66 & 60 & 254.70 \\
        10 & 68 & 62 & 262.80 \\
        \midrule
        Mittelwerte&für&$\kappa$&238.99$\pm$ 36.27\\
        \bottomrule   
    \end{tabular}
\end{table}
\end{minipage}


\subsubsection{Messing}

Für Messing sieht die Messung folgendermaßen aus:

\begin{figure}[H]
    \centering
    \includegraphics{brass.png}
\end{figure}

\begin{minipage}{\linewidth}
    \begin{table}[H]
        \centering
    \captionof{table}{Amplituden und Phasendifferenz für Messing.}
    \begin{tabular}{llll}
        \toprule
        Phasendifferenz & A$_6$ & A$_5$ & $\kappa$  \\
        \midrule
        10 & 32 & 27 & 142.88 \\
        10 & 35 & 31 & 200.03 \\
        10 & 41 & 35 & 153.43 \\
        10 & 45 & 39 & 169.64 \\
        10 & 50 & 43 & 160.96 \\
        10 & 53 & 45 & 148.36 \\
        10 & 55 & 48 & 178.32 \\
        10 & 58 & 50 & 163.56 \\
        10 & 60 & 53 & 195.70 \\
        10 & 62 & 55 & 202.63 \\
        \midrule
        Mittelwerte&für&$\kappa$&171.51$\pm$ 20.67\\
        \bottomrule   
    \end{tabular}
\end{table}
\end{minipage}



\subsubsection{Edelstahl}

Bei der dynamischen Messung für Edelstahl sehen die Messwerte folgendermaßen aus:

\begin{figure}[H]
    \centering
    \includegraphics{steel.png}
\end{figure}

\begin{minipage}{\linewidth}
    \begin{table}[H]
        \centering
    \captionof{table}{Amplituden und Phasendifferenz für Aluminium.}
    \begin{tabular}{llll}
        \toprule
        Phasendifferenz & A$_7$ & A$_8$ & $\kappa$  \\
        \midrule
        30 & 50 & 33 &  \\
        30 & 60 & 38 &  \\
        30 & 65 & 43 &  \\
        30 & 70 & 48 &  \\
        30 & 73 & 51 &  \\
        \midrule
        Mittelwerte&für&$\kappa$&\\
        \bottomrule   
    \end{tabular}
\end{table}
\end{minipage}

\section{Diskussion}

\subsection{Temperaturverläufe}

Bei den vier Temperaturkurven von T1, T4, T5 und T8 ist auf den ersten Blick zu erkennen, dass lediglich die Temperaturverlaufskurve von Edelstahl nahezu gänzlich anders aussieht als alle anderen der sonst steile Anstieg ist hier eher flach. Alle anderen Kurven flachen erst sehr viel später ab. Sie erreichen auch alle unterschiedlich hohe Werte nach 700s. Da Zu erkennen ist, dass T5 mit $55.95 ^\circ C$ den höchsten Wert erreicht, ist zu sagen, dass Aluminium die beste Wärmeleitung hat. 

In der Grafik für die Temperaturunterschiede bei T2-T1 und T7-T8 ist zu erkennen dass zunächst die Temperaturdifferenz schnell größer wird. Das liegt daran, dass der Wärmetransport innerhalb des Stabes einige Zeit dauert. dies sieht man auch im weiteren Verlauf des Graphen. Als die Temperatur nämlich beginnt immer langsamer zu wachsen, wird die Temperaturdifferenz auch wieder langsam kleiner denn der Wärmetransport schafft es nun besser die Temperatur bei T1/T8 anzugleichen wenn T2/T7 langsamer wachsen. Da im allgemeinen die Temperaturdifferenz T2-T1 deutlich kleiner ist als die Temperaturdifferenz T7-T8.

\subsection{Wärmekapazität}

Für die Wärmekapazität wurden folgende Werte herausgesucht:

\begin{align*}
    \kappa (Messing) = 120 \frac{W}{mK} \\
    \kappa (Aluminium) = 200 \frac{W}{mK}
\end{align*}

Die mit der Angström Methode bestimmten Werte weichen davon stark ab. Der berechnete Wert für die Wärmekapazität von Aluminium weicht um 19\% ab. Der Wert für Messing um 43\%. Das sind sehr große Fehler allerdings wurden auch alle Werte für diese Rechnungen aus den Grafiken abgelesen. Daher entstehen allein beim Ablesen enorme Abweichungen. Auch die in den Grafiken dargestellten Werte sind bereits Messwerte, das heißt auch diese unterliegen eventuell schon Ungenauigkeiten durch Fehler bei der Durchführung des Experiments.