\section{Zielsetzung}
Im folgenden Experiment soll die Waermeleitung von Aluminium, Messing und Edelstahl betrachtet und genauer untersucht werden. Besonders Materialkonstanten wie die Waerme- und Temperaturleitfaehigkeit sollen betrachtet werden.
\section{Theoretische Grundlagen}
Bei einem nicht vorhandenen Waermegleichgewicht, kommt es zu Waermetransport. Dieser folgt dem Waermegradienten und kann auf drei Arten geschen. Diese Arten nennen sich Konvektion, Waermestrahlung und Waermeleitung. Bei der hier naeher betrachteten Waermeleitung geschieht der Waermetransport vornehmlich ueber Phononen und frei bewegliche Elektronen.
Bei einem Stab der Laenge L und der Querschnittsflaeche A aus einem Material der Dichte $\rho$ und der spezifischen Waerme c, fliesst die Waermemenge dQ durch die Querschnittsflaeche A in der Zeit dt:
\begin{equation}
    dQ=-\kappa A\frac{\partial T}{\partial x}dt
\end{equation}
Die Waermeleitfaehigkeit $\kappa$ ist dabei eine materialabhaengige Groesse. Der Umstand, dass Waerme immer in Richtung abnehmender Temperaturen fliesst, ist an dem Minuszeichen zu erkennnen und entspricht der Konvention.
Die Waermestromdichte ist gegeben durch 
\begin{equation}
    j_w=-\kappa \frac{\partial T}{\partial x}
\end{equation}
Die eindimensionale Waermeleitungsgleichung laesst sich dann daraus ableiten:
\begin{equation}
    \frac{\partial T}{\partial t}=\frac{\kappa}{\rho c}\frac{\partial^2T}{\partial x^2}
\end{equation}
Der Vorfaktor $\sigma_T=\frac{\kappa}{\rho c}$ nennt sich Temperaturleitfaehigkeit und gibt die 'Geschwindigkeit' an mit der Temperaturunterschiede ausgeglichen werden. Die Loesung dieser Gleichung ist abhaengig von der Stabgeometrie und den Anfangsbedingungengen.\\
Bei Verwendung eines sehr langen Stabes und einer abwechselnden Erwaermung und Abkuehlung mit einer Periodendauer T, wird die Waerme in Form einer sich zeitlich und raeumlich fortpflanzenden Temperaturwelle verbreitet. Diese hat dann die Form:
\begin{equation}
    T(x,t)=T_{max}e^{-\sqrt{\frac{\omega \rho c}{2\kappa}}x}cos(\omega t-\sqrt{\frac{\omega \rho c}{2\kappa}}x)
\end{equation}
Die Phasengeschwindigkeit dieser Welle betraegt also:
\begin{equation}
    v=\frac{\omega}{k}=\frac{\omega}{\sqrt{\frac{\omega \rho c}{2\kappa}}}=\sqrt{\frac{2\kappa\omega}{\rho c}}
\end{equation}
Das Amplitudenverhaeltnis von $A_{nah}$ und $A_{fern}$ an zwei Messtellen $x_{nah}$ und $x_{fern}$ gibt die Daempfung. Unter Ausnutzung von $\omega=\frac{2\pi}{T^*}$ und $\phi=\frac{2\pi\Delta t}{T^*}$ wird die Waermeleitfaehigkeit $\kappa$ erhalten:
\begin{equation}
    \kappa=\frac{\rho c (\Delta x)^2}{2\Delta t ln(\frac{A_{nah}}{A_{fern}})}
\end{equation}
$\Delta t$ entspricht der Phasendifferenz der Welle und $\Delta x$ entspricht dem Abstand der beiden Messpunkte.