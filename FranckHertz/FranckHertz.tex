\documentclass[titlepage=firstcover, captions=tableheading]{scrartcl}
\usepackage{microtype}
\usepackage{amsmath}
\usepackage{polyglossia}
\usepackage{graphicx}
\usepackage{booktabs}
\usepackage{siunitx}
\usepackage{hyperref}
\usepackage{caption}
\usepackage{float}
\usepackage{caption}
\setdefaultlanguage{german}
\title{V601 Franck-Hertz-Versuch}
\author{
Connor Magnus Böckmann \\ email: \href{mailto:connormagnus.boeckmann@tu-dortmund.de}{connormagnus.boeckmann@tu-dortmund.de}
\and Tim Theissel \\ email: \href{mailto:tim.theissel@tu-dortmund.de}{tim.theissel@tu-dortmund.de}}
\begin{document}
\maketitle
\newpage
\tableofcontents
\newpage
\section{Zielsetzung}
Das Ziel des Franck-Hertz-Versuch ist die Ermittlung, der von einem Quecksilber-Atom aufgenommenen Energie, um es in einen angeregten Zustand zu versetzen. Dies erfolgt mit Hilfe von Elektronenstößen mit Quecksilberatomen in Dampf. Gemessen wird dabei die Energie der Elektronen vor dem Stoß über die Beschleunigungsspannung und die Energie nach dem Stoß unter Zuhilfenahme einer Gegenfeldmethode. 
\section{Theoretische Grundlagen}
\subsection{Aufbau und Funktionsweise des Franck-Hertz-Versuchs}
Der Franck-Hertz-Versuch besteht im wesentlichen aus einem Behaeltnis, welches evakuiert wurde. In diesem befindet sich ein sehr kleiner Quecksilbertropfen, welcher verdampft, sodass sich ein Gleichgewichtsdampfdruck $p_{saet}$ einstellt. Dieser ist von der Umgebungstemperatur $T$ abhaengig und laesst sich somit steuern. Ausserdem enthaelt das Gefaess einen Wolfram-Draht, welcher durch einen Gleichstrom erhitzt wird bis er glueht. Der gluehelektrische Effekt sorgt dafuer, dass eine Elektronenwolke den Draht umgeben, welche mit einer gegenueber liegenden, netzfoermigen Elektrode und der daran anliegenden positiven Beschleunigungsspannung $U_B$ abgegriffen und beschleunigt werden kann. Nach Durchlaufen der Beschleunigungsstrecke hat das Elektron eine kinetische Energie von
\begin{equation}
    \frac{m_0\cdot v^2_{vor}}{2}=e_0U_B \nonumber
\end{equation}
, wenn es vorher die Geschwindigkeit $v_{vor}=0$ hatte. Eine Auffaengerelektrode befindet sich hinter dem Beschleunigungselektrodennetz, an welcher der Auffaengerstrom $I_A$ mit einem Messgeraet gemessen werden kann. Zwischen Beschleunigerelektrode und Auffaengerelektrode liegt eine kleine Bremsspannung $U_A$ an, welche dafuer sorgt, dass nur Elektronen die Auffaengerelektrode erreichen, welche die Ungleichung 
\begin{equation}
    \frac{m_0}{2}v^2_z\geq e_0U_A
\end{equation}
erfuellt. Elektronen, welche diese Ungleichung nicht erfuellen kehren zur Beschleunigerelektrode zurueck.
Im Beschleunigerraum befinden sich aber nun Quecksilberatome, mit welchen die Elektronen zusammenstossen koennen. Bei geringer Energie sind diese Stoesse elastisch, wobei auf Grund der sehr unterschiedlich grossen Massen der Elektronen und des Quecksilberatoms eine vernachlaessigbar geringe Energieuebertragung stattfindet.
\newpage \noindent Der zweite Fall ist, dass die Elektronenenergie E gross genug ist, um das Hg-Atom anzuregen. Dies wird erreicht durch das Erhoehen der Beschleunigungsspannung bis die Elektronenenergie groesser oder gleich der Energiedifferenz zwischen dem angeregten Zustand $E_1$ und dem Grundzustand $E_0$ ist. Bei einem solchen Stoss wird eben jene Energie $E_1-E_0$ auf das Atom uebertragen, waehrend das Elektron die Energie $E-(E_1-E_0)$ behaelt. Nach einer Relaxationszeit von etwa $10^{-10}s$ geht das Atom nun wieder in den Grundzustand zurueck und emmitiert dabei einen Lichtquant mit einer Energie von eben jener Energiedifferenz $E_1-E_0$:
   \begin{center}
    $h\nu=E_1-E_0 \nonumber$\\
    $(h=Planksches \space Wirkungsquantum, \nu=Frequenz \space des \space Lichts)\nonumber$
   \end{center}
Nun wird der Auffaengerstrom $I_A$ gegenueber der Beschleunigungsspannung $U_B$ aufgetragen. Wird nun $U_B$ von null an erhoeht, waechst der Elektronenstrom an sobald die Beschleunigungsspannung das fest eingestellte Gegenpotential $U_A$ uebersteigt. Bei weiterer Steigerung von $U_B$ ueberragt diese schliesslich irgendwann $E_1-E_0$ ein bisschen, wodurch die zuvor beschriebenen inelastischen Stoesse auftreten. Sie verlieren dabei nahezu ihre gesamte Energie, wodurch sie nicht mehr gegen das Bremsfeld ankommen. Daher faellt der Auffaengerstrom stark ab. Nun kann die Beschleunigungsspannung weiter gesteigert werden, wodurch von den Elektronen nach dem Stoss erneut Energie aufgenommen werden kann. Daher steigt der Auffaengerstrom dann wieder an, bis die Elektronen erneut die Energie $E_1-E_0$ aufweisen und sie somit einen weiteren inelastischen Stoss ausfuehren koennen, woraufhin erneut der Auffaengerstrom abfaellt. Dieser Ablauf laesst sich noch mehrmals wiederholen. Die ausgegebene, idealisierte Grafik ist in \ref{Fig:Ideal} zu sehen.
\begin{figure}[H]
    \centering
    \captionsetup{justification=centering}
    \includegraphics[height=5cm]{"Ideal_FranckHertz.png"}
    \captionbelow{Idealisierter Zusammenhang aus $I_A$ und $U_B$ \\ Aus: Anleitung V601 Seite 118}
    \label{Fig:Ideal}
\end{figure}
Nach den gemachten Vorueberlegungen muss der Abstand zweier aufeinanderfolgender Peaks dem Anregungspotential entsprechen.
\begin{equation}
    U_1:=\frac{1}{e_0}(E_1-E_0)
\end{equation}
\subsection{Einfluesse auf die Gestalt der Franck-Hertz-Kurve}
Statt der in \ref{Fig:Ideal} dargestellten idealisierten Kurve wird in der Realitaet eine etwas andere Kurve erhalten. Begruendet ist das durch einige Nebeneffekte, welche im Folgenden erlaeutert werden sollen.
\subsubsection{Das Kontaktpotential}
Wenn die Beschleunigerelektrode und der Gluehdraht aus unterschiedlichen Materialien besteht, ist das tatsaechliche Beschleunigerpotential von der Spannung $U_B$ verschieden, wenn die Austrittsarbeit beider Materialien verschieden ist. Fuer das Material des Gluehdrahtes wird fuer gewoehnlich ein Material mit geringer Austrittsarbeit gewaehlt, um bereits bei niedrigen Temperaturen eine hohe Emissionsrate zu erreichen. Daher ist die Austrittsarbeit des Gluehdrahtes $\Phi_G$ bedeutend kleiner, als die der Beschleunigerelektrode $\Phi_B$. Die Potentialverhaeltnisse sind in \ref{Fig:Potential} dargestellt.
\begin{figure}[H]
    \centering
    \captionsetup{justification=centering}
    \includegraphics[height=8cm]{"Potential_FranckHertz.png"}
    \captionbelow{Potentialverhaeltnisse zwischen Beschleunigerelektrode und Gluehdraht \\ Aus: Anleitung V601 Seite 119}
    \label{Fig:Potential}
\end{figure}
\noindent Das effektive Beschleunigerpotential $U_{B, eff}$ hat den Wert:
\begin{equation}
    U_{B,eff}=U_B-\frac{1}{e_0}(\Phi_B-\Phi_G)\nonumber
\end{equation}
Der Ausdruck 
\begin{equation}
    K:=\frac{1}{e_0}(\Phi_B-\Phi_G)\nonumber
\end{equation}
wird Kontaktpotential K genannt. Die Franck-Hertz-Kurve ist um eben jenes Kontaktpotential K verschoben.
\subsubsection{Das Energie-Spektrum der Elektronen}
Ausgang der bisherigen Betrachtung war immer, dass alle Elektronen nach der Beschleunigung die gleiche Energie haben. Diese Annahme ist aber nicht wahr. Die Leitungselektronen besitzen im Metall des Gluehdrahts bereits ein Spektrum an Energien (Fermi-Dirac-Verteilung), weshalb sie bei der Gluehemission bereits unterschiedliche Anfangsgeschwindigkeiten haben. Daher haben sie nach Durchlaufen der Beschleunigungsstrecke ebenfalls ein Spektrum an Energien. Die inelastischen Stoesse setzen also nicht mehr bei einer genau definierten Beschleunigungsspannung ein, sondern eher in einem gewissen Bereich. Aus diesem Grund steigt die Kurve im Vergleich zum Ideal in \ref{Fig:Ideal} nicht mehr so stark, wenn sie sich einem Maximum naehert und faellt danach nicht unstetig auf null ab, sondern naehert sich stetig einem Minimum an Strom $I_A$.\\
Erwaehnenswert sind hier aber auch die elastischen Stoesse zwischen Elektronen und Hg-Atomen. Wie bereits genannt wird hierbei kaum Energie uebertragen, jedoch sorgen diese Stoesse unter Umstaenden fuer grosse Richtungsaenderungen der Elektronen. Zwischen Gluehdraht und Beschleunigerelektrode sind diese nicht weiter von Bedeutung. Die Richtungsaenderungen koennen jedoch zwischen Beschleunigerelektrode und Auffaengerelektrode zu einer Verteilung der z-Komponente der Geschwindigkeiten fuehren. Da aber das Ueberwinden des Gegenfeldes und damit das Erreichen der Auffaengerelektrode von $v_z$ abhaengt, wird hierdurch die Kurve flacher und breiter werden.
\subsubsection{Der Dampfdruck}
Entscheidend fuer den Erfolg des Franck-Hertz-Versuchs sind die Zusammenstoesse von Hg-Atomen und Elektronen. Die mittlere freie Weglaenge $\bar{w}$ muss also klein im Vergleich zur Strecke a zwischen Gluehdraht und Beschleunigerelektrode sein. $\bar{w}$ kann ueber den Saettigungsdampfdruck $p_{saet}$ eingestellt werden.
\begin{equation}
    \bar{w}[cm]=\frac{0,0029}{p_{saet}} \text{[p in mbar]}\nonumber
\end{equation}
Der Saettigungsdampfdruck laesst sich aus der Temperatur T des Gefaesses errechnen:
\begin{equation}
    p_{saet}(T)=5,5\cdot10^7\cdot e^{\frac{-6876}{T}} \text{[p in mbar, T in K]}
\end{equation}
Die gesamte, relevante Dampfdruckkurve ist in \ref{Fig:Dampf} dargestellt.
\begin{figure}[H]
    \centering
    \captionsetup{justification=centering}
    \includegraphics[height=12cm]{"Dampf_FranckHertz.png"}
    \captionbelow{Dampfdruckkurve des Quecksilbers \\ Aus: Anleitung V601 Seite 121}
    \label{Fig:Dampf}
\end{figure}
Nun kann eine Temperatur bestimmt werden, bei der der Franck-Hertz-Effekt beobachtet werden kann. $\bar{w}$ sollte hierbei um den Faktor 1000 bis 4000 kleiner sein als a. a betraegt hier bei dieser Roehre etwa 1cm. Somit gibt es einen Dampfdruckbereich in dem die Roehre optimal arbeitet. Ist der Dampfdruck zu gering, laufen die Elektronen ohne Wechselwirkung mit dem Hg zur Auffaengerelektrode. Theoretisch ist es bei grossen Beschleunigungsspannungen moeglich das Hg-Atom hoeher als bis zum ersten Anregungszustand anzuregen, jedoch ist dies auf Grund der geringen Stosswahrscheinlichkeit selten zu beobachten.\\
Wird $p_{saet}$ zu gross gewaehlt, kommt zu sehr vielen elastischen Stoessen, was zu einer starken Verringerung des Auffaengerstroms fuehrt. Dieser Effekt ruehrt daher, dass die elastischen Stoesse -wie bereits genannt- zu starken Richtungsaenderungen der Elektronen fuehrt.

\end{document}