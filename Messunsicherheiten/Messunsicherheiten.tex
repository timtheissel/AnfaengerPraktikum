\documentclass{scrartcl}
\usepackage{amsmath}
\begin{document}

%Hallo
\section{Aufgabe 1}
\subsection{1.1 Was bezeichnet der Mittelwert?}

Der Mittelwert bezeichnet den Durchsnitt der gemessenen Größen.

\subsection{1.2 Welche Bedeutung hat die Standardabweichung?}

Die Standardabweichung ist ein Maß für die Streuung der Meßwerte um den Mittelwert.
Sie gibt an, wie weit die Werte im Durchschnitt von ihrem Mittelwert entfernt sind.

\subsection{1.3 Worin unterscheidet sich die Streuung der Messwerte und der Fehler des Mittelwertes?}

Die Streuung der Messwerte bezieht sich auf die Messwerte und gibt an wie weit sie vom Mittelwert entfernt sind.
Der Fehler des Mittelwertes ist eine Unsicherheit, die durch Messfehler entsteht.
Da die Werte nicht alle mit maximaler Genauigkeit gemessen wurden und man den Mittelwert aus diesen fehlerbehafteten Größen berechnet,
hat auch dieser einen Fehler.
Daher ist der aus den Messwerten ermittelte Mittelwert nicht exakt richtig. 

Der Fehler des Mittelwertes gibt also an, um welchen Wert dieser kleiner oder größer sein kann, weil bei der Messung die Werte nicht exakt aufgenommen wurden.
Die Streuung hingegen bezieht sich lediglich auf den Abstand der einzelnen Werte zum Mittelwert.

\section{Aufgabe 2}

\begin{displaymath}
    \sigma = \sqrt{\frac{1}{n-1} \sum_{i=1}^{n} (x_i - \bar{x})^2}
\end{displaymath}

Die kleinstmögliche Versuchsanzahl ist n=2, da n=0 die Standardabweichung nicht verändert und n=1 eine 0 im Nenner liefern würde.
Um bei dieser kleinstmöglichen Versuchsanzahl eine Standardabweichung von 10 $\frac{m}{s}$ zu erhalten muss $\sum_{i=1}^{n} (x_i - \bar{x})^2 = 100$ gelten.
Denn:
\begin{displaymath}
    \sigma = \sqrt{\frac{1}{2-1} * 100 } = 10 \frac{m}{s}
\end{displaymath}

Der Einfachheit halber wird $\sum_{i=1}^{n} (x_i - \bar{x})^2 = 100$ mit X bezeichnet.

Um nun an die Anzahl der Versuche zu gelangen sind einige Umformungen nötig.

\begin{align}
    \sigma^2 = \frac{1}{n-1} X            \nonumber \\
    \frac{\sigma^2}{X} = \frac{1}{n-1}    \nonumber \\
    \frac{X}{\sigma^2} = n -1             \nonumber \\
    \frac{X}{\sigma^2} + 1 = n              \nonumber
\end{align}

Mit $\frac{X}{\sigma^2} + 1 = n$ lassen sich nun die nötigen Versuche bestimmen. 

Einsetzen von X=100 und $\sigma$ = 3 $\frac{m}{s}$ liefert:

\begin{displaymath}
    \frac{100}{3^2} + 1 = 12.111111 
\end{displaymath}

Für eine Unsicherheit von $\pm3 \frac{m}{s}$ werden also ungefähr 13 Durchführungen benötigt.
Da aber berits 2 Versuche benötigt werden um eine Unsicherheit von 10 $\frac{m}{s}$ zu erreichen werden von diesem Punkt nur noch 11 weitere Durchführungen benötigt.

Mit $\frac{X}{\sigma^2} + 1 = n$ lassen sich nun die nötigen Versuche bestimmen. 

Einsetzen von X=100 und $\sigma$ = 0.5 $\frac{m}{s}$ liefert:

\begin{displaymath}
    \frac{100}{0.5^2} + 1 = 401
\end{displaymath}

Für eine Unsicherheit von $\pm0.5 \frac{m}{s}$ werden also ungefähr 401 Durchführungen benötigt.
Da aber berits 2 Versuche benötigt werden um eine Unsicherheit von 10 $\frac{m}{s}$ zu erreichen werden von diesem Punkt nur noch 399 weitere Durchführungen benötigt.

\section{Aufgabe 3}

Das Volumen eines Hohlzylinders ist :

\begin{displaymath}
    V=\pi \cdot h \cdot (R_{aussen}^2-R_{innen}^2)
\end{displaymath}

R\textsubscript{außen} = (15$\pm$1) cm ;
R\textsubscript{innen} = (10$\pm$1) cm ;
h = (20$\pm$1) cm

Damit folgt für das Volumen:

V = 7853.98 $cm^3$

Für den Fehler gilt: 

\begin{displaymath}
    \Delta V = \sqrt{\left(\frac{(\partial V)}{(\partial R_{aussen})²}\right)^2 (\Delta R_{aussen} )^2 +
                     \left(\frac{(\partial f)²}{(\partial R_{innen})}\right)^2 (\Delta R_{innen})^2 +
                     \left(\frac{(\partial f)²}{(\partial h)}\right)^2 (\Delta h)^2
    }
\end{displaymath}

\begin{displaymath}
    \Delta V = \sqrt{(2\pi \cdot R_{aussen} \cdot h)^2 (\Delta R_{aussen} )^2 +
                     (-2\pi \cdot R_{innen} \cdot h)^2 (\Delta R_{innen})^2 +
                     (\pi \cdot (R_{aussen}^2-R_{innen}^2)^2 (\Delta h)^2
    }
\end{displaymath}

Mit R\textsubscript{außen} = (15$\pm$1) cm ;
R\textsubscript{innen} = (10$\pm$1) cm ;
h = (20$\pm$1) cm 

ist der Fehler:

\begin{displaymath}
    \Delta V = 229.92 cm^3
\end{displaymath}

Das Volumen beträgt also 7853.98 $\pm$ 229.92 $cm^3$

\section{Aufgabe 4}
\subsection{Strecke}

Für die Strecke gilt:
\begin{displaymath}
    s = v \cdot t 
\end{displaymath}
s = 6s ; v = 200$\frac{m}{s}$

\begin{displaymath}
    s = 6 \cdot 200 = 1200m
\end{displaymath}

Der Fehler ergibt sich zu:

\begin{align}
    \Delta s = \sqrt{\left(\frac{\partial s}{\partial v}\right)^2 (\Delta v )^2} \\ \nonumber
    \Delta s = \sqrt{t^2 (\Delta v )^2} \nonumber
\end{align}

Da $\Delta v$ = 10 $\frac{m}{s}$ ist der Fehler $\Delta s$ = 60$\frac{m}{s}$

Die Strecke ist also (1200$\pm$60)$\frac{m}{s}$

\subsection{Energie}

Die kinetische Energie ist:

\begin{displaymath}
    E=\frac{m}{2} v^2
\end{displaymath}

m = 0.005kg ;  v = 200$\frac{m}{s}$

E=100J

Der Fehler ist:

\begin{align}
    \Delta E = \sqrt{\left(\frac{\partial E}{\partial v}\right)^2 (\Delta v )^2 + \left(\frac{\partial E}{\partial m}\right)^2 (\Delta m)^2}  \nonumber \\
    \Delta E = \sqrt{(mv)^2 (\Delta v )^2 + \left(\frac{v^2}{2}\right)^2 (\Delta m)^2} \nonumber
\end{align}

$\Delta v$ = 10 $\frac{m}{s}$ ; $\Delta m$ = 0.0001kg

$\Delta E$ = 10.2J

Damit beträgt die Energie des Projektils 100$\pm$10.2J.
\end{document}