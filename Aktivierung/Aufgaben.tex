\documentclass[titlepage=firstcover, captions=tableheading]{scrartcl}
\usepackage{microtype}
\usepackage{amsmath}
\usepackage{polyglossia}
\usepackage{graphicx}
\usepackage{booktabs}
\usepackage{siunitx}
\usepackage{hyperref}
\usepackage{caption}
\usepackage{float}
\begin{document}

\section{Bestimmung des Untergrundes}

Zur Bestimmung des Untergrundes wurden 7 Messungen durchgeführt.
Für die Messungen wurde das Zählrohr eingeschaltet ohne eine Probe auf das Zählrohr zu stecken.
Die Messungen liefen jeweils 300s.

\noindent Die enstanden Werte dieser Messungen sind in Tabelle \ref{tab:1} zu finden.

\noindent Aus diesen Werten wird nun ein arithmetischer Mittelwert berechent.
Dieser wird dann noch auf ein Zeitintervall von 30s skaliert, damit dieser Mittelwert in den Folgenden Aufgaben  ohne weitere Anpassung verwendet werden kann.

\noindent Es ergibt sich eine Untergrundrate von $13.9 \approx 14 [N_U/30s]$.

\section{Bestimmung der Halbwertszeit von Vanadium}

In diesem Teil wurde  die aktivierte Vanadiumprobe umgehend auf das Geiger-Müller-Zählrohr gesteckt und es wurden die Impulse in einem Zeitintervall von 30s abgelesen.
Diese Messdaten sind in Tabelle \ref{tab:2} zu finden. 
Von diesen Werten wird der im letzten Teil berechnete Mittelwert abgezogen.
Anschließend werden die entstehenden Werte mit Fehler in eimen halblogarithmischen Diagramm dargestellt.

\begin{figure}[H]
    \centering
    \includegraphics{"22.png"}
    \caption{Bestimmung der Halbwertszeit von Vanadium}
    \label{Fig:Halbwertszeit}
\end{figure}

\noindent Als Ausgleichsfunktion wurde das Zerfallsgesetz verwendet.
Dieses wurde für die Bestimmung der Halbwertszeit dann noch umgestellt.
\begin{align}
    N(t) &= N_0 * e^{-\lambda t} \nonumber \\
    \frac{1}{2} N_0 &= N_0 * e^{-\lambda T} \nonumber \\
    T &= \frac{\ln(2)}{\lambda} \nonumber
\end{align}

\noindent Für die erste Ausgleichsrechnung wurden alle gemessenen Werte verwendet.
Mit der Python-Funktion "$scipy.optimize curve_fit$" ergibt sich:
\begin{align}
    N_0 &=  208.761825 \pm 6.069003 \nonumber \\
    \lambda &= 0.002657 \pm 0.000106 \nonumber\\ 
\end{align}

Daraus lässt sich die Halbwertszeit berechnen zu:
\begin{displaymath}
    T_1 = 261 \pm 10 s 
\end{displaymath}

Bei der zweiten Ausgleichrechnung werden nun nur die Werte bis zur doppelten Halbwertszeit für die Ausgleichsrechnung verwendet.
Also alle Messwerte bis zu einer Zeit von maximal 520s.
Für die Parameter ergeben sich diesmal die Werte:

\begin{align}
    N_0 &=  218.126931 \pm 7.601301 \nonumber \\
    \lambda &= 0.002937 \pm 0.000170 \nonumber\\ 
\end{align}

Die etstehende Halbwertszeit hat nun den Wert:

\begin{displaymath}
    T_2 = 236 \pm 14 s 
\end{displaymath}

\section{Halbwertszeit von Rhodium}

Analog zu der Vorgehensweise bei Vanadium wird auch bei der Halbwertszeit zunächst ein halblogarithmisches Diagramm erstellt.
Diesmal wird die Ungergrundrate wieder angepasst, da das Zeitintervall nun nur 15s beträgt.
Die gemessenen Werte für Rhodium sind in Tabelle \ref{tab:3} zu sehen.

Das entstehende Diagramm mit dem angepassten Untergrund und dem Fehler $\sqrt{N}$ sieht folgendermaßen aus:

\begin{figure}[H]
    \centering
    \includegraphics{"3.png"}
    \caption{Bestimmung der Halbwertszeit von Rhodium}
    \label{Fig:HalbwertszeitRh}
\end{figure}




\section{Tabellen}
\begin{minipage}{\linewidth}
    \begin{table}[H]
        \centering
    
    \begin{tabular}{c}
        \toprule
        $N_U$ [Imp/300s]\\
        \midrule
        129 \\
        143 \\
        144 \\
        136 \\
        139 \\
        126 \\
        158 \\
        \bottomrule
        
    \end{tabular}
    \captionof{table}{Messung zur Bestimmung des Untergrundes}
    \label{tab:1}
\end{table}
\end{minipage}


\begin{minipage}{\linewidth}
    \begin{table}[H]
        \centering
    
    \begin{tabular}{ll}
        \toprule
        t [s] & N [Imp] \\
        \midrule
        30	& 189 \\
        60	& 197 \\
        90	& 150 \\
        120	& 159 \\
        150	& 155 \\
        180	& 132 \\
        210	& 117 \\
        240	& 107 \\
        270	& 94 \\
        300	& 100 \\
        330	& 79 \\
        360	& 69 \\
        390	& 81 \\
        420	& 46 \\
        450	& 49 \\
        480	& 61 \\
        510	& 56 \\
        540	& 40 \\
        570	& 45 \\
        600	& 32 \\
        630	& 27 \\
        660	& 43 \\
        690	& 35 \\
        720	& 19 \\
        750	& 28 \\
        780	& 27 \\
        810	& 36 \\
        840	& 25 \\
        870	& 29 \\
        900	& 18 \\
        930	& 17 \\
        960	& 24 \\
        990	& 21 \\
        1020 &	25 \\
        1050 &	21 \\
        1080 &	24 \\
        1110 &	25 \\
        1140 &	17 \\
        1170 &	20 \\
        1200 &	19 \\
        1230 &	20 \\
        1260 &	18 \\
        1290 &	16 \\
        1320 &	17 \\
        \bottomrule
        
    \end{tabular}
    \captionof{table}{Messwerte zur Bestimmung der Halbwertszeit von Vanadium}
    \label{tab:2}
\end{table}
\end{minipage}


\begin{minipage}{\linewidth}
    \begin{table}[H]
        \centering
    
    \begin{tabular}{ll}
        \toprule
        t [s] & N [Imp] \\
        \midrule
        15 &	 667 \\
        30 &	 585 \\ 
        45 &	 474 \\
        60 &	 399 \\
        75 &	 304 \\
        90 &	 253 \\
        105	&    213 \\
        120	&    173 \\
        135	&    152 \\
        150	&    126 \\
        165	&    111 \\
        180	&     92 \\
        195	&     79 \\
        210	&     74 \\
        225	&     60 \\
        240	&     52 \\
        255	&     56 \\
        270	&     53 \\
        285	&     41 \\
        300	&     36 \\
        315	&     37 \\
        330	&     32 \\
        345	&     36 \\
        360	&     38 \\
        375	&     34 \\
        390	&     40 \\
        405	&     21 \\
        420	&     35 \\
        435	&     33 \\
        450	&     36 \\
        465	&     20 \\
        480	&     24 \\
        495	&     30 \\
        510	&     30 \\
        525	&     26 \\
        540	&     28 \\
        555	&     23 \\
        570	&     20 \\
        585	&     28 \\
        600	&     17 \\
        615	&     26 \\
        630	&     19 \\
        645	&     13 \\
        660	&     17 \\
        \bottomrule
        
    \end{tabular}
    \captionof{table}{Messwerte zur Bestimmung der Halbwertszeit von Rhodium}
    \label{tab:3}
\end{table}
\end{minipage}
\end{document}