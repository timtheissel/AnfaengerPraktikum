\section{Zielsetzung}
Im folgenden Versuch soll der Wellencharakter des Lichts anhand der Beugung an einem Einfach- sowie Doppelspalt untersucht werden. Das Interferenzmuster ist dabei von besonderer Wichtigkeit.
\section{Theoretische Grundlagen}
Beugung ist im Kontext von Lichtstrahlen die Abweichung von der geometrischen Optik beim Passieren von Schlitzen duenner als der Strahldurchmesser des Lichtstrahls. Wie bereits beschrieben benoetigt es die Auffassung des Lichts als Welle, um die auftretenden Phaenomene zu beschreiben. Diese Auffassung wird dem Licht in der Natur eigentlich nicht gerecht, das Modell genuegt zur Beschreibung dieses Experiments aber. Besonders entscheidend ist dabei das so genannte Huygenssche Prinzip. Dieses besagt, dass jeder Punkt einer Wellenfront Quelle einer neuen Elementarwelle ist. Die Einhuellende dieser Elementarwellen bildet dann die neue Wellenfront. 