\section{Zielsetzung}
Im folgenden Experiment sollen die magnetischen Eigenschaften stark paramagnetischer Stoffe untersucht werden. Diese stark paramagnetischen Stoffe sind seltene Erden wie Neodym oder Gadolinium. Besonderes Augenmerk soll dabei auf die Ermittlung der Suszeptibilität $\chi$ aus atomaren Größen, sowie der experimentellen Ermittlung von $\chi$ mit Hilfe einer Brückenschaltung, gelegt werden. Diese Brückenschaltung und die dafür notwendigen Apparaturen sollen in mehr Detail beschrieben werden.
\section{Theoretische Grundlagen}
\subsection{Berechnung der Suszeptibilität}
Die magnetische Flussdichte $\vec{B}$ und die magnetische Feldstärke $\vec{H}$ hängen im Vakuum über die Induktionskonstante $\mu_0$ zusammen. Somit ergibt sich für $\vec{B}=\mu_0\vec{H}$.. In Anwesenheit von Materie verändert sich dieser Ausdruck um die Magnetisierung $\vec{M}$ zu $\vec{B}=\mu_0\vec{H}+\vec{M}$ mit $\vec{M}=\mu_0\chi\vec{H}$. Hierin stellt der Faktor $\chi$ die sogenannte Suszeptibilität dar.\\
Eines der magnetischen Phänomene, welches sich bei allen atomen zeigt, ist der Diamagnetismus. Dabei wird in einem Material ein magnetisches Moment induziert, welches dem äusseren Magnetfeld entgegen gerichtet ist. Daher ist in diesem Fall $\chi<0$. Im Unterschied dazu steht der Paramagnetismus, welcher nur bei Atomen mit nicht verschwindendem Drehimpuls auftritt. Diese Größe ist temperaturabhängig. Die Herleitung erfolgt über die atomare Größe des Gesamtdrehimpulses $\vec{J}$. Dieser setzt sich zusammen aus dem Gesamtbahndrehimpuls $\vec{L}$ und dem Gesamtspin $\vec{S}$, also $\vec{J}=\vec{L}+\vec{S}$. Die magnetischen Momente zu $\vec{L}$ und $\vec{S}$ ergeben sich zu 
\begin{equation*}
    \vec{\mu_L}=-\frac{\mu_B}{\hbar}\vec{L}
\end{equation*}
und
\begin{equation*}
    \vec{\mu_S}=-g_S\frac{\mu_B}{\hbar}\vec{S}
\end{equation*}.
Die Konstante $\mu_B:=\frac{e_0\hbar}{2m_0}$ ist das sogenannte Bohrsche Magneton mit der Ruhemasse des Elektrons $m_0$, dem gyromagnetischen Verhältnis des freien Elektrons $g_S$ und dem reduzierten plankschen Wirkungsquantum $\hbar=\frac{h}{2\pi}$. Für die Beträge der Momente ergebn sich zu 
\begin{equation*}
    |\vec{\mu_L}|=\mu_B\sqrt{L(L+1)}
\end{equation*}
und
\begin{equation*}
    |\vec{\mu_L}|=g_S\mu_B\sqrt{S(S+1)}
\end{equation*}.
Mit diesen Gleichungen kann nun für das magnetische Moment $|\vec{\mu_J}|$ des Gesamtdrehimpulses entsprechend des Vektordiagramms Abb \ref{Fig:Vektor} der Zusammenhang $|\vec{\mu_J}|=|\vec{\mu_S}|cos\alpha+|\vec{\mu_L}cos\beta$ hergestellt werden.
\begin{figure}[H]
    \centering
    \captionsetup{justification=centering}
    \includegraphics[height=7cm]{"Vektor_Paramagnetismus.png"}
    \captionbelow{Vektordiagramm der Drehimpulsvektoren der Elektronenhülle und ihrer magnetischen Momente \\ Aus: \cite{V606}}
    \label{Fig:Vektor}
\end{figure}
Unter Ausnutzung des Cosinussatzes, Einsetzen der vorherigen Gleichungen und der Vereinfachung der Größe $g_S$ auf den Wert 2, ergibt sich $|\vec{\mu_J}|\approx\mu_Bg_J\sqrt{J(J+1)}$ mit dem Land\'{e}-Faktor $g_j:=\frac{3J(J+1)+[S(S+1)-L(L+1)]}{2J(J+1)}$. Durch das Phänomen der Richtungsquantelung aus der Quantenmechanik ist nicht jede räumliche Orientierung möglich, sondern nur solche, bei denen beispielweise die Komponente $\mu_{Jz}$ von $\vec{\mu_J}$ ein ganzzahliges Vielfaches von $\mu_Bg_J$ darstellt, also $\mu_Jz=-\mu_Bg_Jm$. mist hierbei die Orientierungsquantenzahl. Für die potentielle Energie zu einem bestimmten m ergibt sich $E_m=\mu_Bg_JmB$. Nach dem Zeeman-Effekt spaltet sich ein Energieniveau in $2J+1$ Unterniveaus auf. Nun muss zur Berechnung der Magnetisierung die Häufigkeit mit der eine Orientierung auftritt mit dem Betrag des magnetischen Moments für diese Orientierung multiplizieren und über alle Orientierungen summieren. Nach Durchführung dieser komplizierten Rechnungen unter Verwendung von etwa der Hochtemperaturnäherung ergibt sich für die letztlich gesuchte Suszeptibilität $\chi$
\begin{equation}
    \chi=\frac{\mu_0\mu_b^2g_j^2NJ(J+1)}{3kT} \label{Chi}
\end{equation}.
Dies lässt sich für sehr hohe Temperaturen zum Curieschen Gesetz des Paramagnetismus vereinfachen mit $\chi\approx\frac{1}{T}$.
\subsection{Berechnung der Suszeptibilität seltener Erd-Verbindungen}
Seltene Erd-Ionen müssen grosse Drehimpulse der Hüllenelektronen haben, da sie einen starken Paramagnetismus zeigen. Diese Ionen haben eine gesättigte Xe-Hülle und 2 6s-Elektronen, welche aber für die Berechnungen keine Rolle spielen. Vielmehr sind die 4f-Elektronen entscheidend. Diese sind erst vom Cer (z=58) an zu finden. Die nachfolgenden Atome im Periodensystem haben bis zum Ytterbium (z=70) immer jeweils ein 4f-Elektron mehr als der Vorgänger. Die Anordnung dieser 4f-Elektronen wird durch die Hundschen Regeln festgelegt.\\\\
\textbf{Erste Hundsche Regel:} Die Spins $\vec{s_i}$ addieren sich zum maximalen Gesamtspin $\vec{S}=\sum\vec{s_i}$, welcher durch das Pauliprinzip erlaubt wird.\\\\
\textbf{Zweite Hundsche Regel:} Die Bahndrehimpuls $\vec{l_i}$ ergeben sich so, dass der maximale Drehimpuls $\vec{L}=\sum\vec{l_i}$ entsteht, welcher pauli-verträglich ist und Regel 1 nicht verletzt.\\\\
\textbf{Dritte Hundsche Regel:} Der Gesamtdrehimpuls ist $\vec{J}=\vec{L}+\vec{S}$, wenn die Schale mehr als halb voll ist und $\vec{J}=\vec{L}-\vec{S}$, wenn sie nicht zu mehr als der Hälfte gefüllt ist.\\\\
Diese Regeln werden benötigt, um nach Gleichung \ref{Chi} die Suszeptibilität zu berechnen. Es muss also der Drehimpuls J sowie der Lande-Faktor bestimmt werden. Dies wird hier am Beispiel der $Pr^{3+}$-Hülle demonstriert. Es besitzt drei 4f-Elektronen und zwei 6s-Elektronen. Durch die Ionisierung fehlen zwei 6s-Elektronen und ein 4f-Elektron. Nach der ersten Hundschen Regel stellen sich die Spins der übrig gebliebenen 4f-Elektronen parallel, also $S=\frac{1}{2}+\frac{1}{2}=1$. $l_{max}=3$ ergibt sich daher, dass es sich um eine f-Schale handelt. Nach der zweiten Regle kann aber nur ein Elektron den Bahndrehimpuls $l=3$ haben, weshalb das andere dann $l=2$ hat und so $L=3+2=5$ ist. Die 4f-Schale ist zu weniger als der Hälfte gefüllt, wodurch die dritte Regel diktiert, dass $J=L-S=5-1=4$. Mit $L=5$, $S=1$ und $J=4$ ist der Lande-Faktor dann 
\begin{equation*}
    g_j(Pr^{3+})=\frac{3\cdot 4\cdot 5+1\cdot 2-5\cdot 6}{2\cdot 4\cdot 5}=0.8
\end{equation*}.
\section{Beschreibung und Aufbau einer Apparatur zur Messung der Suszeptibilität}
Die Induktivität einer langen Spule beträgt  
\begin{equation}
    L=\mu_0\frac{n^2F}{l} \label{21}
\end{equation} 
mit der Windungszahl n, der Länge l und dem Querschnitt F der Spule. Da es meist nicht gelingt das Innere der Spule vollständig mit Materie zu füllen ergibt sich für eine mit Materie gefüllte Spule mit dem Probenquerschnitt Q der Ausdruck
\begin{equation*}
    L_M=\mu_0\frac{n^2F}{l}+\chi\mu_0\frac{n^2Q}{l} 
\end{equation*}.
So kann mit einer Messung der Differenz die Suszeptibilität bestimmt werden. Diese Differenz ist aber sehr gering, nämlich 
\begin{equation}
    \Delta L=\mu_0\chi\frac{Qn^2}{l} \label{22}
\end{equation}
Um diesen Umstand zu umgehen muss eine möglichst gute Messapparatur verwendet werden mit möglichst gleichen Spulen, welche nach Abbildung \ref{Fig:Bruecke} zu einer Brückenschaltung verbunden werden. 
\begin{figure}[H]
    \centering
    \captionsetup{justification=centering}
    \includegraphics[height=7cm]{"Bruecke_Paramagnetismus.png"}
    \captionbelow{Brückenschaltung zur Bestimmung von Suszeptibilitäten\\ Aus: \cite{V606}}
    \label{Fig:Bruecke}
\end{figure}
Mit diesem Aufbau lässt sich die Suszeptibilität auf zwei Arten bestimmen. Bei der ersten Art wird die Brücke abgeglichen und die Brückenspannung $U_Br$ gemessen, welche entsteht, wenn die Probe in die Spule geschoben wird. Die zweite Art beruht darauf die Brücke mit Probe abzugleichen und mit der Aenderung der Abgleichelemente $\chi$ zu bestimmen.\\
Bei der in Abbildung \ref{Fig:Bruecke} skizzierten Brücke ergibt sich die Abgleichbedingung zu 
\begin{equation*}
    U_{Br}=\frac{r_4r_1-r_3r_2}{(r_1+r_2)(r_3+r_4)}U_{Sp}
\end{equation*}
Die komplexen Widerstände $r_i$ ergeben sich zu 
\begin{align*}
    r_1&=R_M+j\omega L_M
    \frac{1}{r_2}&=\frac{1}{R_p}+\frac{1}{R+j\omega L}
    r_3=R_3
    r_4=R_4
\end{align*}
Mit den vereinfachenden Näherungen, dass $R_p>>R$, $R_p>>\omega L$, $R_3\approx R_4$ und $R\approx R_M$ lässt sich der Betrag der Brückenspannung schreiben als 
\begin{equation*}
    U_{Br}=\frac{\omega \Delta L\cdot U_{Sp}}{4\sqrt{R^2+\omega^2L^2}}
\end{equation*}
Mit den Gleichungen \ref{21} und \ref{22} ergibt sich nach Einsetzen und Umformen
\begin{equation*}
    \chi=\frac{U_{Br}\cdot 4l\sqrt{R^2+\omega^2(\mu_0\frac{n^2F}{l})^2}}{U_{Sp}\cdot\omega\mu_0n^2Q}
\end{equation*}.
Daher ist die Suszeptibilität $\chi$ für grosse Frequenzen $(\omega^2L^2>>R^2)$
\begin{equation*}
    \chi=\frac{4F\cdot U_{Br}}{Q\cdot U_{Sp}}
\end{equation*}\\\\
Die Abgleichbedingung der Brücke ohne Probe ist $r_1R_4=r_2R_3$ mit $R_3\approx R_4$. Nach Einbau der Probe muss $R_3$ sich um den Wert $\Delta R$ ändern, um wieder abgeglichen zu sein. Mit $R_3+R_4=const$ lautet die Abgleichbedingung dann 
\begin{equation*}
    (R_M)+j\omega L_M)(R_3-\Delta R)=(R+j\omega L)(R_3+\Delta R)
\end{equation*}
Der Imaginärteil ergibt sich dann zu $\Delta R=\frac{R_3(L_M-L)}{L+L_M}$ und mit der Ersetzung $L_M=L+\Delta L$ zu 
\begin{equation*}
    \Delta R=\frac{R_3\Delta L}{2L+\Delta L}\approx\frac{\Delta LR_3}{2L}
\end{equation*}
Mit den Gleichungen \ref{21} und \ref{22} ergibt sich nach Einsetzen und Umformen schliesslich
\begin{equation*}
    \chi=2\frac{\Delta R\cdot F}{R_3Q}
\end{equation*}
\section{Unterdrückung der Störspannungen}
Da die Störspannungen verglichen mit der zu messenden Spannung verhältnismässig gross sind, ist es von Nöten diese herauszufiltern. Dies wird mit einem Selektivverstärker realisiert. Da die Signalspannung monofrequent ist, reicht es wenn nur diese Frequenz die Selektion passieren kann. Dieser Selektivverstärker hat eine Filterkurve, welche einer gausssche Glockenkurve entspricht, welche einem Verhältnis aus Ausgangsspannung $U_A$ zu Eingangsspannung $U_E$ entspricht. Die Wirksamkeit dieser Unterdrückung ungewollter Frequenzen wird an der Breite dieser Kurve gemessen. Die so genannte Güte entspricht dabei $Q=\frac{v_0}{v_+-V_-}$. Die Frequenzen $v_-$ und $v_+$ ensprechen den Frequenzen, bei denen das Verhältnis von $U_A$ zu $U_E$ auf $\frac{1}{\sqrt{2}}$ gesunken ist. Es ist somit auch leicht abzusehen, dass ein solcher Filter die Störspannungen nicht vollständig entfernen kann. Stattdessen werden Frequenzen, die nah an $v_0$ liegen, abgeschwächt, aber können trotzdem passieren.
\section{Durchführung}
\subsection{Messung der Filterkurve des Selektivverstärkers}
Zur Bestimmung der Filterkurve wird mit einem Frequenzgenerator eine bestimmte Frequenz generiert und durch den Selektivverstärker in ein Millivoltmeter gespeist. Die angezeigte Spannung wird notiert und aufgetragen gegen die Frequenz. Vorallem um $v_0$ sollten vermehrt Messungen durchgeführt werden.
\subsection{Experimentelle Bestimmung der Suszeptibilität}
Der an die Brückenschaltung angeschlossene Frequenzgenerator wird möglichst exakt auf die Durchlassfrequenz $v_0$ des Selektivverstärkers eingestellt. Die Güte wird auf $Q=20$ eingestellt. Nun wird die Brücke möglichst genau abgeglichen mit Hilfe der Abgleichelemente, also so, dass die Brückenspannung möglichst $U_{Br}=0$ ist. Nun werden die Einstellungen der Abgleichelemente und die Restspannung notiert. Daraufhin wird eine Probe in die Spule geschoben und die enstandene Brückenspannung für die erste Methode der Bestimmung der Suszeptibilität notiert. Schliesslich wird nun die Brücke für die zweite Methode mit der Probe in der Spule abgeglichen und auch hier die Einstellung der Abgleichelemente notiert.
