\section{Vorbereitungsaufgaben}

\subsection{Energiestruktur eines Atoms}

\subsubsection{Welche Quantenzahlen definieren den Zustand eines Elektrons im Wasserstoffatom?}

Der Zstand des Elektrons im Wasserstoffatom wird definiert durch 4 Quantenzahlen:

-Die Hauptquantenzahl n gibt die Schale an.

-Die nebenquantenzahl l gibt das Unterniveau an.

-Die Magnetquantenzahl m gibt die Orientierung des Bahndrehimpulses an.

und die Spinquantenzahl s.

\subsubsection{Was ist die Feinstruktur (Spin-Bahn-Kopplung)?}

Die Feinstruktur ist die Aufspaltung der Energieniveaus über die Hauptniveaus hinaus.

Bei der Spin-Bahn-Kopplung handelt es sich um eine Wechselwirkung zwischen dem Spin und dem eigenen Magnetfeld. Der Gesamtdrehimpuls ergibt sich hierbei aus der Summe des Bahndrehimpulses mit dem Spin.

\subsubsection{Beschreiben sie die LS-Kopplung und die jj-Kopplung.}

Die LS-Kopplung tritt bei leichten Atomen (Ordnungszahl Z<50) auf. Die LS-Kopplung ist die Kopplung zwischen den Bahndrehimpulsen und den Spins der Elektronen. Bei den leichten Atomen ist diese Kopplung stärker als die Spin-Bahn-Kopplung und eine gute Näherung

Bei schwereren Atomen wird die Spin-Bahn-Kopplung stärker und dann ist die jj-Kopplung eine gute Näherung für die Energiezustände des Atoms. 

\subsection{Magnetisches Moment des Atoms}

\subsubsection{Wie ist der Zusammenhang zwischen dem Magnetischen Moment eines Elektrons und seinen Drehimpulsquantenzahlen?}

Der Zusammenhang ist das gyromagnetische Verhältnis. Dieses beschreibt den Faktor zwischen dem Drehimpuls und dem magnetischen Moment eines Teilchens.

\subsubsection{Was ist der Unterschied zwischen dem gyromagnetischen Verhältnis für den Elektronenspin und demjenigen für den Drehimpuls?}

Das gyromagnetische Verhältnis unterscheidet sich um einen g-Faktor, dieser ist beim Bahndrehimpuls 1 und verändert das Verhältnbis somit nicht. Für den Spin ist dieser g-Faaktor allerdings 2 und deshalb unterscheiden sich die Verhältnisse um einen Faktor 2.

\subsection{Aufspaltung des Energieniveaus eines Atoms im homogenen Feld}

Der Lande Faktor ist das Verhältnis zwischen dem Drehimpuls und dem magnetischen Moment.

\subsubsection{In wie viele Niveaus spaltet sich der Zustand mit Gesamtdrehimpuls J?}

Ein Zustand mit Gesamtdrehimpuls J spaltet sich in 2J+1 äquidistante Niveaus auf.

\subsubsection{Wie ändert sich die Aufspaltung mit der Magnetfeldstärke?}

Bei kleinen Feldstärken tritt der Zeeman Effekt auf. In dieser Auspaltung entstehen äquidistante Energieniveaus und die Spin-Bahn-Kopplung ist vorherrschend.

Bei großen Feldstärken ist die LS-Kopplung gestört. Die Energieniveaus sind bei dieser Aufspaltung nicht mehr äquidistant und der Paschen-Bach-Effekt kommt hinzu.

\subsubsection{Was versteht man unter dem Paschen-Bach-Effekt}

Bei großen Magnetfeldstärken tritt der Paschen-Bach-Effekt auf. Durch die großen äußeren Magnetfelder wird die LS-Kopplung gestört. Bei der Betrachtung der Spektrallinien sieht der eigentlich auftretende annormale Zeeman Effekt aus wie der normale Zeeman Effekt.

\subsection{Optische Übergänge zwischen Zeemanaufgespaltenen Energieniveaus}

Optische Übergänge sind Übergänge die mit Photonenemission oder -absorption funktionieren. Dabei verändert sich die Magnetquantenzahl m. Erlaubte Übergänge sind Übergänge bei denen gilt, dass die Differenz aus der alten und neuen Quantenzahl m 0, 1 oder -1 ist. Also:

\begin{align}
    m_alt - m_neu = \Delta m = [-1,0,1]
\end{align}

Es werden bei dem annormalen Zeeman Effekt mehr Spektrallinien beobachtet als beim normalen Zeeman Effekt. Dabei sind die Spektrallinien zirkular polarisiert wenn $\Delta m =  [-1,+1]$ und linear polarisiert, wenn $\Delta m = 0 $.   

\subsection{Optische Übergänge in Cd-Atomen}

\subsubsection{Termschema}

\subsubsection{Berechnen sie die Lande Faktoren g\textsubscript{i} und die Aufspaltung $\Delta E$ der Zeeman Linien}

Mit der Formel,

\begin{align}
    g_i = 1 + \frac{j(j + 1) - l(l+1) + s(s+1)}{2j(j+1)}
\end{align}

wobei j die Gesamtdrehimpulsquantenzahl, l die Bahndrehimpulsquantenzahl und s die Spinquantenzahl ist, lassen sich die Lande Faktoren berechnen.
Für die Zustände $^1p_1$, $^1D_2$, $^3p_1$ und $^3s_1$ ergeben sich folgende Lande Faktoren:

\begin{minipage}{\linewidth}
    \begin{table}[H]
        \centering
    \captionof{table}{Lande Faktoren}
    \begin{tabular}{ll}
        \toprule
        Zustand & Lande Faktor \\
        \midrule
        $^1p_1$ & 1.0 \\
        $^1D_2$ & 1.0 \\
        $^3p_1$ & 1.5 \\
        $^3s_1$ & 2.0 \\
        \bottomrule   
    \end{tabular}
    
    \label{tab:1}
\end{table}
\end{minipage}

Die Aufspaltung $\Delta E$ wird mit folgender Formel berechnet:

\begin{align}
    \Delta E = (mg_{i,2}-m_1g_{i,1})
\end{align}


\section{Auswertung}

\subsection{Eichung des Elektromagneten}

Für die Eichung des Elektromagneten wird eine Hallsonde verwendet. Diese wird mithilfe eines Stativs zwischen die Polschuhe des Elektromagneten gebracht. Nun wird das Magnetfeld in Abhängigkeit von der Stromstärke gemessen.
Die Ergebnisse dieser Messung sind in \ref{tab:2} zu sehen und noch einmal in ? graphisch dargestellt. Diese graphische Darstellung ist erweitert um eine lineare Regression der Form:

\begin{align}
    B(I) = m * I + b.
\end{align}

Diese lineare Regression wurde mit Python durchgeführt. Dabei entstehen folgende Werte für die Parameter m und b

\begin{align}
    m = \SI[separate-uncertainty=true]{12}{m\tesla \per \ampere} \\
    m = \SI[separate-uncertainty=true]{12}{m\tesla}
\end{align}


\section{Messwerte}

\begin{minipage}{\linewidth}
    \begin{table}[H]
        \centering
    \captionof{table}{Messreiehe zur Eichung des Elektromagneten}
    \begin{tabular}{ll}
        \toprule
        Zustand & Lande Faktor \\
        \midrule
        0   & 0 \\
        0.2 & 14 \\
        0.4 & 30 \\
        0.6 & 49 \\
        0.8 & 68 \\
        1   & 90 \\
        1.2 & 109 \\
        1.4 & 129 \\
        1.6 & 149 \\
        1.8 & 169 \\
        2   & 190 \\
        2.4 & 227 \\
        2.8 & 266 \\
        3.2 & 302 \\
        3.6 & 336 \\
        4   & 396 \\
        4.4 & 400 \\
        4.8 & 429 \\
        5.2 & 455 \\
        5.6 & 479 \\
        6   & 500 \\
        6.4 & 516 \\
        6.8 & 530 \\
        7.2 & 543 \\
        7.6 & 555 \\
        8.0 & 566 \\
        8.2 & 569 \\
        \bottomrule   
    \end{tabular}
    
    \label{tab:2}
\end{table}
\end{minipage}