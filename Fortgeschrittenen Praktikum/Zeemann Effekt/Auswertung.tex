\section{Vorbereitungsaufgaben}

\section{Auswertung}

\subsection{Eichung des Elektromagneten}

Für die Eichung des Elektromagneten wird eine Hallsonde verwendet. Diese wird mithilfe eines Stativs zwischen die Polschuhe des Elektromagneten gebracht. Nun wird das Magnetfeld in Abhängigkeit von der Stromstärke gemessen.
Die Ergebnisse dieser Messung sind in ? zu sehen und noch einmal in ? graphisch dargestellt. Diese graphische Darstellung ist erweitert um eine lineare Regression der Form:

\begin{align}
    B(I) = m * I + b.
\end{align}

Diese lineare Regression wurde mit Python durchgeführt. Dabei entstehen folgende Werte für die Parameter m und b

\begin{align}
    m = \SI[separate-uncertainty=true]{12}{m\tesla \per \ampere} \\
    m = \SI[separate-uncertainty=true]{12}{m\tesla}
\end{align}
